% !TeX root = ./thuthesis.tex

% 论文基本信息配置

% 技术方案基本信息配置(简化版)

% 载入所需的宏包

% 定理类环境宏包
\usepackage{amsthm}
% 注意:不使用 amssymb,因为与 unicode-math 冲突
% 可见 unicode-math 包已经提供了所有数学符号

% 定义缺失的数学符号
% unicode-math 环境下重新定义 Box 符号
\renewcommand{\Box}{\square}

% 也可以使用 ntheorem
% \usepackage[amsmath,thmmarks,hyperref]{ntheorem}

% 注意:thuthesis.cls 已经定义了 theorem, proposition, lemma, corollary 等环境
% 所以我们不需要重新定义这些环境

% 添加 TikZ 支持
\usepackage{tikz}
\usetikzlibrary{shapes,arrows,positioning}

% 添加浮动体屏障支持
\usepackage{placeins}

% 添加代码列表支持
\usepackage{listings}
\usepackage{xcolor}

% 定义Python代码高亮颜色
\definecolor{pyComment}{rgb}{0.5,0.5,0.5}
\definecolor{pyKeyword}{rgb}{0,0.5,0}
\definecolor{pyString}{rgb}{0.6,0.1,0.1}

% 定义Python代码样式(附录用,带行号)
\lstdefinestyle{appendixpy}{
    language=Python,
    basicstyle=\footnotesize\ttfamily,
    keywordstyle=\color{pyKeyword}\bfseries,
    commentstyle=\color{pyComment},
    stringstyle=\color{pyString},
    showstringspaces=false,
    numbers=left,
    numberstyle=\tiny\color{gray},
    stepnumber=1,
    tabsize=4,
    breaklines=true,
    breakatwhitespace=true,
    frame=lines,
    rulecolor=\color{black}
}

% 定义正文代码样式(简洁美观,无行号)
\lstdefinestyle{pythoncode}{
    language=Python,
    basicstyle=\small\ttfamily\linespread{1.1}\selectfont,
    keywordstyle=\color{pyKeyword}\bfseries,
    commentstyle=\color{pyComment},
    stringstyle=\color{pyString},
    showstringspaces=false,
    numbers=none,
    tabsize=4,
    breaklines=true,
    breakatwhitespace=true,
    frame=single,
    framesep=8pt,
    xleftmargin=12pt,
    xrightmargin=12pt,
    backgroundcolor=\color{gray!5},
    rulecolor=\color{gray!40},
    belowskip=1em,
    aboveskip=1em,
}

% 设置默认代码样式
\lstset{style=pythoncode}

% 数学字体配置已简化,使用默认设置

% 可以使用 nomencl 生成符号和缩略语说明
% \usepackage{nomencl}
% \makenomenclature

% 表格加脚注
\usepackage{threeparttable}

% 表格中支持跨行
\usepackage{multirow}

% 固定宽度的表格。
% \usepackage{tabularx}

% 跨页表格
\usepackage{longtable}
\usepackage{tabularx}

% 算法
\usepackage{algorithm}
\usepackage{algpseudocode}

% 量和单位
% 注意: 如果使用 unicode-math,需要在 siunitx 之前加载
% \usepackage{siunitx}

% 参考文献使用 BibTeX + natbib 宏包
% 顺序编码制
\usepackage[sort]{natbib}
\bibliographystyle{thuthesis-numeric}
\usepackage{titlesec}

% 让 \paragraph 参与编号(段级别=4)
\setcounter{secnumdepth}{4}

% 定义 \paragraph 的编号样式为纯阿拉伯数字:1, 2, 3, ...
\makeatletter
\renewcommand\theparagraph{\arabic{paragraph}}
% 如果希望每个 section 内从 1 重新开始编号,保留下一行;
% 如果希望全文连续编号,请删掉下一行。
\@addtoreset{paragraph}{section}
\makeatother

% 把 \paragraph 改成块级标题(标题后换行)
\titleformat{\paragraph}[block]
{\normalfont\normalsize\bfseries}% 样式
{\theparagraph}%                  % 显示的编号(比如 1, 2, 3)
{0.8em}%                          % 编号与标题的间距
{}                                % 标题前置代码

% 调整段前段后间距(可按需修改)
\titlespacing*{\paragraph}
{0pt}{3.25ex plus 1ex minus .2ex}{1ex plus .2ex}
\usepackage{indentfirst}  % 首行缩进

% 著者-出版年制
% \usepackage{natbib}
% \bibliographystyle{thuthesis-author-year}

% 生命科学学院要求使用 Cell 参考文献格式(2023 年以前使用 author-date 格式)
% \usepackage{natbib}
% \bibliographystyle{cell}

% 本科生参考文献的著录格式
% \usepackage[sort]{natbib}
% \bibliographystyle{thuthesis-bachelor}

% 参考文献使用 BibLaTeX 宏包
% \usepackage[style=thuthesis-numeric]{biblatex}
% \usepackage[style=thuthesis-author-year]{biblatex}
% \usepackage[style=gb7714-2015]{biblatex}
% \usepackage[style=apa]{biblatex}
% \usepackage[style=mla-new]{biblatex}
% 声明 BibLaTeX 的数据库
% \addbibresource{ref/refs.bib}

% 定义所有的图片文件在 figures 子目录下
\graphicspath{{figures/}}

% 数学命令
\makeatletter
\newcommand\dif{%  % 微分符号
    \mathop{}\!%
    \ifthu@math@style@TeX
        d%
    \else
        \mathrm{d}%
    \fi
}
\makeatother

% hyperref 宏包在最后调用
\usepackage{hyperref}

% cleveref 宏包必须在 hyperref 之后加载
\usepackage[nameinlink, noabbrev]{cleveref}

% ---- 等宽字体安全回退(可选)----
% 若模板因选项/误判试图使用 Menlo 而当前系统不存在,则强制改为 Courier New
\IfFontExistsTF{Menlo}{}{\setmonofont{Courier New}[Scale=MatchLowercase]}
% ---- 结束 ----