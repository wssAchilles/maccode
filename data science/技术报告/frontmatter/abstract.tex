% !TEX root = ../main.tex
\begin{abstract}
    随着全球能源转型和分时电价政策的推广,分布式储能系统的智能化管理成为关键课题。传统电池管理系统(BMS)依赖固定规则策略,难以适应负载波动和电价变化,导致储能系统的经济价值未能充分释放。本项目针对这一问题,设计并实现了一套基于数据驱动的储能智能管理系统,面向高能耗用户(大型别墅、小型社区微网等),构建了"负载预测—优化调度—可视化监控"的完整技术闭环。

    在\textbf{核心算法}层面,本项目解决了两个关键数据科学问题:(1)\textbf{负载时序预测}——采用随机森林(Random Forest)回归模型,融合时间特征(Hour、DayOfWeek)、环境特征(Temperature)和电价特征(Price),预测未来24小时负载序列,实验表明 RMSE 相比历史均值基准\textbf{降低 37.9\%};(2)\textbf{充放电优化调度}——构建混合整数规划(MIP)数学模型,以最小化购电成本为目标,在功率平衡、容量边界、充放电互斥等约束下,使用 Gurobi 求解器在毫秒级内生成最优策略,实现日均电费\textbf{节省 12.5\%}。

    在\textbf{系统工程}层面,本项目构建了完整的端到端应用:后端基于 Flask 框架提供 RESTful API,集成 Firebase 实现用户认证与数据持久化;前端基于 Flutter 开发跨平台移动应用,实现 SOC 实时监控、功率曲线可视化和调度策略展示;系统采用 Docker 容器化封装,部署于 Google App Engine,核心 API 平均响应延迟控制在 300 ms 以内。

    实验结果验证了项目的两个核心假设:随机森林模型显著优于基准方法(RMSE 降低 37.9\% $>$ 20\% 阈值),MIP 优化策略有效降低用电成本(节省率 12.5\% $>$ 10\% 阈值)。本项目为高能耗用户(别墅、社区微网)提供了低门槛的智能管理工具,同时为数据科学教学提供了涵盖数据工程、机器学习、运筹优化和全栈开发的综合实践案例。
    \thusetup{
        keywords = {分布式储能系统, 负载预测, 随机森林, 混合整数规划, 充放电优化, Flutter, 数据驱动},
    }
\end{abstract}
