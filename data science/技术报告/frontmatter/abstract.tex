% !TEX root = ../main.tex
\begin{abstract}
    随着全球能源转型和分时电价政策的推广,分布式储能系统的智能化管理成为关键课题。传统电池管理系统(BMS)依赖固定规则策略,难以适应负载波动和电价变化,导致储能系统的经济价值未能充分释放。本项目针对这一问题,设计并实现了一套基于数据驱动的储能智能管理系统,面向高能耗用户(大型别墅、小型社区微网等),构建了"负载预测—优化调度—可视化监控"的完整技术闭环。

    在\textbf{核心算法}层面,本项目解决了两个关键数据科学问题:(1)\textbf{负载时序预测}——采用随机森林(Random Forest)回归模型,融合时间特征(Hour、DayOfWeek)、环境特征(Temperature)和电价特征(Price),预测未来24小时负载序列,实验表明 RMSE 相比历史均值基准\textbf{降低 37.9\%};(2)\textbf{充放电优化调度}——构建混合整数规划(MIP)数学模型,以最小化购电成本为目标,在功率平衡、容量边界、充放电互斥等约束下,使用 Gurobi 求解器在毫秒级内生成最优策略,实现日均电费\textbf{节省 12.5\%}。

    在\textbf{系统工程}层面,本项目构建了完整的端到端应用:后端基于 Flask 框架提供 RESTful API,集成 Firebase 实现用户认证与数据持久化;前端基于 Flutter 开发跨平台移动应用,实现 SOC 实时监控、功率曲线可视化和调度策略展示;系统采用 Docker 容器化封装,部署于 Google App Engine,核心 API 平均响应延迟控制在 300 ms 以内。

    实验结果验证了项目的两个核心假设:随机森林模型显著优于基准方法(RMSE 降低 37.9\% $>$ 20\% 阈值),MIP 优化策略有效降低用电成本(节省率 12.5\% $>$ 10\% 阈值)。本项目为高能耗用户(别墅、社区微网)提供了低门槛的智能管理工具,同时为数据科学教学提供了涵盖数据工程、机器学习、运筹优化和全栈开发的综合实践案例。
    \thusetup{
        keywords = {分布式储能系统, 负载预测, 随机森林, 混合整数规划, 充放电优化, Flutter, 数据驱动},
    }
\end{abstract}


\begin{abstract*}
    With the global energy transition and the promotion of Time-of-Use (TOU) pricing policies, the intelligent management of distributed energy storage systems has become a critical issue. Traditional Battery Management Systems (BMS) rely on fixed-rule strategies, making them difficult to adapt to load fluctuations and electricity price changes, leading to the underutilization of the storage system's economic value. Addressing this problem, this project designs and implements a data-driven intelligent management system for distributed energy storage. Targeting high-energy consumption users (e.g., large villas, small community microgrids), the system constructs a complete technical closed-loop of ``Load Prediction -- Optimization Scheduling -- Visual Monitoring''.

    In terms of \textbf{Core Algorithms}, this project solves two key data science problems: (1) \textbf{Load Time-Series Prediction}: Utilizing a Random Forest regression model that integrates temporal features (Hour, DayOfWeek), environmental features (Temperature), and pricing features (Price) to predict the load sequence for the next 24 hours. Experiments show that the RMSE is \textbf{reduced by 37.9\%} compared to the historical average baseline. (2) \textbf{Charging/Discharging Optimization Scheduling}: Constructing a Mixed-Integer Programming (MIP) mathematical model aiming to minimize electricity purchase costs. Under constraints such as power balance, capacity limits, and mutual exclusion of charging/discharging, the Gurobi solver is used to generate optimal strategies within milliseconds, achieving a daily average electricity bill \textbf{saving of 12.5\%}.

    In terms of \textbf{System Engineering}, the project builds a complete end-to-end application: The backend provides RESTful APIs based on the Flask framework, integrating Firebase for user authentication and data persistence; the frontend is a cross-platform mobile application developed with Flutter, realizing real-time SOC monitoring, power curve visualization, and scheduling strategy display; the system adopts Docker containerization and is deployed on Google App Engine, with the average response latency of core APIs controlled within 300 ms.

    The experimental results verify the two core hypotheses of the project: the Random Forest model is significantly superior to the baseline method (RMSE reduction 37.9\% $>$ 20\% threshold), and the MIP optimization strategy effectively reduces electricity costs (saving rate 12.5\% $>$ 10\% threshold). This project provides a low-threshold intelligent management tool for high-energy users and offers a comprehensive practical case covering data engineering, machine learning, operations research optimization, and full-stack development for data science education.

    \thusetup{
        keywords* = {Distributed Energy Storage System, Load Forecasting, Random Forest, Mixed-Integer Programming, Charging/Discharging Optimization, Flutter, Data-Driven},
    }
\end{abstract*}


