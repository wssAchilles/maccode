% !TEX root = ../main.tex
\chapter{引言}

\section{项目背景与动机}

\subsection{分布式储能系统的智能管理需求}

分时电价(Time-of-Use, TOU)政策下,峰谷电价差可达2-4倍,为分布式储能系统的"低储高放"套利创造了经济空间。本项目聚焦于\textbf{高能耗用户}——包括大型独栋别墅(High-end Villa)和小型社区微网(Community Microgrid)——这类用户日均用电量显著高于普通住宅,具备配置中大容量储能系统的经济动机。然而,传统电池管理系统(BMS)主要关注单体监测和安全保护,缺乏系统层面的智能优化能力。当前主要痛点包括:

\begin{enumerate}
    \item \textbf{监控能力不足}:用户难以实时掌握 SOC、SOH 等关键指标,缺乏历史趋势分析;
    \item \textbf{策略依赖固定规则}:传统"谷充峰放"策略无法根据负载、电价等因素动态调整;
    \item \textbf{缺乏前瞻性决策}:用户无法获知最优充电时机和预期节省金额;
    \item \textbf{数据价值未挖掘}:历史运行数据未能转化为预测模型和优化策略。
\end{enumerate}

\subsection{为何需要"预测+优化"组合}

本项目构建基于数据驱动的智能储能管理系统,核心技术路线为"机器学习预测 + 数学优化调度"。两类技术互为前提、缺一不可:

\begin{enumerate}
    \item \textbf{优化依赖预测}:MIP模型需要未来24小时负载序列作为输入,若预测不准,优化策略将基于错误假设;
    \item \textbf{预测服务决策}:负载预测本身不产生价值,其意义在于为优化决策提供信息输入;
    \item \textbf{技术互补}:机器学习擅长学习非线性模式但难处理硬约束;数学优化擅长约束下求最优解但依赖准确输入。
\end{enumerate}

\subsection{技术挑战}
尽管“预测+优化”路径理论可行,但在实际工程落地中面临多重挑战:
\begin{enumerate}
    \item \textbf{实时性要求}:移动端用户交互要求系统在 500ms 内完成“数据拉取-预测-优化-响应”全链路,这对后端算法的计算效率提出了极高要求。
    \item \textbf{求解器成本}:商用 MIP 求解器(如 Gurobi)许可证费用昂贵。如何在维持云端高性能运算的同时,通过 WLS(Web License Service)等机制控制部署成本,是规模化应用的关键。
    \item \textbf{数据异构性}:系统需处理千瓦级(kW)的负载数据、摄氏度(°C)的气象数据以及元(RMB/USD)的电价数据,量纲差异大且时间粒度对齐困难。
\end{enumerate}

本项目为高能耗用户(别墅业主、社区物业)提供低门槛的智能管理工具,验证了数据科学方法在能源管理领域的应用价值。

\section{研究目标与问题定义}

\subsection{业务场景}

本项目以高能耗用户储能系统为典型场景。考虑一栋配置储能集群的大型别墅或小型社区微网,采用 \textbf{Tesla Powerwall 储能集群}(总容量 60 kWh、最大功率 20 kW、效率 95\%,相当于 4--5 台 Powerwall 并联),接入分时电价计费方案。

参考加州 PG\&E 费率\cite{pge2024tou},采用简化的三段式电价模型(谷时段 0.3 元/kWh,平时段 0.6 元/kWh,峰时段 1.0 元/kWh)。

\subsection{核心问题定义}

本项目致力于解决两个核心数据科学问题:

\subsubsection{预测问题:负载时序预测}
给定历史负载、温度和时间特征,预测未来 24 小时负载序列。这是解决优化问题的前提输入。

\subsubsection{优化问题:充放电调度}
在给定负载预测、电价和电池参数的约束下,求解最优充放电策略(何时充、何时放、充放多少),以使未来 24 小时的总购电成本最小化。

\subsection{研究目标}

本项目旨在构建一个完整的端到端系统,实现以下目标:

\begin{enumerate}
    \item \textbf{经济目标}:通过优化策略,使日均电费节省率不低于 10\%(假设二);
    \item \textbf{预测目标}:构建高精度负载预测模型,RMSE 相比历史均值法降低 20\% 以上(假设一);
    \item \textbf{工程目标}:实现从数据采集、后端求解到移动端监控的闭环系统,验证“预测+优化”技术路线在实际工程中的可行性。
\end{enumerate}

\subsection{报告组织结构}

本报告其余部分组织如下:\textbf{第二章}对问题进行形式化描述并说明数据来源;\textbf{第三章}详述数据清洗与探索性分析;\textbf{第四章}介绍建模方法与算法设计;\textbf{第五章}展示实验结果与评估;\textbf{第六章}阐述系统工程实现;\textbf{第七章}总结全文与展望。
