% !TEX root = ../main.tex
\chapter{系统架构与工程实现}
\label{chap:system}

\section{系统总体架构}

本系统采用经典的 B/S (Browser/Server) 架构模式,通过分层设计实现关注点分离,确保系统的可扩展性、可维护性和高可用性。图~\ref{fig:system_arch}展示了家庭储能智能管理系统的总体架构。

\begin{figure}[htbp]
    \centering
    \begin{tikzpicture}[
            node distance=0.6cm,
            box/.style={rectangle, draw, rounded corners=3pt, minimum height=0.9cm, text centered, font=\small},
            layer/.style={rectangle, draw, thick, rounded corners=5pt, minimum width=14.5cm, minimum height=2.5cm, fill=white},
            layerlabel/.style={font=\small\bfseries, fill=white, inner sep=2pt},
            arrow/.style={->, thick, >=stealth}
        ]

        % 用户层
        \node[layer, fill=blue!5] (user) at (0, 8) {};
        \node[layerlabel] at (-6.2, 8.9) {用户层 (User Layer)};

        \node[box, fill=blue!20, minimum width=3cm] (login) at (-4.8, 8) {登录页面};
        \node[box, fill=blue!20, minimum width=3cm] (analysis) at (-1.6, 8) {数据分析页面};
        \node[box, fill=blue!20, minimum width=3cm] (optimize) at (1.6, 8) {优化调度页面};
        \node[box, fill=blue!20, minimum width=3cm] (history) at (4.8, 8) {历史记录页面};

        \node[font=\footnotesize] at (0, 7.0) {Flutter Web Application};

        % API 网关层
        \node[layer, fill=green!5] (api) at (0, 5) {};
        \node[layerlabel] at (-6.2, 5.9) {API 网关层 (Gateway)};

        \node[box, fill=green!20, minimum width=2.4cm] (auth) at (-4.8, 5) {/auth};
        \node[box, fill=green!20, minimum width=2.4cm] (data) at (-2.4, 5) {/data};
        \node[box, fill=green!20, minimum width=2.4cm] (anal) at (0, 5) {/analysis};
        \node[box, fill=green!20, minimum width=2.4cm] (opt) at (2.4, 5) {/optimize};
        \node[box, fill=green!20, minimum width=2.4cm] (hist) at (4.8, 5) {/history};

        \node[font=\footnotesize] at (0, 4.0) {Flask REST API (Google App Engine)};

        % 服务层
        \node[layer, fill=orange!5] (service) at (0, 1.8) {};
        \node[layerlabel] at (-6.2, 2.7) {服务层 (Service Layer)};

        \node[box, fill=orange!20, minimum width=4cm, align=center] (mlsvc) at (-4.2, 2) {MLService\\\footnotesize (Scikit-learn)};
        \node[box, fill=orange!20, minimum width=4cm, align=center] (optsvc) at (0, 2) {OptimizationService\\\footnotesize (Gurobi MIP)};
        \node[box, fill=orange!20, minimum width=4cm, align=center] (extsvc) at (4.2, 2) {ExternalDataService\\\footnotesize (CAISO/Weather)};

        % 数据层
        \node[layer, fill=purple!5] (datalayer) at (0, -1.5) {};
        \node[layerlabel] at (-6.2, -0.6) {数据层 (Data Layer)};

        \node[box, fill=purple!20, minimum width=4cm, align=center] (storage) at (-4.2, -1.5) {Cloud Storage\\\footnotesize (文件/模型存储)};
        \node[box, fill=purple!20, minimum width=4cm, align=center] (firestore) at (0, -1.5) {Firestore\\\footnotesize (用户数据/历史)};
        \node[box, fill=purple!20, minimum width=4cm, align=center] (extapi) at (4.2, -1.5) {External APIs\\\footnotesize (CAISO/Weather)};

        % 连接箭头
        \draw[arrow, blue!60] (0, 6.9) -- node[right, font=\footnotesize] {Firebase Auth Token} (0, 6.3);
        \draw[arrow, green!60] (0, 3.9) -- (0, 3.2);
        \draw[arrow, orange!60] (-4.2, 0.8) -- (-4.2, -0.2);
        \draw[arrow, orange!60] (0, 0.8) -- (0, -0.2);
        \draw[arrow, orange!60] (4.2, 0.8) -- (4.2, -0.2);

    \end{tikzpicture}
    \caption{家庭储能智能管理系统总体架构图}
    \label{fig:system_arch}
\end{figure}

如图~\ref{fig:system_arch}所示,系统整体分为三个层次:前端层、后端层和数据层,各层职责明确、松耦合协作。

\subsection{前端层 (Frontend)}

前端层基于 \textbf{Flutter} 框架开发\cite{flutter2023},采用 Dart 语言编写。Flutter 是 Google 推出的跨平台 UI 框架,具有"一套代码、多端运行"的特性,可同时编译为 iOS、Android 和 Web 应用,显著降低了多平台适配的开发成本。

前端层的核心职责包括:
\begin{itemize}
    \item \textbf{数据可视化}:通过 \texttt{fl\_chart} 图表库展示负载预测曲线、储能充放电策略、电价分布等关键指标,支持用户直观理解系统运行状态;
    \item \textbf{用户交互}:提供参数配置界面,允许用户调整储能容量、功率限制、电价方案等优化参数;
    \item \textbf{实时监控}:展示当前负载、温度、电价等实时数据,以及优化策略的执行效果。
\end{itemize}

\subsection{后端层 (Backend)}

后端层基于 \textbf{Python Flask} 框架\cite{flask2023}构建轻量级微服务,托管于 \textbf{Google App Engine (GAE)} 平台。Flask 作为 Python 生态中成熟的 Web 框架,具有灵活、可扩展的特点,便于集成机器学习和优化求解器等计算集型组件。

后端层承担以下核心功能:
\begin{itemize}
    \item \textbf{RESTful API 服务}:对外暴露统一的 HTTP 接口,支持前端数据查询、预测请求和优化任务提交;
    \item \textbf{负载预测推理}:加载训练好的随机森林模型(\texttt{rf\_model.joblib}),根据实时特征进行负载预测;
    \item \textbf{优化决策求解}:调用 Gurobi 求解器执行混合整数规划,生成最优充放电策略;
    \item \textbf{定时任务调度}:通过 APScheduler 实现外部数据(气象、电网负载)的定时采集与同步。
\end{itemize}

Google App Engine 提供了完全托管的 PaaS (Platform as a Service) 环境,支持自动扩缩容、负载均衡和版本管理,使开发者无需关注底层基础设施运维。

\subsection{数据层 (Data \& Infrastructure)}

数据层采用 \textbf{Google Firebase} 生态系统\cite{firebase2023},实现无服务器 (Serverless) 风格的数据管理。Firebase 提供了一套完整的后端即服务 (BaaS) 解决方案,包括:

\begin{itemize}
    \item \textbf{Cloud Firestore}:NoSQL 文档数据库,用于存储用户配置、优化历史记录等结构化数据,支持实时同步和离线缓存;
    \item \textbf{Cloud Storage}:对象存储服务,用于托管训练数据集(CSV 文件)和机器学习模型文件;
    \item \textbf{Firebase Authentication}:身份认证服务,支持邮箱/密码、Google 账号等多种登录方式,确保用户数据隔离与安全访问。
\end{itemize}

Serverless 架构的优势在于按需付费、自动扩展,特别适合本项目这类请求量波动较大的应用场景。

\subsection{技术栈选型}

表~\ref{tab:tech_stack}汇总了本系统各层的核心技术选型及其选用理由。

\begin{table}[htbp]
    \centering
    \caption{系统技术栈选型}
    \label{tab:tech_stack}
    \begin{tabular}{llp{7cm}}
        \toprule
        \textbf{层次} & \textbf{技术/框架}          & \textbf{选型理由}                                             \\
        \midrule
        前端          & Flutter (Dart)          & 跨平台开发,一套代码支持 iOS/Android/Web;\texttt{fl\_chart} 提供丰富的图表组件 \\
        后端          & Flask (Python 3.11)     & 轻量灵活,与 Scikit-learn、Gurobi 无缝集成                           \\
        云平台         & Google App Engine       & 全托管 PaaS,自动扩缩容,免运维                                        \\
        数据库         & Firebase Firestore      & 实时同步,Serverless 架构,按需计费                                   \\
        存储          & Google Cloud Storage    & 高可用对象存储,支持大文件托管                                           \\
        认证          & Firebase Authentication & 开箱即用的身份认证,支持多种登录方式                                        \\
        \bottomrule
    \end{tabular}
\end{table}

该技术栈的组合充分利用了 Google Cloud Platform (GCP) 生态的协同优势:App Engine 与 Firebase 服务之间通过服务账号实现安全的内部通信,无需额外配置网络策略;Python 生态为机器学习和优化求解提供了成熟的工具链支持;Flutter 则确保了用户界面的一致性和开发效率。

\section{后端服务开发}

后端服务是系统的核心枢纽,负责业务逻辑处理、算法推理和数据持久化。本节从 API 接口设计、Firebase 集成和自动化任务调度三个维度,详细阐述后端服务的工程实现。

\subsection{RESTful API 接口设计}

后端采用 RESTful 风格设计 API 接口,遵循"资源导向"原则,通过 HTTP 方法语义化地表达操作意图。表~\ref{tab:api_endpoints}列出了系统的核心接口。

\begin{table}[htbp]
    \centering
    \caption{核心 RESTful API 接口}
    \label{tab:api_endpoints}
    \begin{tabular}{llp{6.5cm}}
        \toprule
        \textbf{方法} & \textbf{端点}              & \textbf{功能描述}                    \\
        \midrule
        POST        & \texttt{/api/predict}    & 负载预测:接收特征向量,返回未来 24 小时负载预测值      \\
        POST        & \texttt{/api/optimize}   & 优化求解:接收负载预测和参数配置,返回最优充放电策略       \\
        GET         & \texttt{/api/history}    & 历史查询:获取指定用户的优化历史记录               \\
        GET         & \texttt{/api/data/live}  & 实时数据:返回当前负载、温度、电价等实时监测数据         \\
        POST        & \texttt{/api/auth/login} & 用户认证:验证 Firebase ID Token,返回用户信息 \\
        \bottomrule
    \end{tabular}
\end{table}

\subsubsection{接口鉴权机制}

系统采用 \textbf{Firebase Authentication} 实现用户身份认证与接口鉴权。认证流程如下:

\begin{enumerate}
    \item 用户在前端通过 Firebase SDK 完成登录(支持邮箱/密码、Google 账号等方式),获取 \textbf{ID Token};
    \item 前端在每次 API 请求时,将 ID Token 置于 HTTP 请求头的 \texttt{Authorization} 字段;
    \item 后端中间件拦截请求,调用 Firebase Admin SDK 的 \texttt{auth.verify\_id\_token()} 方法验证令牌有效性;
    \item 验证通过后,从令牌中解析用户 UID,注入请求上下文供后续业务逻辑使用。
\end{enumerate}

该机制的核心实现位于 \texttt{middleware/} 目录,通过 Python 装饰器模式实现非侵入式鉴权:

\begin{lstlisting}[style=pythoncode]
@app.route('/api/history')
@require_auth  # 鉴权装饰器
def get_history():
    user_id = g.user_id  # 从上下文获取已验证的用户ID
    return history_service.get_user_history(user_id)
\end{lstlisting}

图~\ref{fig:firebase_auth}展示了 Firebase Authentication 控制台的用户管理界面,可直观查看已注册用户及其认证状态。

\begin{figure}[htbp]
    \centering
    \includegraphics[width=0.9\textwidth]{figures/firebase9.png}
    \caption{Firebase 用户认证管理后台}
    \label{fig:firebase_auth}
\end{figure}

\subsection{数据持久化:Firebase 集成}

数据持久化采用 Firebase 生态的两大核心服务:Cloud Firestore(文档数据库)和 Cloud Storage(对象存储),分别承担结构化数据和大文件的存储需求。

\subsubsection{NoSQL 数据库设计:Cloud Firestore}

Cloud Firestore 是 Google 提供的 Serverless NoSQL 文档数据库,具有实时同步、自动扩展和强一致性等特点。本系统设计了以下集合(Collection)结构:

\begin{itemize}
    \item \textbf{users}:用户配置集合,存储每个用户的储能参数偏好(如电池容量、功率限制、电价方案);
    \item \textbf{history}:优化历史集合,记录每次优化任务的输入参数、求解结果和时间戳,支持按用户 UID 索引查询。
\end{itemize}

每条记录以 JSON 文档形式存储,结构灵活,无需预定义 Schema。图~\ref{fig:firestore_structure}展示了 Firestore 控制台中的文档结构示例。

\begin{figure}[htbp]
    \centering
    \includegraphics[width=0.9\textwidth]{figures/firebase5.png}
    \caption{Firestore 数据库文档结构}
    \label{fig:firestore_structure}
\end{figure}

后端通过 \texttt{FirebaseService} 类封装 Firestore 操作,核心方法包括:

\begin{lstlisting}[style=pythoncode]
class FirebaseService:
    def save_optimization_result(self, user_id: str, result: dict):
        """保存优化结果至用户历史记录"""
        doc_ref = self.db.collection('history').document()
        doc_ref.set({
            'user_id': user_id,
            'result': result,
            'created_at': firestore.SERVER_TIMESTAMP
        })
    
    def get_user_history(self, user_id: str, limit: int = 10):
        """查询用户最近的优化历史"""
        query = self.db.collection('history') \
            .where('user_id', '==', user_id) \
            .order_by('created_at', direction='DESCENDING') \
            .limit(limit)
        return [doc.to_dict() for doc in query.stream()]
\end{lstlisting}

\subsubsection{云端对象存储:Cloud Storage}

对于机器学习模型文件(\texttt{.joblib})和大容量训练数据(CSV),系统使用 \textbf{Cloud Storage} 进行托管。相较于 Firestore 的 1MB 文档大小限制,Cloud Storage 支持 TB 级文件存储,更适合此类场景。

图~\ref{fig:firebase_storage}展示了 Cloud Storage 中的模型文件目录结构。

\begin{figure}[htbp]
    \centering
    \includegraphics[width=0.9\textwidth]{figures/firebase1.png}
    \caption{Firebase Storage 云端模型存储}
    \label{fig:firebase_storage}
\end{figure}

\texttt{StorageService} 类封装了文件的上传与下载逻辑,支持:
\begin{itemize}
    \item \textbf{模型热加载}:后端启动时从 Storage 下载最新模型文件至本地缓存,避免将大文件打包进部署镜像;
    \item \textbf{增量数据同步}:通过 \texttt{append\_and\_trim\_csv()} 方法实现实时数据的增量追加,并自动清理过期记录。
\end{itemize}

\subsection{自动化任务调度}

为实现系统的"自我更新"能力,后端集成了两种定时任务调度机制:Google App Engine Cron Jobs 和 APScheduler。

\subsubsection{GAE Cron Jobs}

Google App Engine 提供原生的 Cron 服务,通过 \texttt{cron.yaml} 配置文件定义定时任务。本系统配置了以下任务:

\begin{lstlisting}[style=pythoncode,language={}]
cron:
# 任务 1: 每小时抓取 CAISO 和天气数据
- description: "Hourly data fetching from CAISO and OpenWeather"
  url: /tasks/fetch-data
  schedule: every 1 hours
  timezone: UTC
  retry_parameters:
    min_backoff_seconds: 60
    max_doublings: 5

# 任务 2: 每天凌晨 4:00 UTC 重训模型
- description: "Daily model retraining at 4:00 AM UTC"
  url: /tasks/train-model
  schedule: every day 04:00
  timezone: UTC
  retry_parameters:
    min_backoff_seconds: 120
    max_doublings: 3
\end{lstlisting}

配置说明:
\begin{itemize}
    \item \textbf{数据抓取任务}(\texttt{/tasks/fetch-data}):每小时触发一次,调用 \texttt{ExternalDataService} 从 CAISO 和 OpenWeatherMap 获取最新的电网负载与气象数据,并同步至 Cloud Storage。配置了失败重试机制(最小退避 60 秒,最大翻倍次数 5 次);
    \item \textbf{模型重训练任务}(\texttt{/tasks/train-model}):每日 UTC 时间 04:00(对应北京时间 12:00)触发,使用累积的新数据增量更新随机森林模型,确保预测精度随数据积累持续优化。
\end{itemize}

GAE Cron 服务会自动在请求头中添加 \texttt{X-Appengine-Cron: true} 标识,该标识无法被外部伪造,后端通过检查此请求头确保任务仅由 Cron 服务触发,防止未授权访问。

\subsubsection{APScheduler 进程内调度}

对于需要更细粒度控制的任务(如心跳检测、缓存刷新),系统使用 Python 的 \texttt{APScheduler} 库实现进程内调度。该调度器在 Flask 应用启动时初始化,与 Web 服务共享进程空间,适合轻量级周期任务。

通过 GAE Cron 与 APScheduler 的组合,系统实现了从分钟级到日级的多层次任务调度,确保数据时效性和模型准确性。

\section{移动端应用开发}

本节介绍基于 Flutter 框架开发的移动端应用,涵盖 UI/UX 设计理念、核心功能模块及前端技术实现。

\subsection{UI/UX 设计哲学}

移动端应用采用 Google 官方推出的 \textbf{Material Design 3} 设计语言,该设计体系强调"以用户为中心"的交互体验,通过动态色彩系统 (Dynamic Color)、圆润的组件形态和一致的视觉层次,为用户提供简洁、直观的操作界面。

在布局设计上,应用采用\textbf{卡片式 (Card-based) 布局}作为核心信息容器。卡片组件将相关数据聚合在视觉边界内,便于用户快速识别和理解信息结构。每个功能模块(如电池状态、电价信息、负载预测)均封装为独立卡片,支持响应式布局以适配不同屏幕尺寸。

数据可视化方面,应用集成了 \texttt{fl\_chart} 图表库。该库基于 Flutter 的 Canvas API 实现高性能渲染,支持折线图、柱状图、饼图等多种图表类型,并提供流畅的动画过渡效果和手势交互(如缩放、平移、点击高亮),能够直观呈现时间序列数据和优化结果。

\subsection{核心功能模块}

\subsubsection{用户认证与系统主页}

\begin{figure}[htbp]
    \centering
    \begin{minipage}[b]{0.45\textwidth}
        \centering
        \includegraphics[width=\textwidth]{figures/app0.png}
        \centerline{(a) 用户登录界面}
    \end{minipage}
    \hfill
    \begin{minipage}[b]{0.45\textwidth}
        \centering
        \includegraphics[width=\textwidth]{figures/app1.png}
        \centerline{(b) 系统主页}
    \end{minipage}
    \caption{用户认证与系统主页}
    \label{fig:app_auth_home}
\end{figure}

图~\ref{fig:app_auth_home}展示了应用的入口界面。图~\ref{fig:app_auth_home}(a) 为基于 Firebase Authentication 实现的安全登录页面,支持邮箱/密码认证方式,登录成功后 Firebase 返回的 ID Token 将作为后续 API 请求的身份凭证。图~\ref{fig:app_auth_home}(b) 是用户登录后进入的系统主页 (Dashboard),采用信息密度适中的卡片布局,用户可一目了然地查看当前电池荷电状态 (SOC)、实时电价以及今日能耗概览,实现对家庭储能系统的全局感知。

\subsubsection{智能决策与实时监控}

\begin{figure}[htbp]
    \centering
    \begin{minipage}[b]{0.45\textwidth}
        \centering
        \includegraphics[width=\textwidth]{figures/app3.png}
        \centerline{(a) 负载预测曲线}
    \end{minipage}
    \hfill
    \begin{minipage}[b]{0.45\textwidth}
        \centering
        \includegraphics[width=\textwidth]{figures/app4.png}
        \centerline{(b) 充放电调度计划}
    \end{minipage}
    \caption{实时负载预测与优化调度监控}
    \label{fig:app_optimization}
\end{figure}

图~\ref{fig:app_optimization}展示了应用的核心业务功能——智能决策与实时监控模块。图~\ref{fig:app_optimization}(a) 通过折线图直观呈现未来 24 小时的负载预测曲线,该数据由后端随机森林模型实时推理生成;图~\ref{fig:app_optimization}(b) 展示了 Gurobi 混合整数规划求解器输出的电池充放电计划,蓝色柱状图表示充电功率,橙色柱状图表示放电功率。通过两图的联动展示,用户可以清晰理解系统的"低储高放"策略:在低电价时段(如夜间)充电储能,在高电价时段(如用电高峰)放电供能,从而实现电费成本的最小化。

\subsubsection{数据分析与历史回溯}

\begin{figure}[htbp]
    \centering
    \begin{minipage}[b]{0.45\textwidth}
        \centering
        \includegraphics[width=\textwidth]{figures/app5.png}
        \centerline{(a) 历史能耗趋势}
    \end{minipage}
    \hfill
    \begin{minipage}[b]{0.45\textwidth}
        \centering
        \includegraphics[width=\textwidth]{figures/app6.png}
        \centerline{(b) 详细数据报表}
    \end{minipage}
    \caption{历史数据分析与详细报表}
    \label{fig:app_history}
\end{figure}

图~\ref{fig:app_history}展示了历史数据的统计分析功能,体现了系统的数据沉淀价值。图~\ref{fig:app_history}(a) 以面积图形式呈现过去一周或一个月的能耗趋势,支持时间范围切换和数据点详情查看;图~\ref{fig:app_history}(b) 提供详细的数据列表视图,包含每日的能耗量、费用支出、节省金额等关键指标,用户可滚动浏览历史记录并导出报表。该模块帮助用户量化评估智能调度策略的实际收益,增强对系统的信任度。

\subsection{前端技术实现}

\subsubsection{状态管理架构}

应用采用 \textbf{Provider} 模式实现全局状态管理。Provider 是 Flutter 官方推荐的轻量级状态管理方案,基于 \texttt{InheritedWidget} 机制实现跨组件的数据共享与响应式更新。

系统定义了以下核心状态类:
\begin{itemize}
    \item \texttt{AuthProvider}:管理用户认证状态,包括登录态、用户信息和 Token 刷新;
    \item \texttt{DataProvider}:管理实时数据状态,包括当前负载、电价、温度等时序数据;
    \item \texttt{OptimizationProvider}:管理优化结果状态,包括预测曲线、充放电计划和历史记录。
\end{itemize}

通过 \texttt{MultiProvider} 在应用根节点注入上述状态对象,子组件可通过 \texttt{context.watch<T>()} 订阅状态变化并自动重建 UI,实现数据流的单向传递和视图的响应式更新。

\subsubsection{异步数据交互}

前端通过 RESTful API 与后端 Flask 服务进行数据交互。网络请求封装于 \texttt{ApiService} 类中,使用 Dart 的 \texttt{http} 包发起异步 HTTP 请求。典型的请求流程如下:

\begin{lstlisting}[style=pythoncode,language={}]
// 发起优化请求
Future<OptimizationResult> fetchOptimization() async {
  setState(() => isLoading = true);  // 显示加载指示器
  try {
    final response = await http.get(
      Uri.parse('$baseUrl/api/optimize'),
      headers: {'Authorization': 'Bearer $idToken'},
    );
    if (response.statusCode == 200) {
      return OptimizationResult.fromJson(jsonDecode(response.body));
    } else {
      throw ApiException('请求失败: ${response.statusCode}');
    }
  } catch (e) {
    showErrorSnackBar('网络错误,请检查连接');  // 错误提示
    rethrow;
  } finally {
    setState(() => isLoading = false);  // 隐藏加载指示器
  }
}
\end{lstlisting}

如上述代码所示,应用实现了完整的异步交互模式:
\begin{itemize}
    \item \textbf{加载状态管理}:请求发起时显示 Loading Spinner(圆形进度指示器),请求完成后自动隐藏,避免用户在等待期间产生困惑;
    \item \textbf{错误处理机制}:捕获网络异常和服务端错误,通过 SnackBar 组件向用户展示友好的错误提示信息;
    \item \textbf{Token 认证}:每次请求在 HTTP Header 中携带 Firebase ID Token,后端通过 Token 验证用户身份和权限。
\end{itemize}

通过上述技术实现,移动端应用为用户提供了流畅、可靠的交互体验,将后端的机器学习预测和数学优化能力以直观、易用的方式呈现给终端用户。

\section{容器化与云端部署}

本节介绍系统的容器化构建与云端部署方案,证明系统已具备生产环境的运行能力。

\subsection{Docker 容器化构建}

为确保开发环境与生产环境的一致性,本系统采用 Docker 容器技术对后端服务进行封装。容器化不仅解决了"在我机器上能跑"的经典问题,还为后续的云端部署和弹性扩缩容奠定了基础。

\subsubsection{Dockerfile 编写}

后端服务的 Dockerfile 采用多阶段构建策略,核心配置如下:

\begin{lstlisting}[style=pythoncode,language={}]
# 基础镜像:Python 3.11 精简版
FROM python:3.11-slim

# 设置工作目录
WORKDIR /app

# 安装系统依赖(Gurobi 运行时所需)
RUN apt-get update && apt-get install -y \
    build-essential \
    && rm -rf /var/lib/apt/lists/*

# 复制依赖文件并安装 Python 包
COPY requirements.txt .
RUN pip install --no-cache-dir -r requirements.txt

# 复制应用代码
COPY . .

# 暴露服务端口
EXPOSE 8080

# 启动命令
CMD ["gunicorn", "-b", "0.0.0.0:8080", "main:app"]
\end{lstlisting}

Dockerfile 的设计要点包括:
\begin{itemize}
    \item \textbf{基础镜像选择}:采用 \texttt{python:3.11-slim} 作为基础镜像,相比完整版镜像体积减少约 70\%,加快构建和部署速度;
    \item \textbf{依赖分层安装}:将 \texttt{requirements.txt} 单独复制并安装,利用 Docker 的缓存机制,在代码变更时无需重新安装依赖,显著提升迭代效率;
    \item \textbf{核心依赖}:\texttt{requirements.txt} 包含 \texttt{gurobipy}(Gurobi 优化求解器)、\texttt{scikit-learn}(机器学习库)、\texttt{flask}(Web 框架)等关键组件;
    \item \textbf{端口暴露}:通过 \texttt{EXPOSE 8080} 声明服务端口,与 Google App Engine 的默认端口配置保持一致。
\end{itemize}

\subsubsection{容器化优势}

容器化技术为本系统带来以下核心价值:
\begin{itemize}
    \item \textbf{环境一致性}:开发、测试、生产环境使用相同的容器镜像,消除"环境差异"导致的问题;
    \item \textbf{依赖隔离}:Gurobi、scikit-learn 等库的版本被锁定在容器内,避免与宿主机或其他应用的依赖冲突;
    \item \textbf{快速部署}:容器镜像可在秒级启动,支持快速回滚和蓝绿部署;
    \item \textbf{可移植性}:同一镜像可部署至任何支持 Docker 的云平台(GCP、AWS、Azure)。
\end{itemize}

\subsection{Google App Engine 部署}

后端服务部署于 Google App Engine (GAE) Flexible Environment,该环境支持自定义运行时和容器化部署,兼具 PaaS 的便捷性和 IaaS 的灵活性。

\subsubsection{服务配置}

GAE 服务通过 \texttt{app.yaml} 配置文件定义运行时环境和资源规格:

\begin{lstlisting}[style=pythoncode,language={}]
runtime: python311  # 运行时环境:Python 3.11

# 启动命令:使用 gunicorn 启动 Flask 应用
# --timeout 300: 工作进程超时 5 分钟(适应大文件分析)
# --workers 1: 单工作进程(内存优化)
entrypoint: gunicorn -b :$PORT --timeout 300 --workers 1 main:app

instance_class: F4  # 高性能实例,解决 Pandas/Scikit-learn 内存需求

# 自动扩缩容配置(成本控制)
automatic_scaling:
  min_instances: 0  # 无流量时关闭,不产生费用
  max_instances: 1  # 防止意外流量导致高额账单

# 环境变量(敏感信息已脱敏)
env_variables:
  STORAGE_BUCKET_NAME: "project-id.firebasestorage.app"
  GCP_PROJECT_ID: "project-id"
  GRB_LICENSEID: "xxxxxxx"      # Gurobi 许可证
  OPENWEATHER_API_KEY: "xxxxxx" # 天气 API 密钥
  WEATHER_CITY_LAT: "34.05"     # 洛杉矶纬度
  WEATHER_CITY_LON: "-118.24"   # 洛杉矶经度
\end{lstlisting}

配置要点说明:
\begin{itemize}
    \item \textbf{运行时}:采用 GAE Standard Environment 的 Python 3.11 运行时,相比 Flexible Environment 具有更快的冷启动速度和更低的成本;
    \item \textbf{实例规格}:选用 \texttt{F4} 高性能实例类型,提供充足的 CPU 和内存资源,满足 Pandas 数据处理、Scikit-learn 模型推理和 Gurobi 优化求解的计算需求;
    \item \textbf{启动命令}:使用 Gunicorn WSGI 服务器,配置 5 分钟超时以支持长时间运行的优化任务;
    \item \textbf{成本控制}:通过 \texttt{min\_instances: 0} 实现"按需启动",无流量时自动关闭实例,避免空闲费用;\texttt{max\_instances: 1} 限制最大实例数,防止流量激增导致的意外账单;
    \item \textbf{环境变量}:通过环境变量注入敏感配置(API 密钥、许可证等),避免硬编码,符合安全最佳实践。
\end{itemize}

\subsubsection{CI/CD 部署流程}

系统采用 \texttt{gcloud} 命令行工具实现持续部署。典型的部署流程如下:

\begin{lstlisting}[style=pythoncode,language={}]
# 1. 构建容器镜像并推送至 Google Container Registry
gcloud builds submit --tag gcr.io/PROJECT_ID/energy-backend

# 2. 部署至 App Engine
gcloud app deploy app.yaml --quiet

# 3. 查看部署状态
gcloud app browse
\end{lstlisting}

部署过程中,GAE 自动完成以下操作:(1) 从 Google Container Registry (GCR) 拉取镜像;(2) 创建新版本实例;(3) 健康检查通过后切换流量;(4) 旧版本实例优雅下线。整个过程实现零停机部署,确保服务的高可用性。

\begin{figure}[htbp]
    \centering
    \includegraphics[width=0.85\textwidth]{figures/firebase6.png}
    \caption{Firebase 控制台:云端服务运行状态}
    \label{fig:cloud_deployment}
\end{figure}

图~\ref{fig:cloud_deployment}展示了 Firebase 控制台中的服务运行状态,可以看到 Firestore 数据库、Cloud Storage 存储桶和 Authentication 服务均处于正常运行状态,体现了系统在云端的高可用性。

\subsection{Firebase 集成架构}

本系统采用 Firebase 生态系统作为后端即服务 (BaaS) 基础设施,与 Google App Engine 形成"Serverless + PaaS"的混合云架构。

\subsubsection{Firebase 服务矩阵}

Firebase 在系统中承担以下核心角色:

\begin{table}[htbp]
    \centering
    \caption{Firebase 服务功能矩阵}
    \label{tab:firebase_services}
    \begin{tabular}{lll}
        \toprule
        \textbf{服务}             & \textbf{用途} & \textbf{特性}            \\
        \midrule
        Firebase Authentication & 用户身份认证      & 支持邮箱/OAuth,自动 Token 管理 \\
        Cloud Firestore         & 实时数据存储      & NoSQL 文档数据库,毫秒级同步      \\
        Cloud Storage           & 文件与模型存储     & 对象存储,支持大文件和流式读写        \\
        Firebase Hosting        & 前端静态托管      & 全球 CDN,SSL 证书自动配置      \\
        \bottomrule
    \end{tabular}
\end{table}

\subsubsection{混合架构优势}

"Serverless + PaaS"混合架构为系统带来显著的运维成本优势:
\begin{itemize}
    \item \textbf{零服务器管理}:Firebase 服务完全托管,无需关注底层服务器的配置、扩容和安全补丁;
    \item \textbf{按需计费}:Firestore 按读写次数计费,Storage 按存储量计费,避免资源闲置浪费;
    \item \textbf{全球加速}:Firebase Hosting 内置 CDN,Flutter Web 应用可在全球范围内实现低延迟访问;
    \item \textbf{安全合规}:Firebase 提供企业级安全保障,数据传输全程 TLS 加密,符合 GDPR 等合规要求。
\end{itemize}

\section{本章小结}

本章详细介绍了家庭储能智能管理系统的架构设计与工程实现。从系统总体架构出发,阐述了 B/S 分层架构的设计理念;在后端服务开发中,展示了 Flask RESTful API、Firebase 集成和自动化任务调度的实现细节;在移动端应用开发中,介绍了 Material Design 3 设计语言、核心功能模块和 Provider 状态管理架构;最后,通过 Docker 容器化和 Google App Engine 部署,证明了系统具备生产环境的运行能力。

至此,系统完成了从算法模型到移动端应用的\textbf{全链路打通}:第三章的随机森林预测模型和 Gurobi 优化求解器被封装为后端微服务,通过 RESTful API 对外提供能力;移动端应用以直观的可视化界面呈现预测结果和优化策略;云端部署确保了系统的高可用性和弹性扩展能力。这标志着本项目已形成一个\textbf{可用的、闭环的智能能源管理产品},具备了从原型验证走向实际应用的技术基础。

本章完成了系统的工程构建。下一章将基于真实数据集,对核心算法的预测精度和优化策略的经济效益进行定量评估,验证第一章提出的研究假设。
