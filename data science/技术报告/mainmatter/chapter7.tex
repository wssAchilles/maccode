% !TEX root = ../main.tex
\chapter{总结与展望}

\section{项目核心贡献}
\label{sec:summary}

本项目交付了一套面向高能耗用户(大型别墅、社区微网)的智能调度系统,实现了从数据采集到移动端展示的完整闭环。核心贡献体现在以下三个维度:

\subsection{数据与算法闭环}

系统构建了"数据清洗 $\to$ 预测建模 $\to$ 优化决策"的端到端(End-to-End)处理链路:

\begin{itemize}
    \item \textbf{数据工程}:完成300万条原始记录的ETL流程,输出11,486条高质量小时级样本;
    \item \textbf{负载预测}:随机森林模型RMSE相比历史均值法基准\textbf{降低37.9\%},验证了机器学习在时序预测任务中的有效性;
    \item \textbf{优化调度}:MIP模型在毫秒级内生成全局最优充放电策略,自动发现"谷充峰放"套利模式。
\end{itemize}

\subsection{工程落地能力}

项目突破实验室原型的局限,构建了可部署的产品系统:

\begin{itemize}
    \item \textbf{前端}:Flutter跨平台应用,实现SOC监控、预测曲线、调度策略的可视化展示;
    \item \textbf{后端}:Flask微服务部署于Google App Engine,核心API平均响应延迟控制在300ms以内;
    \item \textbf{数据层}:Firebase Serverless架构实现用户认证、数据持久化和模型存储;
    \item \textbf{容器化}:Docker封装确保环境一致性,支持一键部署。
\end{itemize}

\subsection{商业价值验证}

实验结果证明系统具备真实的经济价值:

\begin{itemize}
    \item \textbf{日均电费节省率12.5\%},超过预设的10\%阈值;
    \item 按储能集群(60kWh/20kW)配置估算,年节省约19,528元;
    \item 敏感性分析表明,当峰谷价差$\geq$0.8元/kWh时,系统具有显著商业化前景。
\end{itemize}

\subsection{工程经验与教训}

在从原型到产品的工程化过程中,我们积累了以下关键经验:

\begin{enumerate}
    \item \textbf{云端状态同步难题}:在早期设计中,试图在前端实时计算优化策略,但因移动端算力限制导致严重卡顿。最终确立了“后端计算、前端渲染”的架构,并通过 Firebase Firestore 的 Snapshot Listener 实现了毫秒级的状态同步。
    \item \textbf{模型冷启动问题}:新用户接入初期因缺乏历史数据导致预测精度骤降。我们引入了基于相似用户的“预训练模型迁移”策略,有效缓解了冷启动前 72 小时的性能瓶颈。
    \item \textbf{许可证管理}:容器化部署初期常因 Gurobi 许可证绑定物理机器 ID 导致扩容失败。引入 WLS (Web License Service) 后,实现了与硬件解耦的弹性伸缩。
\end{enumerate}

\section{项目局限性}
\label{sec:limitations}

\subsection{模型理想化假设}

当前优化模型将电池视为理想设备,存在以下简化:

\begin{itemize}
    \item \textbf{忽略电池老化}:未考虑循环老化(Cycle Aging)对容量的影响,可能导致长期运行成本被低估。
    \item \textbf{固定效率假设}:电池效率$\eta$设为常数0.95,未考虑极端温度下效率下降的物理特性。
\end{itemize}

\subsection{部署依赖性}

系统的生产环境部署存在以下约束:

\begin{itemize}
    \item \textbf{云端依赖}:当前架构需要稳定的互联网连接,离线状态下无法工作;
    \item \textbf{商业许可成本}:Gurobi求解器的商业许可费用较高,对于大规模部署场景是潜在障碍。
\end{itemize}

\section{未来改进方向}
\label{sec:future_work}

针对上述局限性,提出以下改进路径:

\subsection{边缘计算部署}

将推理模型和优化求解迁移至家庭本地设备(如树莓派、Jetson Nano),实现离线自主决策。通过模型量化(Quantization)和使用开源求解器(COIN-OR、Google OR-Tools)消除云端和商业许可依赖。

\subsection{光储一体化}

集成光伏发电预测模块,构建"光伏自用优先、余电存储、不足购电"的多能源协同优化策略,预期将经济效益从当前的12.5\%提升至20\%--30\%。

\subsection{算法演进}

\begin{itemize}
    \item \textbf{深度学习}:引入LSTM或Transformer模型处理长时序依赖,提升多日预测精度;
    \item \textbf{强化学习}:采用DQN/PPO算法,使系统在与环境交互中自适应学习最优策略,增强对预测误差的鲁棒性。
\end{itemize}
