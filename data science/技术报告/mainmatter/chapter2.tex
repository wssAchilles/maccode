% !TEX root = ../main.tex
\chapter{相关工作与技术基础}

\section{数据来源说明}

本项目涉及多源异构数据的融合处理,数据来源可分为静态历史数据和动态实时数据两类。表~\ref{tab:data_sources}概述了各数据源的基本信息。

\begin{table}[htbp]
    \centering
    \caption{数据来源汇总}
    \label{tab:data_sources}
    \begin{tabular}{llll}
        \toprule
        \textbf{数据类型} & \textbf{来源}        & \textbf{时间粒度} & \textbf{用途} \\
        \midrule
        历史负载          & 商业建筑能耗数据集          & 分钟级 $\to$ 小时级 & 模型训练        \\
        电网负载          & CAISO API          & 实时(小时级)       & 特征补充        \\
        气象数据          & OpenWeatherMap API & 实时            & 温度特征        \\
        \bottomrule
    \end{tabular}
\end{table}

\subsection{历史负载数据(静态数据)}

历史负载数据来源于某商业建筑的真实能耗监测系统,时间跨度为 2018 年 7 月至 2019 年 10 月,覆盖 7 层楼共 14 个 CSV 文件,原始记录约 300 万条。

\textbf{数据代理性说明}:虽然原始数据来自商业建筑,但其负载曲线(日均用电 400--500 元,峰值负载 10--15 kW)与大型多层别墅或小型社区微网具有高度相似的波动特性和量级。因此,本项目将该数据集作为高能耗用户的\textbf{代理数据集(Proxy Dataset)},用于模型训练和系统验证。

\textbf{原始数据格式}:每个文件包含分钟级采样的功率读数,列名以 \texttt{kW} 后缀标识功率测点(如 \texttt{HVAC\_kW}、\texttt{Lighting\_kW}),以 \texttt{degC} 后缀标识温度测点。数据处理模块(\texttt{EnergyDataProcessor})自动识别并聚合所有功率列,计算楼层总负载:

\begin{equation}
    L_{\text{floor}}(t) = \sum_{i=1}^{N} P_i(t)
\end{equation}

\noindent 其中 $P_i(t)$ 为第 $i$ 个功率测点在时刻 $t$ 的读数,$N$ 为该楼层的功率测点数。

\textbf{数据预处理}:原始分钟级数据经重采样聚合为小时级数据,压缩比约 60:1,最终得到 11,486 条有效样本。该数据集用于训练随机森林负载预测模型。

\subsection{实时环境与电网数据(动态数据)}

为实现预测模型的在线推理和实时监控,系统集成了两类外部 API 数据源,实现了从静态历史数据到动态实时数据的扩展。本节详细介绍各数据源的技术实现,并通过系统运行截图展示数据采集功能的实际效果。

\subsubsection{气象数据:OpenWeatherMap API}

气象数据通过 OpenWeatherMap 的 Current Weather API 获取,采集地点为洛杉矶(纬度 34.05°N,经度 118.24°W),与历史负载数据的地理位置保持一致。

\textbf{技术实现}:系统使用 Python \texttt{requests} 库发起 HTTP GET 请求,通过经纬度参数定位目标城市,解析 JSON 响应中的 \texttt{main.temp} 字段获取实时温度(摄氏度)。核心调用逻辑如下:

\begin{lstlisting}
params = {'lat': 34.05, 'lon': -118.24, 
          'appid': API_KEY, 'units': 'metric'}
response = requests.get(weather_api_url, params=params)
temperature = response.json()['main']['temp']
\end{lstlisting}

\noindent 其中 \texttt{units='metric'} 参数确保返回值为摄氏度单位,无需额外的单位转换。

\begin{figure}[htbp]
    \centering
    \includegraphics[width=0.85\textwidth]{figures/数据抓取2.png}
    \caption{OpenWeatherMap 气象数据采集日志截图}
    \label{fig:weather_log}
\end{figure}

图~\ref{fig:weather_log}展示了系统后台实时采集的日志输出。从日志中可以观察到:系统成功调用 OpenWeatherMap API 获取了洛杉矶地区的实时温度数据(如 15.6°C),并将其与时间戳一同记录。该温度值作为影响空调负载的关键因素,被纳入预测模型的特征向量,用于捕捉气温变化对建筑能耗的影响。

\subsubsection{电网负载数据:CAISO API}

电网负载数据通过 California Independent System Operator (CAISO) 的公开接口获取,反映加州电网的实时供需状态。该数据可作为区域性负载趋势的参考指标,补充单体建筑负载数据的局限性。

\textbf{技术实现}:系统使用 Python \texttt{gridstatus} 库封装 CAISO API 调用,通过 \texttt{iso.get\_load(date)} 方法获取指定日期的负载数据。该库提供了简洁的高层接口,屏蔽了底层 HTTP 请求和数据解析的复杂性。

\textbf{时区对齐处理}:由于后端服务部署在 Google App Engine(默认 UTC 时区),而 CAISO 数据按太平洋时间 (PST/PDT) 发布,系统在数据获取前进行时区转换,确保请求日期的正确性:

\begin{lstlisting}
pacific_tz = pytz.timezone('America/Los_Angeles')
now_pacific = datetime.now(pacific_tz)
date_str = now_pacific.strftime('%Y-%m-%d')
df = iso.get_load(date=date_str)
\end{lstlisting}

\noindent 获取的时间戳统一转换为 UTC naive datetime 后存储,避免后续处理中的时区混淆。

\begin{figure}[htbp]
    \centering
    \includegraphics[width=0.85\textwidth]{figures/数据抓取1.png}
    \caption{CAISO 电网实时负载数据获取测试}
    \label{fig:caiso_log}
\end{figure}

图~\ref{fig:caiso_log}展示了通过 CAISO 接口获取的实时电网负载数据。从测试输出可以看到:系统成功获取了加州电网的当日负载曲线,数据包含时间戳和对应的负载值(MW 级别)。系统已自动完成时区对齐,将太平洋时间转换为 UTC 时间存储,确保与其他数据源的时间戳一致性。

\textbf{时区对齐的重要性}:时区处理是本项目数据工程中的关键环节。分时电价(Price)是基于\textbf{当地时间}定义的(如 18:00--22:00 为峰时),而外部 API 返回的天气和电网数据通常采用 \textbf{UTC 时间戳}。若时区错位,会导致特征与标签的时间对应关系紊乱——例如将 UTC 时间的 02:00(对应太平洋时间 18:00 峰时段)错误标记为谷时电价 0.3 元/kWh,使模型学习到"高负载时段对应低电价"的错误模式。这种数据污染将导致优化策略完全失效:本应在峰时放电的电池可能被错误调度为充电,不仅无法节省电费,反而增加成本。因此,严格的时区归一化是保证模型有效性和优化策略正确性的前提。

\subsection{数据持久化机制}

本项目通过 \texttt{StorageService} 类实现数据的云端持久化。该服务封装了 Google Cloud Storage API,支持以下功能:

\begin{itemize}
    \item \textbf{凭证管理}:支持通过环境变量 \texttt{GOOGLE\_APPLICATION\_CREDENTIALS}(文件路径)或 \texttt{GCP\_SERVICE\_ACCOUNT\_JSON}(JSON 字符串)加载服务账号凭证,兼容本地开发和云端部署环境;
    \item \textbf{增量写入}:\texttt{append\_and\_trim\_csv()} 方法实现数据的增量追加,并通过滑动窗口机制限制文件大小(默认保留最近 5000 条记录);
    \item \textbf{数据同步}:实时采集的数据(负载、温度、电价)被格式化为统一的 CSV 行格式后追加至云端数据文件。
\end{itemize}

\texttt{ExternalDataService} 的 \texttt{fetch\_and\_publish()} 方法整合了上述数据采集与持久化流程,执行步骤如下:(1) 获取 CAISO 电网负载;(2) 获取 OpenWeatherMap 温度;(3) 数据验证与缺失值填充;(4) 时间对齐与特征构造;(5) 调用 \texttt{StorageService} 持久化。该方法由后端定时任务调度器(\texttt{APScheduler})每小时自动触发。

\subsection{多源数据融合}

本项目的数据架构体现了"多源异构数据融合"的特点:
\begin{itemize}
    \item \textbf{数据异构性}:静态 CSV 文件与动态 API 响应的格式差异,通过统一的 DataFrame 结构进行标准化;
    \item \textbf{时间对齐}:不同数据源的采样时间戳通过 UTC 归一化后对齐至小时级;
    \item \textbf{特征统一}:最终构造的特征向量包含 \texttt{Hour}、\texttt{DayOfWeek}、\texttt{Temperature}、\texttt{Price}、\texttt{Site\_Load} 五个字段,供预测模型和优化模块使用。
\end{itemize}

\section{数据探索性分析}

在构建机器学习模型之前,探索性数据分析(Exploratory Data Analysis, EDA)是理解数据特性、发现潜在规律、指导特征工程的关键步骤。本项目通过 \texttt{AnalysisService} 类实现了系统化的 EDA 流程,涵盖时序特征分析、相关性分析和数据质量评估三个维度。

\subsection{时序特征分析}

建筑能耗数据具有显著的时间依赖性:
\begin{itemize}
    \item \textbf{日周期}:白天(9:00--18:00)负载显著高于夜间,反映办公作息特点,为提取\texttt{Hour}特征提供依据;
    \item \textbf{周周期}:工作日负载均值较周末高约20\%--30\%,\texttt{DayOfWeek}特征捕获此规律;
    \item \textbf{时间聚合}:使用Pandas的\texttt{resample('1H')}将分钟级数据聚合至小时级,在平滑噪声的同时保留周期性结构。
\end{itemize}

\subsection{相关性分析}

相关性分析用于量化特征变量之间的关系。系统的 \texttt{calculate\_correlations()} 方法计算Pearson和Spearman两类相关系数,并通过 $p$ 值判断显著性($p < 0.05$ 时相关性显著)。

\textbf{重点发现}:负载与温度呈显著正相关——温度升高导致空调制冷需求增加。Spearman系数略高于Pearson系数,提示存在非线性关系,为后续采用随机森林等非线性模型提供依据。

\subsection{数据质量评估}

系统的 \texttt{perform\_quality\_check()} 方法从缺失值、异常值和分布特性三方面评估数据质量。

\textbf{缺失值检测}:统计各列缺失率,超过5\%标记为"高缺失风险"。

\textbf{异常值检测}:采用IQR(四分位距)方法,检测超出 $[Q_1 - 1.5 \times \text{IQR}, Q_3 + 1.5 \times \text{IQR}]$ 范围的观测值。

\textbf{分布特性}:计算偏度和峰度,评估数据正态性。通过Shapiro-Wilk或D'Agostino-Pearson检验量化判断($p < 0.05$拒绝正态假设)。

系统综合上述指标计算0--100分的数据质量评分,为预处理优先级提供依据。

\subsection{EDA 对特征工程的指导意义}

探索性分析为特征工程提供了以下指导:
\begin{enumerate}
    \item 日周期和周周期的存在,证明了\texttt{Hour}和\texttt{DayOfWeek}作为特征的合理性;
    \item 负载与温度的显著相关性,支持将\texttt{Temperature}作为核心预测特征;
    \item 非线性关系的存在,为选择随机森林等非线性模型提供了依据。
\end{enumerate}

\section{数据预处理与管道构建}

原始能耗数据需经过系统化的 ETL(Extract-Transform-Load)流程,才能转化为机器学习模型可用的训练数据。本项目通过 \texttt{EnergyDataProcessor} 类实现了完整的数据预处理管道,涵盖数据清洗、多源合并、时间重采样和特征工程四个阶段。图~\ref{fig:etl_pipeline}展示了整体流程。

\begin{figure}[htbp]
    \centering
    \begin{tikzpicture}[
            node distance=1.2cm,
            box/.style={rectangle, draw, rounded corners, minimum width=2.8cm, minimum height=0.8cm, align=center, font=\small},
            arrow/.style={->, >=stealth, thick}
        ]
        \node[box] (raw) {原始 CSV 文件\\(14个文件, 300万条)};
        \node[box, right=of raw] (agg) {功率列聚合\\(\texttt{aggregate})};
        \node[box, right=of agg] (merge) {多楼层合并\\(\texttt{merge})};
        \node[box, below=of merge] (resample) {小时重采样\\(\texttt{resample})};
        \node[box, left=of resample] (feature) {特征工程\\(\texttt{add\_features})};
        \node[box, left=of feature] (output) {输出数据\\(11,486条)};

        \draw[arrow] (raw) -- (agg);
        \draw[arrow] (agg) -- (merge);
        \draw[arrow] (merge) -- (resample);
        \draw[arrow] (resample) -- (feature);
        \draw[arrow] (feature) -- (output);
    \end{tikzpicture}
    \caption{数据预处理 ETL 流程}
    \label{fig:etl_pipeline}
\end{figure}

\subsection{数据清洗与聚合}

原始数据存在多楼层分散存储、列名不统一、时间粒度过细等问题,需通过清洗和聚合操作进行标准化。

\subsubsection{功率列自动识别与聚合}

每个楼层的 CSV 文件包含数十个功率测点(如 \texttt{HVAC\_kW}、\texttt{Lighting\_kW}、\texttt{Plug\_kW} 等),需聚合为单一的楼层总负载。\texttt{aggregate\_power\_columns()} 方法实现了自动化的列识别与聚合:

\begin{lstlisting}
# 自动识别所有包含 'kW' 的功率列
power_columns = [col for col in df.columns if 'kW' in col]
# 按行求和生成楼层总负载
df['Total_Load'] = df[power_columns].sum(axis=1)
\end{lstlisting}

该方法的优势在于:(1) 无需硬编码列名,适应不同楼层的测点配置差异;(2) 通过字符串匹配自动过滤非功率列,避免将温度等无关数据纳入求和。

\subsubsection{多楼层数据合并}

不同楼层的数据文件需按时间戳对齐合并。\texttt{merge\_floors()} 方法采用内连接(Inner Join)策略:

\begin{equation}
    D_{\text{merged}} = D_{\text{F1}} \bowtie_{\text{Date}} D_{\text{F2}} \bowtie_{\text{Date}} \cdots \bowtie_{\text{Date}} D_{\text{F}n}
\end{equation}

\noindent 内连接确保仅保留所有楼层均有记录的时间点,避免因部分楼层数据缺失导致的空值问题。合并后计算全站总负载:

\begin{equation}
    L_{\text{site}}(t) = \sum_{f=1}^{F} L_{\text{floor},f}(t)
\end{equation}

\noindent 其中 $F$ 为楼层总数。本项目中,7 层楼的数据合并后形成统一的 \texttt{Site\_Load} 字段。

\subsubsection{无效数据过滤}

原始数据中存在日期解析异常(如格式错误、空值)的记录。系统在读取阶段使用 \texttt{pd.to\_datetime(errors='coerce')} 将无法解析的日期转换为 \texttt{NaT},随后通过 \texttt{dropna(subset=['Date'])} 删除这些无效行,确保时间索引的完整性。

\subsection{时间重采样}

分钟级数据的高频噪声不利于模型学习长期趋势。\texttt{resample\_to\_hourly()} 方法将数据降采样至小时级:

\begin{lstlisting}
# 设置时间索引并按小时重采样
df = df.set_index('Date')
hourly_df = df.resample('1H').mean()  # 取小时内均值
hourly_df = hourly_df.ffill()          # 前向填充空隙
\end{lstlisting}

重采样策略说明:
\begin{itemize}
    \item \textbf{聚合函数}:采用均值(\texttt{mean})而非求和,保持功率量纲(kW)的物理意义不变;
    \item \textbf{空隙填充}:重采样可能因原始数据缺失产生 \texttt{NaN},使用前向填充(\texttt{ffill})以最近有效值补全,保证时间序列的连续性;
    \item \textbf{压缩比}:从分钟级到小时级,数据量压缩约 60:1,在保留日内周期性的同时显著降低计算开销。
\end{itemize}

\subsection{特征工程}

特征工程将原始数据转化为模型可学习的结构化特征,是连接数据与算法的关键环节。

\subsubsection{时间特征提取}

\texttt{add\_time\_features()} 方法从时间戳中提取两类周期性特征:

\begin{itemize}
    \item \textbf{小时特征} (\texttt{Hour}):取值范围 0--23,捕捉日内周期性(如工作时段与休息时段的负载差异);
    \item \textbf{星期特征} (\texttt{DayOfWeek}):取值范围 0--6(0 表示周一,6 表示周日),捕捉周内周期性。从业务逻辑角度,该特征的核心作用是区分\textbf{工作日模式}(周一至周五,负载较高且日内波动明显)与\textbf{周末模式}(周六周日,负载较低且曲线平缓)。虽然随机森林可直接处理整数编码,但在概念上应将其视为\textbf{有序分类变量}而非连续数值——"周三"与"周四"的差异并非简单的数值递增关系,而是两种不同的用电行为模式。
\end{itemize}

这两个特征使模型能够学习到"周一上午 9 点的负载通常较高"这类时间依赖规律。

\subsubsection{分时电价特征构建}

电价是储能优化的核心输入之一。\texttt{add\_price\_feature()} 方法基于加州分时电价(Time-of-Use, TOU)机制构建电价特征。表~\ref{tab:tou_price}列出了电价分时规则。

\begin{table}[htbp]
    \centering
    \caption{分时电价 (TOU) 规则}
    \label{tab:tou_price}
    \begin{tabular}{llll}
        \toprule
        \textbf{时段类型} & \textbf{时间范围}              & \textbf{电价 (元/kWh)} & \textbf{策略含义} \\
        \midrule
        谷时 (Off-Peak) & 00:00--08:00, 22:00--24:00 & 0.3                 & 储能充电          \\
        平时 (Standard) & 08:00--18:00               & 0.6                 & 正常用电          \\
        峰时 (On-Peak)  & 18:00--22:00               & 1.0                 & 储能放电          \\
        \bottomrule
    \end{tabular}
\end{table}

电价特征的构建逻辑如下:

\begin{lstlisting}
def get_price(hour: int) -> float:
    if 0 <= hour < 8:
        return 0.3   # 谷时
    elif 8 <= hour < 18:
        return 0.6   # 平时
    elif 18 <= hour < 22:
        return 1.0   # 峰时
    else:
        return 0.3   # 谷时 (22:00-24:00)

df['Price'] = df['Hour'].apply(get_price)
\end{lstlisting}

该特征为后续储能优化模块提供了成本计算的关键输入:优化算法将倾向于在谷时(低电价)充电、峰时(高电价)放电,从而最大化套利收益。

\subsection{自动化预处理流水线}

为实现端到端的自动化处理,\texttt{preprocess\_all\_data()} 函数整合了上述所有步骤,形成完整的批处理流水线:(1) 自动扫描 \texttt{data/raw/} 目录发现所有 CSV 文件;(2) 根据文件名前缀提取年份,支持按年份筛选;(3) 对每个文件执行功率聚合、温度提取;(4) 按时间戳内连接所有楼层数据;(5) 重采样为小时级并添加时间和电价特征;(6) 输出至 \texttt{data/processed/} 目录。

该流水线遵循"配置化"原则,数据路径、年份范围均可参数化指定,并在关键步骤输出处理日志,支持问题追溯。

\subsection{最终特征向量}

经过完整的 ETL 流程,最终输出的训练数据包含表~\ref{tab:final_features}所示的特征字段。该特征向量将作为随机森林负载预测模型的输入,其中 \texttt{Site\_Load} 为预测目标变量,其余字段为特征变量。

\begin{table}[htbp]
    \centering
    \caption{预处理后的特征字段}
    \label{tab:final_features}
    \begin{tabular}{lllp{6cm}}
        \toprule
        \textbf{字段名}         & \textbf{类型} & \textbf{取值范围} & \textbf{说明}     \\
        \midrule
        \texttt{Date}        & datetime    & --            & 时间戳(小时级)        \\
        \texttt{Site\_Load}  & float       & $>0$          & 全站总负载 (kW),预测目标 \\
        \texttt{Temperature} & float       & 约 10--40      & 环境温度 (°C)       \\
        \texttt{Hour}        & int         & 0--23         & 小时,周期性特征        \\
        \texttt{DayOfWeek}   & int         & 0--6          & 星期几,周期性特征       \\
        \texttt{Price}       & float       & 0.3/0.6/1.0   & 分时电价 (元/kWh)    \\
        \bottomrule
    \end{tabular}
\end{table}

至此,数据工程阶段的全部工作已完成。经过系统化的 ETL 流程,原始的分散、异构、高噪声数据被转化为结构统一、质量可靠的特征矩阵。这些清洗后的高质量数据为第三章的模型训练奠定了坚实基础——随机森林预测模型将基于上述特征学习负载变化规律,而优化模块则将利用电价特征实现峰谷套利策略。
