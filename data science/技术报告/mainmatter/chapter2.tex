% !TEX root = ../main.tex
\chapter{问题描述与数据说明}

\section{问题形式化表述}

本项目涉及两类核心数学问题:数据驱动的负载预测问题和约束条件下的储能优化问题。

\subsection{负载预测问题}

\subsubsection{预测目标}
设 $L_t$ 表示 $t$ 时刻(小时级)的全屋总负载(单位:kW),对应数据表中的 \texttt{Site\_Load} 字段。预测任务的目标是:给定当前时刻 $t_0$ 的可观测信息(历史负载、天气、时间等),估计未来 24 小时的负载序列 $\{\hat{L}_{t_0+1}, \hat{L}_{t_0+2}, \ldots, \hat{L}_{t_0+24}\}$。

\subsubsection{特征向量}
输入特征向量 $\mathbf{X}_t$ 定义为 4 维向量:
\begin{equation}
    \mathbf{X}_t = \begin{bmatrix} \text{Hour}_t \\ \text{DayOfWeek}_t \\ \text{Temperature}_t \\ \text{Price}_t \end{bmatrix} \in \mathbb{R}^4
    \label{eq:feature_vector}
\end{equation}

\noindent 预测模型本质上是学习映射 $f: \mathbb{R}^4 \to \mathbb{R}$,使得 $\hat{L}_t = f(\mathbf{X}_t)$。

\subsection{储能调度优化问题}

在获得负载预测 $\hat{L}_t$ 后,需制定储能系统的充放电策略。该问题被形式化为混合整数规划(MIP)。

\textbf{决策变量}:
对于未来 24 小时($t=1, \dots, 24$):
\begin{itemize}
    \item $P_{\text{ch}}(t)$:充电功率(kW),连续变量;
    \item $P_{\text{dis}}(t)$:放电功率(kW),连续变量;
    \item $E(t)$:电池储能量(kWh),连续变量;
    \item $z(t)$:充放电状态指示,二进制变量(1表示充电,0表示放电/待机)。
\end{itemize}

\textbf{目标函数}:
最小化未来 24 小时的总购电成本:
\begin{equation}
    \min \sum_{t=1}^{24} p(t) \cdot P_{\text{grid}}(t) = \min \sum_{t=1}^{24} p(t) \cdot \left[ \hat{L}(t) + P_{\text{ch}}(t) - P_{\text{dis}}(t) \right]
    \label{eq:optimization_objective}
\end{equation}
\noindent 其中 $p(t)$ 为该时刻的分时电价。

\textbf{主要约束}:
\begin{align}
     & E(t) = E(t-1) + \eta P_{\text{ch}}(t) - P_{\text{dis}}(t)/\eta & \text{(能量守恒)} \\
     & 0 \leq E(t) \leq E_{\text{cap}}                                & \text{(容量约束)} \\
     & 0 \leq P_{\text{ch}}(t) \leq P_{\max} \cdot z(t)               & \text{(充电互斥)} \\
     & 0 \leq P_{\text{dis}}(t) \leq P_{\max} \cdot (1 - z(t))        & \text{(放电互斥)}
\end{align}
\noindent 其中 $\eta$ 为充放电效率(0.95),$E_{\text{cap}}$ 为电池容量。

\section{数据来源与采集方式}

本项目涉及多源异构数据的融合处理,数据来源可分为静态历史数据和动态实时数据两类。表~\ref{tab:data_sources}概述了各数据源的基本信息。

\begin{table}[htbp]
    \centering
    \caption{数据来源汇总}
    \label{tab:data_sources}
    \begin{tabular}{llll}
        \toprule
        \textbf{数据类型} & \textbf{来源}        & \textbf{时间粒度} & \textbf{用途} \\
        \midrule
        历史负载          & 商业建筑能耗数据集          & 分钟级 $\to$ 小时级 & 模型训练        \\
        电网负载          & CAISO API          & 实时(小时级)       & 特征补充        \\
        气象数据          & OpenWeatherMap API & 实时            & 温度特征        \\
        \bottomrule
    \end{tabular}
\end{table}

\subsection{历史负载数据}

历史负载数据来源于某商业建筑的真实能耗监测系统,时间跨度为 2018 年 7 月至 2019 年 10 月,覆盖 7 层楼共 14 个 CSV 文件,原始记录约 300 万条。虽然原始数据来自商业建筑,但其负载曲线(峰值负载 10--15 kW)与大型多层别墅具有高度相似的波动特性,因此作为本项目的\textbf{代理数据集}。

\subsubsection{数据质量画像}
经初步分析,原始数据存在以下质量问题,这也验证了第三章 ETL 流程的必要性:
\begin{itemize}
    \item \textbf{数据缺失率}:约 1.2\% 的时间戳存在记录缺失,主要由传感器掉线引起。
    \item \textbf{时间对齐偏移}:约 5\% 的楼层数据文件在合并时存在时间戳偏移,需通过重采样对齐。
    \item \textbf{无效日期}:存在少量非标准格式的日期字段(无法解析为 Datetime 对象),需剔除处理。
\end{itemize}

\subsection{实时环境与电网数据}

为实现预测模型的在线推理,系统集成了外部 API。

\subsubsection{气象数据}
通过 OpenWeatherMap API 获取洛杉矶实时温度。系统使用 Python \texttt{requests} 库定时抓取,如图~\ref{fig:weather_log}所示。

\begin{figure}[htbp]
    \centering
    \includegraphics[width=0.85\textwidth]{figures/数据抓取2.png}
    \caption{OpenWeatherMap 气象数据采集日志截图}
    \label{fig:weather_log}
\end{figure}

\subsubsection{电网负载数据}
通过 CAISO(加州独立系统运营商)接口获取实时电网负载,用于捕捉宏观用电趋势。系统处理了 API 返回数据的时区问题,将太平洋时间(PST/PDT)统一转换为 UTC 时间存储。

\begin{figure}[htbp]
    \centering
    \includegraphics[width=0.85\textwidth]{figures/数据抓取1.png}
    \caption{CAISO 电网实时负载数据获取测试}
    \label{fig:caiso_log}
\end{figure}

\section{数据结构与字段说明}

\subsection{多源数据融合}

本项目的数据架构体现了"多源异构数据融合"的特点:
\begin{itemize}
    \item \textbf{数据异构性}:静态 CSV 文件与动态 API 响应的格式差异,通过统一的 DataFrame 结构进行标准化;
    \item \textbf{时间对齐}:不同数据源的采样时间戳通过 UTC 归一化后对齐至小时级;
    \item \textbf{特征统一}:最终构造的特征向量包含 \texttt{Hour}、\texttt{DayOfWeek}、\texttt{Temperature}、\texttt{Price}、\texttt{Site\_Load} 五个字段。
\end{itemize}

\subsection{最终数据表结构}

经过清洗和特征工程(将在第三章详述)后,最终用于模型训练的数据表结构如表~\ref{tab:final_features}所示。

\begin{table}[htbp]
    \centering
    \caption{预处理后的特征字段}
    \label{tab:final_features}
    \begin{tabular}{lllp{6cm}}
        \toprule
        \textbf{字段名}         & \textbf{类型} & \textbf{取值范围} & \textbf{说明}     \\
        \midrule
        \texttt{Date}        & datetime    & --            & 时间戳(小时级)        \\
        \texttt{Site\_Load}  & float       & $>0$          & 全站总负载 (kW),预测目标 \\
        \texttt{Temperature} & float       & 约 10--40      & 环境温度 (°C)       \\
        \texttt{Hour}        & int         & 0--23         & 小时,周期性特征        \\
        \texttt{DayOfWeek}   & int         & 0--6          & 星期几,周期性特征       \\
        \texttt{Price}       & float       & 0.3/0.6/1.0   & 分时电价 (元/kWh)    \\
        \bottomrule
    \end{tabular}
\end{table}
