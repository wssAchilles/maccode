% !TEX root = ../main.tex
\chapter{补充内容}

本附录补充项目实现细节、接口说明与示例代码,并展示系统运行截图与详细日志。

\section{系统运行截图}

\subsection{移动端应用界面}
图~\ref{fig:app_screens}展示了 Flutter 移动端应用的核心功能界面,包括首页概览、实时监控、历史记录等。

\begin{figure}[htbp]
    \centering
    \includegraphics[width=0.3\textwidth]{figures/app0.png}
    \includegraphics[width=0.3\textwidth]{figures/app1.png}
    \includegraphics[width=0.3\textwidth]{figures/app2.png}
    \caption{移动端应用界面:登录与概览}
    \label{fig:app_screens_1}
    \label{fig:app_screens}
\end{figure}

\begin{figure}[htbp]
    \centering
    \includegraphics[width=0.3\textwidth]{figures/app3.png}
    \includegraphics[width=0.3\textwidth]{figures/app4.png}
    \includegraphics[width=0.3\textwidth]{figures/app5.png}
    \caption{移动端应用界面:数据详情与设置}
    \label{fig:app_screens_2}
\end{figure}

\begin{figure}[htbp]
    \centering
    \includegraphics[width=0.45\textwidth]{figures/app6.png}
    \caption{移动端应用界面:关于页面}
    \label{fig:app_screens_3}
\end{figure}

\subsection{Firebase 后台管理}
图~\ref{fig:firebase_console}展示了 Firebase 控制台的数据管理界面。

\begin{figure}[htbp]
    \centering
    \includegraphics[width=0.45\textwidth]{figures/firebase2.png}
    \includegraphics[width=0.45\textwidth]{figures/firebase3.png}
    \caption{Firebase Authentication 用户管理}
    \label{fig:firebase_auth_console}
    \label{fig:firebase_console}
\end{figure}

\begin{figure}[htbp]
    \centering
    \includegraphics[width=0.45\textwidth]{figures/firebase4.png}
    \includegraphics[width=0.45\textwidth]{figures/firebase5.png}
    \caption{Firebase Firestore 数据库文档视图}
    \label{fig:firebase_firestore_console}
\end{figure}

\begin{figure}[htbp]
    \centering
    \includegraphics[width=0.45\textwidth]{figures/firebase6.png}
    \includegraphics[width=0.45\textwidth]{figures/firebase7.png}
    \caption{Firebase Storage 文件存储视图}
    \label{fig:firebase_storage_console}
\end{figure}

\begin{figure}[htbp]
    \centering
    \includegraphics[width=0.45\textwidth]{figures/firebase8.png}
    \includegraphics[width=0.45\textwidth]{figures/firebase9.png}
    \caption{Firebase 监控与规则设置}
    \label{fig:firebase_monitoring}
\end{figure}

\section{部署与运维日志}

\subsection{Google App Engine}
图~\ref{fig:gae_dashboard}展示了 GAE 仪表盘的运行状态。

\begin{figure}[htbp]
    \centering
    \includegraphics[width=0.9\textwidth]{figures/GAE.png}
    \caption{Google App Engine 仪表盘}
    \label{fig:gae_dashboard}
\end{figure}

\begin{figure}[htbp]
    \centering
    \includegraphics[width=0.9\textwidth]{figures/API.png}
    \caption{API 接口调用统计}
    \label{fig:api_stats}
\end{figure}

\subsection{优化求解器日志详情}

\begin{figure}[htbp]
    \centering
    \includegraphics[width=0.9\textwidth]{figures/gurobi4.png}
    \caption{Gurobi 详细求解日志}
    \label{fig:gurobi_detailed_log}
\end{figure}

\section{项目作业展示}
\begin{figure}[htbp]
    \centering
    \includegraphics[width=0.9\textwidth]{figures/作业图.png}
    \caption{课程作业总体设计图}
    \label{fig:assignment_overview}
\end{figure}

\section{后端接口与定时任务}
\subsection{核心 REST API}
\begin{itemize}
    \item \texttt{/api/optimization/run}:输入24小时负载...
\end{itemize}
(后续接口说明保持不变)