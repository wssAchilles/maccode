% !TEX root = ../main.tex
\chapter{实验结果与模型评估}

本章对系统的核心算法进行定量评估,从预测精度和经济效益两个维度展开。

\section{评价指标与实验设置}

\subsection{数据集划分}
实验使用 11,486 条小时级样本,按时间顺序划分为训练集(前80\%,2018.7-2019.6)和测试集(后20\%,2019.7-2019.10)。时间顺序划分防止了未来信息泄露,更贴近真实预测场景。

\subsection{评价指标}
\subsubsection{预测指标}
为了全面评估模型性能,我们采用均方根误差 (RMSE)、平均绝对误差 (MAE) 和决定系数 ($R^2$):
\begin{align}
    \text{RMSE} & = \sqrt{\frac{1}{n} \sum_{i=1}^{n} (y_i - \hat{y}_i)^2}                           \\
    \text{MAE}  & = \frac{1}{n} \sum_{i=1}^{n} |y_i - \hat{y}_i|                                    \\
    R^2         & = 1 - \frac{\sum_{i=1}^{n} (y_i - \hat{y}_i)^2}{\sum_{i=1}^{n} (y_i - \bar{y})^2}
\end{align}

\subsubsection{经济指标}
采用电费节省率评估优化效果:
\begin{equation}
    \text{Saving Rate} = \frac{C_{\text{baseline}} - C_{\text{optimized}}}{C_{\text{baseline}}} \times 100\%
\end{equation}

\subsection{典型日仿真设置}
为了验证优化策略在实际场景中的有效性,我们基于 \texttt{optimization\_service.py} 构建了典型日测试用例。
\begin{itemize}
    \item \textbf{负载曲线}:模拟了典型的商业建筑用电模式,夜间(00:00--05:00)维持在 100kW 左右的基准负载,日间(09:00--17:00)攀升至 300kW 左右的高位负载,晚间(18:00--22:00)存在明显的峰值波动。
    \item \textbf{电价结构}:采用三阶分时电价:谷时 0.3 元/kWh,平时 0.6 元/kWh,峰时 1.0 元/kWh。该价差结构为储能套利提供了理论空间。
\end{itemize}

\subsection{基准模型}
\begin{itemize}
    \item \textbf{预测基准}:历史均值法(预测值 = 24小时前负载);
    \item \textbf{优化基准}:无储能系统(直接购电)。
\end{itemize}

\section{结果展示}

\subsection{负载预测性能}
随机森林模型在测试集上的表现优异,RMSE 为 1,338 kW,MAE 为 985 kW,$R^2$ 达到 0.92。相比历史均值法基准(RMSE 2,156 kW),误差降低了 \textbf{37.9\%}。

\begin{figure}[htbp]
    \centering
    \includegraphics[width=0.85\textwidth]{figures/模型训练1.png}
    \caption{随机森林模型训练日志与评估指标}
    \label{fig:rf_metrics}
\end{figure}

\begin{figure}[htbp]
    \centering
    \includegraphics[width=0.9\textwidth]{figures/模型训练3.png}
    \caption{未来24小时负载预测值与真实值对比}
    \label{fig:prediction_curve}
\end{figure}

如图~\ref{fig:prediction_curve}所示,预测曲线准确捕捉了负载的日内波动趋势,特别是对傍晚高峰和夜间低谷的拟合效果较好。

\subsection{储能优化效益}
基于预测负载生成的充放电策略实现了显著的成本降低(表~\ref{tab:cost_analysis})。
\begin{table}[htbp]
    \centering
    \caption{优化前后电费成本对比}
    \label{tab:cost_analysis}
    \begin{tabular}{lrr}
        \toprule
        \textbf{方案}   & \textbf{日购电成本}   & \textbf{说明}         \\
        \midrule
        无储能基准         & 428.6 元          & 直接购电                \\
        \textbf{优化策略} & \textbf{375.1 元} & \textbf{日节省 12.5\%} \\
        \bottomrule
    \end{tabular}
\end{table}

\section{结果分析与讨论}

\subsection{特征重要性分析}
随机森林的特征重要性分析(图~\ref{fig:feature_importance_eval})表明,\textbf{温度}(Temperature)和\textbf{小时}(Hour)是影响负载变化的最关键因素。这与EDA中发现的“温度相关性”和“日内周期性”规律一致,验证了特征工程的有效性。

\begin{figure}[htbp]
    \centering
    \includegraphics[width=0.85\textwidth]{figures/模型训练2.png}
    \caption{特征重要性排序}
    \label{fig:feature_importance_eval}
\end{figure}

\subsection{优化策略经济性分析}
Gurobi 求解器自动发现了“谷充峰放”的套利模式(图~\ref{fig:schedule_strategy}):
\begin{itemize}
    \item \textbf{峰时放电}:18:00--22:00集中放电,替代1.0元/kWh高价电。
\end{itemize}

\begin{figure}[htbp]
    \centering
    \includegraphics[width=0.9\textwidth]{figures/gurobi2.png}
    \caption{典型日优化充放电调度策略}
    \label{fig:schedule_strategy}
\end{figure}

敏感性分析表明,当峰谷价差超过 0.7 元/kWh 时,投资回收期约为 3.1 年,具备商业可行性。

\begin{figure}[htbp]
    \centering
    \includegraphics[width=0.9\textwidth]{figures/gurobi3.png}
    \caption{优化前后单日电费成本对比}
    \label{fig:cost_comparison}
\end{figure}

\subsection{能量吞吐与物理特性分析}
除了经济指标,我们还对系统的物理运行特性进行了分析。在典型日测试中:
\begin{itemize}
    \item \textbf{总充电量}:45.6 kWh(集中在凌晨谷时段);
    \item \textbf{总放电量}:41.2 kWh(集中在晚间峰时段);
    \item \textbf{系统循环效率}:计算得 $\eta_{\text{cycle}} = \frac{41.2}{45.6} \approx 90.4\%$,符合锂电池组包含充放电损耗(双向 $\eta=0.95$)的物理特性。
\end{itemize}
这表明优化调度不仅经济可行,而且在物理能量守恒层面是严谨可靠的。

\subsection{系统性能分析}
除了算法效果,系统的工程性能也满足实时性要求:
\begin{itemize}
    \item \textbf{求解效率}:MIP 模型包含120个变量,Gurobi可在毫秒级内收敛;
    \item \textbf{响应延迟}:核心 API(预测+优化)平均响应时间约 158 ms,满足移动端交互需求。
\end{itemize}

\begin{figure}[htbp]
    \centering
    \includegraphics[width=0.85\textwidth]{figures/gurobi1.png}
    \caption{Gurobi 求解器运行日志与收敛时间}
    \label{fig:gurobi_log}
\end{figure}