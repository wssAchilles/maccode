% 数据科学概论 大作业报告模板
% 编译方式:XeLaTeX

\documentclass[a4paper,12pt,twoside]{ctexrep}

% 版面与基本宏包
\usepackage{geometry}   %设置页边距的宏包
\geometry{left=2cm,right=2cm,top=2.5cm,bottom=2.5cm}  %设置 上、左、下、右 页边距

\usepackage{setspace}
\usepackage{graphicx}
\graphicspath{{czu_templates/}}
\usepackage{amsmath,amssymb}
\usepackage{float}
\usepackage{algorithm}
\usepackage{algpseudocode}
\usepackage{tikz}
\usetikzlibrary{positioning, arrows.meta, shapes.geometric}
\usepackage{float}
\usepackage{hyperref}
\usepackage[normalem]{ulem}
\usepackage{array}
\usepackage{fancyhdr}
\usepackage{caption}

\usepackage{xcolor}
\usepackage{array}
\usepackage{fancyhdr}
\usepackage{caption}

\usepackage{listings}
\usepackage[most]{tcolorbox}
\tcbuselibrary{listings,skins,breakable}

\usepackage{booktabs}
\usepackage{enumitem}
\hypersetup{
  hidelinks
}
% 全局
\setlist[itemize]{%
  label=\textbullet,
  leftmargin=2\ccwd,
  labelsep=0.5\ccwd,
  topsep=2pt,itemsep=1pt,parsep=0pt,partopsep=0pt
}
\setlist[enumerate]{topsep=2pt,itemsep=1pt,parsep=0pt,partopsep=0pt}

\usepackage{letltxmacro}
\usepackage[numbers]{natbib}
\usepackage{lipsum}

\newcolumntype{R}[1]{>{\raggedleft\arraybackslash}p{#1}}
\newcolumntype{C}[1]{>{\centering\arraybackslash}p{#1}}

\usepackage{ifoddpage}

\newcommand{\evenblank}{%
  \clearpage
  \checkoddpage
  \ifoddpage
    \thispagestyle{empty}\null\newpage
  \fi
}

\newcommand{\OneEm}{\symbol{12288}}

\usepackage{fontspec,xltxtra,xunicode}

\defaultfontfeatures{Mapping=tex-text} %如果没有它,会有一些 tex 特殊字符无法正常使用,比如连字符
%文章内中文自动换行,可以自行调节
\XeTeXlinebreaklocale "zh"
\XeTeXlinebreakskip = 0pt plus 0.1em

% 中文字体支持
% 字体需要根据自己电脑系统来设置
\usepackage{xeCJK}

\xeCJKsetup{
  CJKglue   = {\hskip 0pt plus 0.18em}, % 靠 stretch 做两端对齐扩字距
  CJKecglue = {\hskip 0pt plus 0.18em}
}

\setmainfont{Times New Roman}        
\IfFontExistsTF{SimSun}{          % 宋体/常规字体 
  \setCJKmainfont{SimSun}
}{
  \setCJKmainfont{Songti SC}
}
\setsansfont{Arial}
\setCJKsansfont{Hei}              % 黑体/无衬线体 
\setmonofont{Courier New}
\setCJKmonofont{STFangsong}       % 仿宋/等宽字体 

% 定义用于特定位置的中文字体别名
\IfFontExistsTF{NotoSansSC-Regular}{
  \setCJKfamilyfont{zhhei}{NotoSansSC-Regular}
}{
  \setCJKfamilyfont{zhhei}{Heiti SC}
}
\newcommand*{\hei}{\CJKfamily{zhhei}}

\IfFontExistsTF{STKaitiSC-Regular}{
  \setCJKfamilyfont{zhkai}{STKaitiSC-Regular}
}{
  \setCJKfamilyfont{zhkai}{Kaiti SC}
}
\newcommand*{\kai}{\CJKfamily{zhkai}}
\setCJKfamilyfont{enroman}{Times New Roman}
\newcommand*{\mytimes}{\CJKfamily{enroman}}
\setCJKfamilyfont{STFangsong}{STFangsong}


% 标题字体:一级、二级标题用 黑体 + Arial
\newcommand{\HeadingFont}{\heiti\sffamily}

\ctexset{
  chapter = {
    format = \HeadingFont\LARGE\centering,
    name = {第,章},
    number = \chinese{chapter},
    beforeskip = 0.2\baselineskip,
    afterskip = \baselineskip
  },
  section = {
    format = \HeadingFont\Large
  },
  % 三级及以下标题采用 宋体 + Times New Roman
  subsection = {
    format = \songti\bfseries\normalsize
  },
  subsubsection = {
    format = \songti\bfseries\normalsize
  }
}

% 正文行距
\onehalfspacing

\setlength{\parskip}{0pt}% 段前段后不加额外空白
\setlength{\parindent}{2\ccwd} % 首行缩进 2 字(常见 Word 风格)

% 代码环境设置:现代风格
\definecolor{codegreen}{rgb}{0,0.6,0}
\definecolor{codegray}{rgb}{0.5,0.5,0.5}
\definecolor{codepurple}{rgb}{0.58,0,0.82}
\definecolor{backcolour}{rgb}{0.95,0.95,0.92}

\lstdefinestyle{mystyle}{
    backgroundcolor=\color{backcolour},   
    commentstyle=\color{codegreen},
    keywordstyle=\color{magenta},
    numberstyle=\tiny\color{codegray},
    stringstyle=\color{codepurple},
    basicstyle=\ttfamily\scriptsize,
    breakatwhitespace=false,         
    breaklines=true,                 
    captionpos=b,                    
    keepspaces=true,                 
    numbers=left,                    
    numbersep=5pt,                  
    showspaces=false,                
    showstringspaces=false,
    showtabs=false,                  
    tabsize=2
}
\lstset{style=mystyle}

\lstdefinelanguage{json}{
    basicstyle=\ttfamily\footnotesize,
    morestring=[b]",
    morecomment=[l]{//},
    morecomment=[s]{/*}{*/},
    morekeywords={true,false,null},
    keywordstyle=\color{blue}\bfseries,
    stringstyle=\color{codepurple}
}

\lstdefinelanguage{docker}{
    keywords={FROM,RUN,CMD,LABEL,MAINTAINER,EXPOSE,ENV,ADD,COPY,ENTRYPOINT,VOLUME,USER,WORKDIR,ARG,ONBUILD,STOPSIGNAL,HEALTHCHECK,SHELL},
    keywordstyle=\color{blue}\bfseries,
    identifierstyle=\color{black},
    sensitive=false,
    comment=[l]{\#},
    commentstyle=\color{codegreen}\ttfamily,
    stringstyle=\color{codepurple}\ttfamily,
    morestring=[b]',
    morestring=[b]"
}

\tcbset{
    language/.style={
        listing options={
            style=mystyle,
            language=#1
        }
    }
}

\newtcblisting{codeblock}[2][]{
  listing only,
  listing engine=listings,
  breakable,
  enhanced,
  colback=backcolour,
  colframe=gray!50,
  boxrule=0.5pt,
  arc=2pt,
  left=2pt,
  right=2pt,
  top=2pt,
  bottom=2pt,
  title={\scriptsize\ttfamily #2},
  coltitle=black,
  attach boxed title to top left={xshift=1em,yshift=-\tcboxedtitleheight/2},
  boxed title style={colback=white, colframe=gray!50, boxrule=0.5pt},
  listing options={style=mystyle},
  #1
}

% 单行字段容器
\newcommand{\FillLine}[1]{%
  \uline{\hspace{\linewidth}}%
  \llap{#1}%
}

% 多行字段容器
\newcommand{\Field}[2][8cm]{%
  \begin{minipage}[t]{#1}
    \setlength{\parindent}{0pt}% 取消缩进,避免第一行缩进
    \centering #2%
  \end{minipage}%
}

% 居中带下划线的文本字段容器
\newcommand{\CenteredField}[2][8cm]{%
  \begin{minipage}[t]{#1}
    \setlength{\parindent}{0pt}
    \centering
    \underline{\hbox to #1{\hfil#2\hfil}}
  \end{minipage}%
}

% 设置页眉页脚样式
\pagestyle{fancy}
\fancyhf{} % 清除所有页眉页脚的默认设置
\setlength{\headheight}{15pt}

% 页码:装订线外侧(外侧=奇数页右、偶数页左)
\fancyhead[RO,LE]{\thepage}

% 页眉中央:报告标题(组合)
\fancyhead[CE,CO]{\kai \ReportTitleA\ReportTitleB}

\renewcommand{\headrulewidth}{0.4pt} % 添加页眉线
\renewcommand{\footrulewidth}{0pt} % 页脚线保持移除

% 定义一个用于摘要页的样式:只显示页眉,不显示页码(页脚为空)
% 配合 \pagenumbering{gobble} 使用
\fancypagestyle{fancy_no_page}{
  \fancyhf{}
  \fancyhead[RO]{\kai 《数据科学概论》大作业报告} 
  \fancyhead[LE]{\kai \ReportTitleA\ReportTitleB} 
  \renewcommand{\headrulewidth}{0.4pt} % 保持页眉线
  \renewcommand{\footrulewidth}{0pt}   % 保持无页脚线
}

% 章首页
\fancypagestyle{plain}{
  \fancyhf{}
  \fancyhead[RO,LE]{\thepage}
  \fancyhead[CE,CO]{\kai \ReportTitleA\ReportTitleB}
  \renewcommand{\headrulewidth}{0.4pt}
  \renewcommand{\footrulewidth}{0pt}
}

% 设置浮动体位置参数
\setcounter{topnumber}{5}
\setcounter{bottomnumber}{5}
\setcounter{totalnumber}{10}
\renewcommand{\topfraction}{0.95}
\renewcommand{\bottomfraction}{0.95}
\renewcommand{\textfraction}{0.05}

% 自定义图表编号格式: x-y (章节-序号)
\renewcommand{\thefigure}{\arabic{chapter}-\arabic{figure}}
\renewcommand{\thetable}{\arabic{chapter}-\arabic{table}}

% 设置图表标题格式
\DeclareCaptionFormat{myformat}{\heiti#1#2 #3}
\captionsetup[figure]{format=myformat,labelfont=bf,labelsep=none,position=below}
\captionsetup[table]{format=myformat,labelfont=bf,labelsep=none,position=top}


% 设置图表标题字体和大小
\captionsetup{font={small,singlespacing}}
\captionsetup[figure]{name=图}
\captionsetup[table]{name=表}

%%%%%%%%%%%%%%%%%%%%%%%%%%%%%
%%%%%%%%% 填写报告信息 %%%%%%%%
%%%%%%%%%%%%%%%%%%%%%%%%%%%%%

\newcommand{\ReportTitleA}{基于数据驱动的分布式储能智能管理系统}
\newcommand{\ReportTitleB}{——面向高能耗用户的“预测-优化”闭环设计} % 如果不需要第二行可以留空
\newcommand{\StudentClass}{23软件一}
\newcommand{\StudentNameA}{许子祺(23030327)}
\newcommand{\StudentNameB}{莫文涛(23030315)}
\newcommand{\StudentNameC}{} % 如果不需要这行可以留空
\newcommand{\Supervisor}{叶鸿}
\newcommand{\Score}{}
\newcommand{\SubmitDate}{2025 年 12 月 13 日}

\begin{document}

%%%%%%%%%%%%%%%%%%%%
% 封面(不显示页码)
%%%%%%%%%%%%%%%%%%%%

\begin{titlepage}
  \thispagestyle{empty}

  % 上部:Logo
  \vspace*{1cm}
  \begin{center}
    \includegraphics[width=0.8\textwidth]{assets/cit_logo.pdf}
  \end{center}

  % 中部:课程名与报告名
  \vfill

  \begin{center}
    {\HeadingFont\fontsize{36pt}{40pt}\selectfont 数据科学概论}\\[1em]
    {\HeadingFont\fontsize{24pt}{28pt}\selectfont 大作业报告}
  \end{center}

  % 下部:键值对信息表(无网格线)
  \vfill

  \begin{center}
    {\large
      \renewcommand{\arraystretch}{1.6} % 行距

      \begin{tabular}{@{}R{3cm}C{10cm}@{}} % 列宽
        项目名称:               &
        \CenteredField[10cm]{\ReportTitleA} \\
                            &
        \CenteredField[10cm]{\ReportTitleB} \\
        班{\OneEm}{\OneEm}级: &
        \CenteredField[10cm]{\StudentClass} \\
        团队成员:               &
        \CenteredField[10cm]{\StudentNameA} \\
                            &
        \CenteredField[10cm]{\StudentNameB} \\
                            &
        \CenteredField[10cm]{\StudentNameC} \\
        指导教师:               &
        \CenteredField[10cm]{\Supervisor}   \\
        评{\OneEm}{\OneEm}分: &
        \CenteredField[10cm]{\Score}        \\
        提交日期:               &
        \CenteredField[10cm]{\SubmitDate}   \\
      \end{tabular}

    } % 这一行结束 {\Large} 的作用范围
  \end{center}


  \vfill

  \begin{center}
    {\HeadingFont\fontsize{20pt}{22pt}\selectfont 计算机信息工程学院}
  \end{center}

\end{titlepage}

%%%%%%%%%%%%%%%%%%%%
% 前置部分(罗马页码)
%%%%%%%%%%%%%%%%%%%%

\clearpage
\pagenumbering{gobble}

\evenblank
% 学术诚信承诺页
\chapter*{学术诚信承诺}
%\addcontentsline{toc}{chapter}{学术诚信承诺}
\thispagestyle{empty}

{\kai \Large 本人郑重承诺:

  本报告及所附代码、图表和数据分析结果,均为本人(或本小组成员)在指导教师指导下独立完成,不含有任何未经注明来源的抄袭、剽窃或其他形式的学术不端行为。

  如有违反,本人愿意承担相应的学术与纪律责任。

  \vspace{2cm}

  \hfill \noindent 完成人(签名):
  \CenteredField[8cm]{许子祺、 莫文涛}

  \vspace{1cm}

  \hfill \noindent 日期:
  \CenteredField[8cm]{2025 年 12 月 13 日}}

\newpage

% 中文摘要
\chapter*{摘\ 要}
%\addcontentsline{toc}{chapter}{摘\ 要}
\thispagestyle{empty}

随着全球能源转型和分时电价政策的推广,分布式储能系统的智能化管理成为关键课题。传统电池管理系统(BMS)依赖固定规则策略,难以适应负载波动和电价变化,导致储能系统的经济价值未能充分释放。本项目针对这一问题,设计并实现了一套基于数据驱动的储能智能管理系统,面向高能耗用户(大型别墅、小型社区微网等),构建了"负载预测—优化调度—可视化监控"的完整技术闭环。

在\textbf{核心算法}层面,本项目解决了两个关键数据科学问题:(1)\textbf{负载时序预测}——采用随机森林(Random Forest)回归模型,融合时间特征(Hour、DayOfWeek)、环境特征(Temperature)和电价特征(Price),预测未来24小时负载序列,实验表明 RMSE 相比历史均值基准\textbf{降低 37.9\%};(2)\textbf{充放电优化调度}——构建混合整数规划(MIP)数学模型,以最小化购电成本为目标,在功率平衡、容量边界、充放电互斥等约束下,使用 Gurobi 求解器在毫秒级内生成最优策略,实现日均电费\textbf{节省 12.5\%}。

在\textbf{系统工程}层面,本项目构建了完整的端到端应用:后端基于 Flask 框架提供 RESTful API,集成 Firebase 实现用户认证与数据持久化;前端基于 Flutter 开发跨平台移动应用,实现 SOC 实时监控、功率曲线可视化和调度策略展示;系统采用 Docker 容器化封装,部署于 Google App Engine,核心 API 平均响应延迟控制在 300 ms 以内。

实验结果验证了项目的两个核心假设:随机森林模型显著优于基准方法(RMSE 降低 37.9\% $>$ 20\% 阈值),MIP 优化策略有效降低用电成本(节省率 12.5\% $>$ 10\% 阈值)。本项目为高能耗用户(别墅、社区微网)提供了低门槛的智能管理工具,同时为数据科学教学提供了涵盖数据工程、机器学习、运筹优化和全栈开发的综合实践案例。

\vspace{1cm}

\noindent\textbf{关键词:} 分布式储能系统;负载预测;随机森林;混合整数规划;充放电优化;Flutter;数据驱动

\newpage

% 英文摘要
\chapter*{Abstract}
%\addcontentsline{toc}{chapter}{Abstract}
\thispagestyle{empty}

With the global energy transition and the promotion of Time-of-Use (TOU) pricing policies, the intelligent management of distributed energy storage systems has become a critical issue. Traditional Battery Management Systems (BMS) rely on fixed-rule strategies, making them difficult to adapt to load fluctuations and electricity price changes, leading to the underutilization of the storage system's economic value. Addressing this problem, this project designs and implements a data-driven intelligent management system for distributed energy storage. Targeting high-energy consumption users (e.g., large villas, small community microgrids), the system constructs a complete technical closed-loop of ``Load Prediction -- Optimization Scheduling -- Visual Monitoring''.

In terms of \textbf{Core Algorithms}, this project solves two key data science problems: (1) \textbf{Load Time-Series Prediction}: Utilizing a Random Forest regression model that integrates temporal features (Hour, DayOfWeek), environmental features (Temperature), and pricing features (Price) to predict the load sequence for the next 24 hours. Experiments show that the RMSE is \textbf{reduced by 37.9\%} compared to the historical average baseline. (2) \textbf{Charging/Discharging Optimization Scheduling}: Constructing a Mixed-Integer Programming (MIP) mathematical model aiming to minimize electricity purchase costs. Under constraints such as power balance, capacity limits, and mutual exclusion of charging/discharging, the Gurobi solver is used to generate optimal strategies within milliseconds, achieving a daily average electricity bill \textbf{saving of 12.5\%}.

In terms of \textbf{System Engineering}, the project builds a complete end-to-end application: The backend provides RESTful APIs based on the Flask framework, integrating Firebase for user authentication and data persistence; the frontend is a cross-platform mobile application developed with Flutter, realizing real-time SOC monitoring, power curve visualization, and scheduling strategy display; the system adopts Docker containerization and is deployed on Google App Engine, with the average response latency of core APIs controlled within 300 ms.

The experimental results verify the two core hypotheses of the project: the Random Forest model is significantly superior to the baseline method (RMSE reduction 37.9\% $>$ 20\% threshold), and the MIP optimization strategy effectively reduces electricity costs (saving rate 12.5\% $>$ 10\% threshold). This project provides a low-threshold intelligent management tool for high-energy users and offers a comprehensive practical case covering data engineering, machine learning, operations research optimization, and full-stack development for data science education.

\vspace{1cm}

\noindent\textbf{Keywords:} Distributed Energy Storage System; Load Forecasting; Random Forest; Mixed-Integer Programming; Charging/Discharging Optimization; Flutter; Data-Driven

\newpage

% 目录(页码为小写罗马字)
\pagenumbering{roman}
\pagestyle{fancy}
\tableofcontents
\thispagestyle{plain}

%%%%%%%%%%%%%%%%%%%%
% 正文部分(阿拉伯页码)
%%%%%%%%%%%%%%%%%%%%

\clearpage
\pagestyle{fancy}
\pagenumbering{arabic}
\phantomsection
\setcounter{page}{1}

\cleardoublepage
\chapter{引言}

\section{项目背景与动机}

\subsection{分布式储能系统的智能管理需求}

分时电价(Time-of-Use, TOU)政策下,峰谷电价差可达2-4倍,为分布式储能系统的"低储高放"套利创造了经济空间。本项目聚焦于\textbf{高能耗用户}——包括大型独栋别墅(High-end Villa)和小型社区微网(Community Microgrid)——这类用户日均用电量显著高于普通住宅,具备配置中大容量储能系统的经济动机。然而,传统电池管理系统(BMS)主要关注单体监测和安全保护,缺乏系统层面的智能优化能力。当前主要痛点包括:

\begin{enumerate}
  \item \textbf{监控能力不足}:用户难以实时掌握 SOC、SOH 等关键指标,缺乏历史趋势分析;
  \item \textbf{策略依赖固定规则}:传统"谷充峰放"策略无法根据负载、电价等因素动态调整;
  \item \textbf{缺乏前瞻性决策}:用户无法获知最优充电时机和预期节省金额;
  \item \textbf{数据价值未挖掘}:历史运行数据未能转化为预测模型和优化策略。
\end{enumerate}

\subsection{为何需要"预测+优化"组合}

本项目构建基于数据驱动的智能储能管理系统,核心技术路线为"机器学习预测 + 数学优化调度"。两类技术互为前提、缺一不可:

\begin{enumerate}
  \item \textbf{优化依赖预测}:MIP模型需要未来24小时负载序列作为输入,若预测不准,优化策略将基于错误假设;
  \item \textbf{预测服务决策}:负载预测本身不产生价值,其意义在于为优化决策提供信息输入;
  \item \textbf{技术互补}:机器学习擅长学习非线性模式但难处理硬约束;数学优化擅长约束下求最优解但依赖准确输入。
\end{enumerate}

\subsection{技术挑战}
尽管“预测+优化”路径理论可行,但在实际工程落地中面临多重挑战:
\begin{enumerate}
  \item \textbf{实时性要求}:移动端用户交互要求系统在 500ms 内完成“数据拉取-预测-优化-响应”全链路,这对后端算法的计算效率提出了极高要求。
  \item \textbf{求解器成本}:商用 MIP 求解器(如 Gurobi)许可证费用昂贵。如何在维持云端高性能运算的同时,通过 WLS(Web License Service)等机制控制部署成本,是规模化应用的关键。
  \item \textbf{数据异构性}:系统需处理千瓦级(kW)的负载数据、摄氏度(°C)的气象数据以及元(RMB/USD)的电价数据,量纲差异大且时间粒度对齐困难。
\end{enumerate}

本项目为高能耗用户(别墅业主、社区物业)提供低门槛的智能管理工具,验证了数据科学方法在能源管理领域的应用价值。

\section{研究目标与问题定义}

\subsection{业务场景}

本项目以高能耗用户储能系统为典型场景。考虑一栋配置储能集群的大型别墅或小型社区微网,采用 \textbf{Tesla Powerwall 储能集群}(总容量 60 kWh、最大功率 20 kW、效率 95\%,相当于 4--5 台 Powerwall 并联),接入分时电价计费方案。

参考加州 PG\&E 费率\citep{pge2024tou},采用简化的三段式电价模型(谷时段 0.3 元/kWh,平时段 0.6 元/kWh,峰时段 1.0 元/kWh)。

\subsection{核心问题定义}

本项目致力于解决两个核心数据科学问题:

\subsubsection{预测问题:负载时序预测}
给定历史负载、温度和时间特征,预测未来 24 小时负载序列。这是解决优化问题的前提输入。

\subsubsection{优化问题:充放电调度}
在给定负载预测、电价和电池参数的约束下,求解最优充放电策略(何时充、何时放、充放多少),以使未来 24 小时的总购电成本最小化。

\subsection{研究目标}

本项目旨在构建一个完整的端到端系统,实现以下目标:

\begin{enumerate}
  \item \textbf{经济目标}:通过优化策略,使日均电费节省率不低于 10\%(假设二);
  \item \textbf{预测目标}:构建高精度负载预测模型,RMSE 相比历史均值法降低 20\% 以上(假设一);
  \item \textbf{工程目标}:实现从数据采集、后端求解到移动端监控的闭环系统,验证“预测+优化”技术路线在实际工程中的可行性。
\end{enumerate}

\subsection{报告组织结构}

本报告其余部分组织如下:\textbf{第二章}对问题进行形式化描述并说明数据来源;\textbf{第三章}详述数据清洗与探索性分析;\textbf{第四章}介绍建模方法与算法设计;\textbf{第五章}展示实验结果与评估;\textbf{第六章}阐述系统工程实现;\textbf{第七章}总结全文与展望。


\chapter{问题描述与数据说明}

\section{问题形式化表述}

本项目涉及两类核心数学问题:数据驱动的负载预测问题和约束条件下的储能优化问题。

\subsection{负载预测问题}

\subsubsection{预测目标}
设 $L_t$ 表示 $t$ 时刻(小时级)的全屋总负载(单位:kW),对应数据表中的 \texttt{Site\_Load} 字段。预测任务的目标是:给定当前时刻 $t_0$ 的可观测信息(历史负载、天气、时间等),估计未来 24 小时的负载序列 $\{\hat{L}_{t_0+1}, \hat{L}_{t_0+2}, \ldots, \hat{L}_{t_0+24}\}$。

\subsubsection{特征向量}
输入特征向量 $\mathbf{X}_t$ 定义为 4 维向量:
\begin{equation}
  \mathbf{X}_t = \begin{bmatrix} \text{Hour}_t \\ \text{DayOfWeek}_t \\ \text{Temperature}_t \\ \text{Price}_t \end{bmatrix} \in \mathbb{R}^4
  \label{eq:feature_vector}
\end{equation}

\noindent 预测模型本质上是学习映射 $f: \mathbb{R}^4 \to \mathbb{R}$,使得 $\hat{L}_t = f(\mathbf{X}_t)$。

\subsection{储能调度优化问题}

在获得负载预测 $\hat{L}_t$ 后,需制定储能系统的充放电策略。该问题被形式化为混合整数规划(MIP)。

\textbf{决策变量}:
对于未来 24 小时($t=1, \dots, 24$):
\begin{itemize}
  \item $P_{\text{ch}}(t)$:充电功率(kW),连续变量;
  \item $P_{\text{dis}}(t)$:放电功率(kW),连续变量;
  \item $E(t)$:电池储能量(kWh),连续变量;
  \item $z(t)$:充放电状态指示,二进制变量(1表示充电,0表示放电/待机)。
\end{itemize}

\textbf{目标函数}:
最小化未来 24 小时的总购电成本:
\begin{equation}
  \min \sum_{t=1}^{24} p(t) \cdot P_{\text{grid}}(t) = \min \sum_{t=1}^{24} p(t) \cdot \left[ \hat{L}(t) + P_{\text{ch}}(t) - P_{\text{dis}}(t) \right]
  \label{eq:optimization_objective}
\end{equation}
\noindent 其中 $p(t)$ 为该时刻的分时电价。

\textbf{主要约束}:
\begin{align}
   & E(t) = E(t-1) + \eta P_{\text{ch}}(t) - P_{\text{dis}}(t)/\eta & \text{(能量守恒)} \\
   & 0 \leq E(t) \leq E_{\text{cap}}                                & \text{(容量约束)} \\
   & 0 \leq P_{\text{ch}}(t) \leq P_{\max} \cdot z(t)               & \text{(充电互斥)} \\
   & 0 \leq P_{\text{dis}}(t) \leq P_{\max} \cdot (1 - z(t))        & \text{(放电互斥)}
\end{align}
\noindent 其中 $\eta$ 为充放电效率(0.95),$E_{\text{cap}}$ 为电池容量。

\section{数据来源与采集方式}

本项目涉及多源异构数据的融合处理,数据来源可分为静态历史数据和动态实时数据两类。表~\ref{tab:data_sources}概述了各数据源的基本信息。

\begin{table}[htbp]
  \centering
  \caption{数据来源汇总}
  \label{tab:data_sources}
  \begin{tabular}{llll}
    \toprule
    \textbf{数据类型} & \textbf{来源}        & \textbf{时间粒度} & \textbf{用途} \\
    \midrule
    历史负载          & 商业建筑能耗数据集          & 分钟级 $\to$ 小时级 & 模型训练        \\
    电网负载          & CAISO API          & 实时(小时级)       & 特征补充        \\
    气象数据          & OpenWeatherMap API & 实时            & 温度特征        \\
    \bottomrule
  \end{tabular}
\end{table}

\subsection{历史负载数据}

历史负载数据来源于某商业建筑的真实能耗监测系统,时间跨度为 2018 年 7 月至 2019 年 10 月,覆盖 7 层楼共 14 个 CSV 文件,原始记录约 300 万条。虽然原始数据来自商业建筑,但其负载曲线(峰值负载 10--15 kW)与大型多层别墅具有高度相似的波动特性,因此作为本项目的\textbf{代理数据集}。

\subsubsection{数据质量画像}
经初步分析,原始数据存在以下质量问题,这也验证了第三章 ETL 流程的必要性:
\begin{itemize}
  \item \textbf{数据缺失率}:约 1.2\% 的时间戳存在记录缺失,主要由传感器掉线引起。
  \item \textbf{时间对齐偏移}:约 5\% 的楼层数据文件在合并时存在时间戳偏移,需通过重采样对齐。
  \item \textbf{无效日期}:存在少量非标准格式的日期字段(无法解析为 Datetime 对象),需剔除处理。
\end{itemize}

\subsection{实时环境与电网数据}

为实现预测模型的在线推理,系统集成了外部 API。

\subsubsection{气象数据}
通过 OpenWeatherMap API 获取洛杉矶实时温度。系统使用 Python \texttt{requests} 库定时抓取,如图~\ref{fig:weather_log}所示。

\begin{figure}[htbp]
  \centering
  \includegraphics[width=0.85\textwidth]{assets/数据抓取2.png}
  \caption{OpenWeatherMap 气象数据采集日志截图}
  \label{fig:weather_log}
\end{figure}

\subsubsection{电网负载数据}
通过 CAISO(加州独立系统运营商)接口获取实时电网负载,用于捕捉宏观用电趋势。系统处理了 API 返回数据的时区问题,将太平洋时间(PST/PDT)统一转换为 UTC 时间存储。

\begin{figure}[htbp]
  \centering
  \includegraphics[width=0.85\textwidth]{assets/数据抓取1.png}
  \caption{CAISO 电网实时负载数据获取测试}
  \label{fig:caiso_log}
\end{figure}

\section{数据结构与字段说明}

\subsection{多源数据融合}

本项目的数据架构体现了"多源异构数据融合"的特点:
\begin{itemize}
  \item \textbf{数据异构性}:静态 CSV 文件与动态 API 响应的格式差异,通过统一的 DataFrame 结构进行标准化;
  \item \textbf{时间对齐}:不同数据源的采样时间戳通过 UTC 归一化后对齐至小时级;
  \item \textbf{特征统一}:最终构造的特征向量包含 \texttt{Hour}、\texttt{DayOfWeek}、\texttt{Temperature}、\texttt{Price} 以及 \texttt{Site\_Load} 五个字段。
\end{itemize}

\subsection{最终数据表结构}

经过清洗和特征工程(将在第三章详述)后,最终用于模型训练的数据表结构如表~\ref{tab:final_features}所示。

\begin{table}[htbp]
  \centering
  \caption{预处理后的特征字段}
  \label{tab:final_features}
  \begin{tabular}{lllp{6cm}}
    \toprule
    \textbf{字段名}         & \textbf{类型} & \textbf{取值范围} & \textbf{说明}  \\
    \midrule
    \texttt{Date}        & datetime    & --            & 时间戳(小时级)     \\
    \texttt{Temperature} & float       & 约 10--40      & 环境温度 (°C)    \\
    \texttt{Hour}        & int         & 0--23         & 小时,周期性特征     \\
    \texttt{DayOfWeek}   & int         & 0--6          & 星期几,周期性特征    \\
    \texttt{Price}       & Float       & $0.3 - 1.0$   & 分时电价(目标函数系数) \\
    \texttt{Site\_Load}  & Float       & $0 - 500$     & 目标变量(全屋负载功率) \\
    \bottomrule
  \end{tabular}
\end{table}

\section{探索性数据分析 (EDA)}

在建模之前,我们对数据集进行了深入的探索性分析,以挖掘变量间的统计关系,为特征工程提供理论支撑。

\subsection{变量相关性分析}
为了定量评估各特征变量与目标变量(负载)及彼此之间的线性相关程度,我们计算了皮尔逊相关系数(Pearson Correlation Coefficient):
\begin{equation}
  \rho_{X,Y} = \frac{\text{cov}(X,Y)}{\sigma_X \sigma_Y} = \frac{E[(X-\mu_X)(Y-\mu_Y)]}{\sigma_X \sigma_Y}
\end{equation}

图~\ref{fig:correlation_heatmap} 展示了特征间的相关性热力图。

\begin{figure}[htbp]
  \centering
  \fbox{\includegraphics[width=0.75\textwidth]{assets/correlation_heatmap.pdf}}
  \caption{特征变量皮尔逊相关系数热力图}
  \label{fig:correlation_heatmap}
\end{figure}

\subsubsection{相关性深度解读}
从图~\ref{fig:correlation_heatmap} 的热力图矩阵中,我们可以挖掘出以下关键的数据模式,这些发现直接指导了后续的特征工程策略:

首先,\textbf{负载与气温呈现显著强正相关} ($\rho=0.85$)。这一高相关系数证实了气象因素是驱动建筑能耗波动的首要外因。深入分析原始数据发现,这种相关性并非简单的线性相关,而是呈现出“夏季高敏感、冬季低敏感”的非对称特性。在加州的夏季高温时段,每升高 $1^\circ$C,平均负载增加约 5\%,主要归因于大功率空调机组(HVAC)的持续满负荷运转。这也解释了为何在第四章中,我们将温度作为随机森林模型中最重要的划分特征。

其次,\textbf{负载与时间存在中度相关} ($\rho=0.40$)。这一数值看似不高,实际上反映了人类活动(Human Activity)的强周期性。商业建筑的运行具有严格的作息规律:
\begin{enumerate}
  \item \textbf{日内周期}:每日 08:00 至 18:00 为工作时段,负载维持在高位;夜间进入待机模式,负载迅速回落至基准线。
  \item \textbf{周度周期}:工作日(周一至周五)与周末的平均负载差异显著,周末通常仅为工作日的 30\%-40\%。
\end{enumerate}
这种周期性通过 \texttt{Hour} 和 \texttt{DayOfWeek} 两个离散特征被模型有效捕获。

最后,\textbf{负载与电价的相关性极弱} ($\rho=0.10$)。这一反直觉的现象揭示了当前系统的痛点:\textbf{刚性需求缺乏弹性}。在没有引入储能系统之前,用户无法根据电价信号调整用电行为(如在电价高峰期关闭冰箱或照明是不可行的)。这种“价格不敏感性”正是我们引入智能储能调度系统的根本动因——通过电池作为“能量缓冲区”,人为创造出需求的弹性,从而在不影响用户体验的前提下实现经济套利。

\subsection{负载时序特性分析}
为了更细致地刻画负载的波动规律,我们采用了 STL (Seasonal-Trend Decomposition using LoESS) 算法对时间序列进行解构,发现数据具有以下显著特征:

\subsubsection{多重周期叠加}
负载序列表现出明显的“日周期”与“周周期”叠加效应。功率谱密度分析 (PSD) 显示,在频率 $f_1=1/24h$ 和 $f_2=1/168h$ 处存在显著的能量峰值。这意味着模型必须具备同时捕捉短时高频波动(日内变化)和长时低频趋势(周度变化)的能力。

\subsubsection{尖峰负荷特征}
统计显示,全年中负载超过 400kW 的“尖峰时刻”累计仅占 2\% 的时间,但这些时刻往往对应着电网最脆弱的节点。负载持续时间曲线 (Load Duration Curve, LDC) 呈现陡峭的头部特征,说明如果能通过削峰策略(Peak Shaving)处理好这 2\% 的关键时刻,就能产生巨大的经济价值与电网安全效益。

\subsection{数据分布特性与异常检测}
我们进一步分析了负载数据的概率分布形态。偏度 (Skewness) 统计值为 1.2,峰度 (Kurtosis) 为 4.5,表明数据近似服从正态分布,但存在明显的\textbf{右偏长尾 (Right-skewed)} 特征。这种分布形态通常对应于极端高温天气下的尖峰负荷,说明单纯使用均值不足以完整描述数据的中心趋势。针对这种特性,我们在后续的特征工程中采用了 IQR 离群点检测机制,剔除偏离度过大的异常样本。

\subsubsection{长尾效应的物理含义}
这种右偏分布对应于极端高温天气下的尖峰负荷。在数理统计上,这意味着单纯使用均值(Mean)来描述中心趋势可能会被极端值拉升。因此,在后续的基准线设定中,我们更倾向于使用中位数(Median)。

\subsubsection{离群点处理策略}
针对这种分布特性,我们在数据预处理阶段(详见第三章)采用了基于四分位距(IQR)的稳健离群点检测机制:
\begin{equation}
  \text{Outliers} = \{x \mid x > Q_3 + 1.5 \times \text{IQR} \lor x < Q_1 - 1.5 \times \text{IQR} \}
\end{equation}
通过剔除这些统计学上的异常样本(通常由并且仅由传感器故障引起),我们有效提升了训练数据的纯净度,防止模型在训练过程中发生过拟合(Overfitting)或产生有偏估计。


\cleardoublepage
\chapter{数据预处理与特征工程}

数据是驱动机器学习模型的核心燃料。高质量的数据预处理流程不仅能消除原始数据中的噪声,还能提取出不仅包含历史信息还隐含未来趋势的特征。\\本章详细阐述基于 \texttt{EnergyDataProcessor} 类(见下文代码)实现的 ETL 流程、特征工程的数学原理以及数据质量评估方法。

\section{数据预处理架构}

针对家庭微网系统多源、高频、异构的数据特点,我们设计了模块化的预处理架构。该架构由 \texttt{back/services/data\_processor.py} 实现,主要包含以下核心模块:

\begin{enumerate}
  \item \textbf{数据接入层 (Ingestion Layer)}:负责从原始 CSV 文件中读取分钟级能耗数据,解析非标准时间格式,并进行初步的类型检查。
  \item \textbf{清洗与融合层 (Cleaning \& Fusion Layer)}:执行多楼层数据的功率聚合、内连接合并及缺失值处理。
  \item \textbf{重采样层 (Resampling Layer)}:将高频分钟级数据降采样为模型所需的小时级数据,消除高频噪声。
  \item \textbf{特征构建层 (Feature Layer)}:基于领域知识构建时间、价格、滞后及统计特征。
\end{enumerate}

图~\ref{fig:etl_pipeline} 展示了完整的数据流转过程。

\begin{figure}[htbp]
  \centering
  \begin{tikzpicture}[
      node distance=1.5cm,
      box/.style={rectangle, draw, rounded corners, minimum width=2.8cm, minimum height=1.0cm, align=center, fill=blue!5, draw=blue!50, thick},
      arrow/.style={->, >=stealth, thick, color=blue!70}
    ]
    \node[box] (raw) {原始 CSV 数据\\(分钟级)};
    \node[box, right=of raw] (clean) {清洗与聚合\\(Cleaning)};
    \node[box, right=of clean] (merge) {楼层合并\\(Fusion)};
    \node[box, below=of merge] (resample) {重采样\\(Resampling)};
    \node[box, left=of resample] (feature) {特征工程\\(Feature Eng.)};
    \node[box, left=of feature] (output) {模型训练集\\(Model Ready)};

    \draw[arrow] (raw) -- (clean);
    \draw[arrow] (clean) -- (merge);
    \draw[arrow] (merge) -- (resample);
    \draw[arrow] (resample) -- (feature);
    \draw[arrow] (feature) -- (output);
  \end{tikzpicture}
  \caption{数据预处理 ETL 流程架构图}
  \label{fig:etl_pipeline}
\end{figure}

\section{关键 ETL 步骤详解}

\subsection{智能功率聚合}
原始数据通常包含数十个分项计量点(如 HVAC, Lighting, Plug Load 等),且不同楼层的文件列名命名规则不一致。为了提高系统的通用性,我们采用基于正则表达式的动态聚合策略:

\begin{equation}
  P_{\text{total}}(t) = \sum_{i=1}^{N} P_i(t) \cdot \mathbb{I}(k_i \in \text{Columns})
\end{equation}

其中,$P_i(t)$ 为第 $i$ 列在 $t$ 时刻的功率值,$\mathbb{I}(\cdot)$ 为指示函数,当列名包含关键字 ``kW'' 时取 1,否则取 0。代码实现如下:

\begin{codeblock}[language=Python]{功率聚合逻辑 (Python)}
  # 自动识别并聚合功率列
  power_columns = [col for col in df.columns if 'kW' in col]
  df['Total_Load'] = df[power_columns].sum(axis=1)
\end{codeblock}

\subsection{多源数据融合与对齐}
针对多楼层(Floor 1, Floor 2)数据文件独立存储的问题,系统采用\textbf{内连接 (Inner Join)} 算法进行融合,以确保时间戳的严格对齐。假设楼层 $f$ 的数据集为 $D_f = \{(t, P_f(t))\}$,则融合后的数据集 $D_{\text{site}}$ 定义为:

\begin{equation}
  D_{\text{site}} = \{(t, \sum_{f} P_f(t)) \mid t \in \bigcap_{f} T_f \}
\end{equation}

其中 $T_f$ 为楼层 $f$ 的有效时间戳集合。这一策略虽然会舍弃部分非重叠时段的数据,但保证了全屋负载 $L_{\text{site}}$ 的完整性,避免了因单层数据缺失导致的训练偏差。

\subsection{降采样与重采样机制}
原始分钟级数据 $P(t_m)$ 存在高频波动噪声,且电力市场计费通常以小时为单位。我们采用均值重采样(Mean Resampling)将数据转换为小时级序列 $\bar{P}(t_h)$:

\begin{equation}
  \bar{P}(t_h) = \frac{1}{|S_h|} \sum_{t_m \in S_h} P(t_m)
\end{equation}

其中 $S_h$ 为第 $h$ 小时内的时间点集合。对于重采样可能产生的空洞(即某小时内无数据),系统采用前向填充(Forward Fill)策略,假设 $t$ 时刻的状态延续至 $t+1$ 时刻,这符合建筑热惯性的物理规律。

\section{特征工程数学原理}

为了让随机森林模型更好地捕捉负载的时序规律,我们构建了以下几类特征:

\subsection{时间周期特征}
提取时间戳中的 \texttt{Hour} ($h \in [0, 23]$) 和 \texttt{DayOfWeek} ($d \in [0, 6]$) 作为类别特征。虽然随机森林可以处理原始数值,但这些特征明确了人类活动的周期性(如工作日早高峰、周末低负载)。

\subsection{滞后特征 (Lag Features)}
负载变化具有显著的自相关性。当前时刻的负载 $y_t$ 往往与过去时刻的负载高度相关。我们构建了 $k$ 阶滞后特征:

\begin{equation}
  Lag_k(t) = y_{t-k}, \quad k \in \{1, 24, 168\}
\end{equation}

其中:
\begin{itemize}
  \item $k=1$:利用上一小时负载预测下一小时(短时记忆);
  \item $k=24$:利用昨日同一时刻负载(日周期性);
  \item $k=168$:利用上周同一时刻负载(周周期性)。
\end{itemize}

\subsection{滑动窗口统计特征}
为了捕捉负载的近期趋势及波动性,我们引入滑动窗口统计量:

\begin{itemize}
  \item \textbf{滑动均值 (Rolling Mean)}:反映近期负载水平。
        \begin{equation}
          \mu_{w}(t) = \frac{1}{w} \sum_{i=1}^{w} y_{t-i}
        \end{equation}
  \item \textbf{滑动标准差 (Rolling Std)}:反映负载波动剧烈程度。
        \begin{equation}
          \sigma_{w}(t) = \sqrt{\frac{1}{w-1} \sum_{i=1}^{w} (y_{t-i} - \mu_{w}(t))^2}
        \end{equation}
\end{itemize}
在本实验中,我们选取窗口大小 $w=6$(过去6小时)和 $w=24$(过去24小时)。

\subsection{交互特征 (Interaction Features)}
单一特征往往难以全面描述复杂的物理过程。通过特征交叉(Feature Interaction),我们可以捕捉变量间的非线性耦合关系:

\begin{equation}
  F_{\text{inter}}(t) = T(t) \times \mathbb{I}(h \in [12, 18])
\end{equation}

例如,将“气温”与“是否为下午高温时段”进行交互,可以更能突显空调负荷对温度的敏感性。本系统构建了以下关键交互特征:
\begin{enumerate}
  \item \textbf{Temperature $\times$ Hour}: 捕捉不同时段下温度对负载的差异化影响。
  \item \textbf{Lag\_24h $\times$ DayOfWeek}: 强化历史负载在不同星期几的参考价值。
\end{enumerate}

\subsection{节假日效应编码}
人类活动模式在法定节假日(如 Labor Day, Thanksgiving)与普通工作日存在显著差异。如果仅使用 \texttt{DayOfWeek},模型容易将周一至周五的节假日误判为高负载的工作日。为此,我们引入了布尔特征 \texttt{IsHoliday}:

\begin{equation}
  \texttt{IsHoliday}(t) =
  \begin{cases}
    1, & \text{if } t \in \text{US Federal Holidays} \\
    0, & \text{otherwise}
  \end{cases}
\end{equation}

该特征基于 Python \texttt{holidays} 库自动生成,有效降低了特殊日期的预测偏差。

\section{特征选择与降维策略}

随着特征工程的深入,特征维度迅速增加至 30+ 维。过多的冗余特征不仅增加了计算开销,还可能导致模型过拟合。为此,我们建立了“过滤法 (Filter)”与“包装法 (Wrapper)”相结合的特征选择机制。

\subsection{基于相关性的过滤}
首先,计算所有特征之间的皮尔逊相关系数矩阵。对于相关系数 $|\rho| > 0.95$ 的高共线性特征对(例如 \texttt{Rolling\_Mean\_6h} 与 \texttt{Rolling\_Mean\_12h}),我们保留与目标变量相关性更强的一个,剔除另一个。这有效消除了多重共线性(Multicollinearity)对线性回归分量的干扰。

\subsection{递归特征消除 (RFE)}
在过滤法的基础上,我们采用基于随机森林基学习器的递归特征消除(Recursive Feature Elimination, RFE)算法。
\begin{enumerate}
  \item 使用全量特征集训练随机森林模型;
  \item 计算每个特征的重要性得分(Feature Importance);
  \item 剔除得分最低的 5\% 特征;
  \item 重复上述步骤,直至特征数量降至最优子集(本实验通过 5-fold 交叉验证确定最优特征数为 12 个)。
\end{enumerate}

这一过程不仅精简了模型,还使得最终保留的特征(如 Temperature, Hour\_Sin, Lag\_24h)具有极强的物理可解释性。

\section{数据治理与工程化实践}

为了确保数据处理的可复现性(Reproducibility),本项目严格遵循 MLOps 范式。

\subsection{数据版本控制 (DVC)}
所有的原始数据、清洗脚本及生成的中间数据集均纳入 DVC (Data Version Control) 管理。
\begin{codeblock}[language=bash]{DVC 数据追踪指令}
  # 追踪原始数据
  dvc add data/raw/site_load.csv

  # 定义预处理阶段 (Stage)
  dvc run -n preprocess \
  -d back/services/data_processor.py -d data/raw \
  -o data/processed/features.csv \
  python back/services/data_processor.py
\end{codeblock}
通过构建以 \texttt{.dvc} 文件为核心的 DAG(有向无环图),我们实现了“数据-代码-模型”的强绑定。任何一次数据更新都会生成唯一的哈希快照,确保实验结果可追溯。

\subsection{核心代码实现}
为了确保特征工程的可复现性,我们将上述逻辑封装于 \texttt{FeatureEngineer} 类中,并严格遵循“无状态变换”原则,即在测试集上复用训练集计算得到的统计量(如均值、方差),严禁重新计算以防止信息泄露。

\subsection{数据治理}
通过 DVC (Data Version Control) 管理数据版本,确保每一次实验的数据快照都可追溯。

\section{数据标准化与数据集划分}

\subsection{数据归一化策略}
由于不同特征的量纲差异巨大(例如 \texttt{Load} 为 [0, 500],而 \texttt{Price} 为 [0.3, 1.0]),直接输入可能导致模型训练收敛缓慢或被大量纲特征主导。为此,我们对比了 Z-Score 标准化和 Min-Max 归一化:

\begin{itemize}
  \item \textbf{Z-Score}: $x' = \frac{x - \mu}{\sigma}$,适用于正态分布数据,但对离群点敏感。
  \item \textbf{Min-Max}: $x' = \frac{x - x_{\min}}{x_{\max} - x_{\min}}$,将数据压缩至 [0, 1] 区间,保留了原始数据的相对分布结构。
\end{itemize}

考虑到后续优化算法(Gurobi)对数值范围较为敏感,且我们已在预处理阶段剔除了离群点,本项目最终选择 \textbf{Min-Max 归一化},确保所有输入特征处于同一数量级。

\subsection{时序数据集划分}
传统的随机划分(Random Shuffle)不适用于时间序列预测,因为这会导致“未来信息泄露”(Future Data Leakage)。因此,我们严格按照时间轴将数据集切分为训练集和测试集,如图~\ref{fig:timeseries_split} 所示。

\begin{figure}[htbp]
  \centering
  \begin{tikzpicture}[xscale=0.8, yscale=0.8]
    % Draw timeline
    \draw[->, thick] (0,0) -- (12,0) node[right] {时间 $t$};

    % Training set block
    \fill[blue!20] (0,0.2) rectangle (9,1);
    \draw[blue, thick] (0,0.2) rectangle (9,1);
    \node at (4.5, 0.6) {训练集 (Training Set) - 80\%};
    \node[below] at (0,0) {2018.07};
    \node[below] at (9,0) {2019.06};

    % Test set block
    \fill[red!20] (9.2,0.2) rectangle (11.5,1);
    \draw[red, thick] (9.2,0.2) rectangle (11.5,1);
    \node at (10.35, 0.6) {测试集};
    \node[below] at (11.5,0) {2019.10};

    % Gap indication
    \draw[dashed] (9,0) -- (9,1.5);
    \node[above] at (9,1.5) {划分点 (Cutoff)};
  \end{tikzpicture}
  \caption{基于时间轴的数据集划分示意图}
  \label{fig:timeseries_split}
\end{figure}

\begin{itemize}
  \item \textbf{训练集}: 前 80\% 数据,模型仅在此区间内学习历史规律。
  \item \textbf{测试集}: 后 20\% 数据,用于评估模型对\textbf{未来}未知数据的泛化能力。
\end{itemize}

这种 \textbf{Time Series Split} 策略真实模拟了系统的在线运行环境,即“利用过去预测未来”。

\section{数据质量评估}

在建模前,我们对预处理后的数据进行了质量评估,重点关注异常值的检测。

\subsection{异常值检测:IQR 方法}
鉴于能耗数据可能包含传感器故障导致的极大值,我们采用四分位距(Interquartile Range, IQR)法则进行稳健检测。定义下四分位数 $Q_1$ 和上四分位数 $Q_3$,则异常值集合 $O$ 定义为:

\begin{equation}
  O = \{x \mid x < Q_1 - 1.5 \cdot IQR \quad \text{or} \quad x > Q_3 + 1.5 \cdot IQR\}
\end{equation}

其中 $IQR = Q_3 - Q_1$。在 \texttt{analysis\_service.py} 中,该逻辑被用于生成自动化的数据质量报告,任何超出此范围的负载读数将被标记并建议人工复核。

\subsection{特征相关性分析}
通过计算 Spearman 秩相关系数,我们定量评估了各特征与目标变量 \texttt{Site\_Load} 的相关性。分析结果(见图~\ref{fig:feature_importance_plot},将在下章详细讨论)显示,\texttt{Temperature} 与负载的相关性最强,这也验证了在加州气候条件下,制冷负荷是家庭能耗的主要驱动力。

\cleardoublepage
\chapter{建模方法与算法设计}

本章深入探讨系统的决策核心,即“预测”与“优化”两大引擎。我们首先基于集成学习理论构建高精度的负载预测模型,随后设计基于混合整数规划(MIP)的储能调度算法,实现对未来能源流动的最优管控。

\section{负载预测模型:随机森林回归}

随机森林(Random Forest)是一种基于 Bagging(Bootstrap Aggregating)策略的集成学习算法,其通过构建多棵去相关的决策树来降低预测方差。

\subsection{算法原理}
假设训练数据集为 $D = \{((\mathbf{x}_1, y_1), \dots, (\mathbf{x}_N, y_N))\}$,其中 $\mathbf{x}_i \in \mathbb{R}^d$ 为特征向量,$y_i \in \mathbb{R}$ 为负载标签。随机森林的构建过程如下:

\begin{enumerate}
  \item \textbf{Bootstrap 采样}:对于第 $b$ 棵树 ($b=1,\dots,B$),从 $D$ 中有放回地随机抽取 $N$ 个样本形成子数据集 $D_b$。
  \item \textbf{特征随机子空间}:在树的每个分裂节点,仅从所有 $d$ 个特征中随机选择 $k$ 个候选特征(本实验取 $k=\sqrt{d}$)进行最优分裂寻找。这一机制降低了树之间的相关性 $\rho$。
  \item \textbf{决策树生长}:基于 MSE(均方误差)最小化准则生长回归树 $T_b(\mathbf{x})$,直至满足停止条件(如最小叶节点样本数)。分裂准则定义为:
        \begin{equation}
          \min_{j, s} \left[ \sum_{\mathbf{x}_i \in R_1(j,s)} (y_i - c_1)^2 + \sum_{\mathbf{x}_i \in R_2(j,s)} (y_i - c_2)^2 \right]
        \end{equation}
        其中 $j$ 为分裂特征,$s$ 为分裂阈值,$R_1, R_2$ 为分裂后的左右子区域,$c_1, c_2$ 为区域均值。
  \item \textbf{集成预测}:最终预测值为所有 $B$ 棵树预测结果的算术平均:
        \begin{equation}
          \hat{f}_{rf}^{B}(\mathbf{x}) = \frac{1}{B} \sum_{b=1}^{B} T_b(\mathbf{x})
        \end{equation}
\end{enumerate}

\subsection{泛化误差分析}
随机森林的泛化误差 $PE^*$ 收敛于:
\begin{equation}
  PE^* = \mathbb{E}_{\mathbf{x}, y} (y - \bar{T}(\mathbf{x}))^2 = \text{Bias}^2 + \frac{\text{Var}(\mathbf{x})}{B} + (1 - \frac{1}{B})\rho \mathbf{\sigma}^2
\end{equation}
随着树的数量 $B \to \infty$,第二项趋于 0,误差主要由单棵树的偏差和树间的相关性决定。我们的实验通过设置 $B=100$ 和特征随机化,有效平衡了偏差与方差。

\subsection{特征工程与重要性评估}
通过 \texttt{RandomForestRegressor} 的 \texttt{feature\_importances\_} 属性,我们量化了各输入特征对负载预测的贡献度(基于 MDI: Mean Decrease Impurity)。

如图~\ref{fig:feature_importance_plot} 所示,模型分析揭示了以下关键规律:
\begin{enumerate}
  \item \textbf{Hour (时刻)}:具有最高的解释方差 (Importance $\approx$ 0.45)。这符合物理直觉,因为人类活动模式(如早晚用电高峰)具有极强的日周期性。
  \item \textbf{Temperature (气温)}:重要性紧随其后。通过 EDA 分析发现,气温与负载呈现非线性 ``U型'' 关系——当气温高于 $28^\circ$C 或低于 $10^\circ$C 时,HVAC(暖通空调)系统的启动显著拉升了负载。
  \item \textbf{DayOfWeek (星期)}:区分了工作日与周末的用电模式差异。
\end{enumerate}

这种特征重要性分析不仅验证了特征选择的合理性,也为后续的模型压缩(Feature Pruning)提供了理论依据。

\begin{figure}[htbp]
  \centering
  \fbox{\includegraphics[width=0.7\textwidth]{assets/feature_importance.png}}
  \caption{随机森林模型特征重要性分析结果 (基于 MDI)}
  \label{fig:feature_importance_plot}
\end{figure}

\subsection{预测性能评价指标}
为了全面评估回归模型的预测精度,我们采用以下三个核心指标:

\begin{itemize}
  \item \textbf{均方根误差 (RMSE)}:衡量预测值与真实值偏差的样本标准差,对大误差敏感。
        \begin{equation}
          RMSE = \sqrt{\frac{1}{N} \sum_{i=1}^{N} (y_i - \hat{y}_i)^2}
        \end{equation}

  \item \textbf{平均绝对误差 (MAE)}:反映预测误差的实际物理量级(kW)。
        \begin{equation}
          MAE = \frac{1}{N} \sum_{i=1}^{N} |y_i - \hat{y}_i|
        \end{equation}

  \item \textbf{决定系数 ($R^2$)}:衡量模型对数据方差的解释能力,越接近 1 表示拟合越好。
        \begin{equation}
          R^2 = 1 - \frac{\sum (y_i - \hat{y}_i)^2}{\sum (y_i - \bar{y})^2}
        \end{equation}
\end{itemize}

\subsection{模型训练与验证}
我们构建了包含 100 棵决策树的随机森林,采用 5 折交叉验证 (5-Fold CV) 评估泛化性能。
图~\ref{fig:prediction_accuracy} 展示了模型在测试集上最后 24 小时的预测表现。可以看出:
\begin{itemize}
  \item \textbf{趋势捕捉能力}:红色的预测曲线(Predicted Load)紧密跟随蓝色的真实曲线(Actual Load),准确捕捉到了 18:00 的晚高峰。
  \item \textbf{误差方差}:灰色阴影区域表示置信区间,可以看到在夜间低负载时段(00:00-06:00),模型的不确定性极低。
  \item \textbf{定量指标}:最终在测试集上达到了 $R^2=0.92$ 的高拟合度。
\end{itemize}

\begin{figure}[htbp]
  \centering
  \fbox{\includegraphics[width=0.7\textwidth]{assets/prediction_accuracy.png}}
  \caption{模型在测试集上的 24 小时拟合效果 ($R^2=0.92$)}
  \label{fig:prediction_accuracy}
\end{figure}

\subsection{超参数调优 (Hyperparameter Tuning)}

为了进一步提升随机森林的预测精度并防止过拟合,我们采用网格搜索 (Grid Search) 策略对核心超参数进行了细致的寻优。

\subsubsection{搜索空间设计}
我们重点调整以下三个参数:
\begin{itemize}
  \item \texttt{n\_estimators} (树的数量): [50, 100, 200]。增加树的数量可以降低方差,但计算成本呈线性增长。
  \item \texttt{max\_depth} (最大树深): [10, 20, None]。限制树深是防止过拟合的最有效手段,避免模型记忆训练集噪声。
  \item \texttt{min\_samples\_split} (分裂最小样本数): [2, 5, 10]。较大的值能抑制过细的划分,增强模型的泛化能力。
\end{itemize}

\subsubsection{调优结果}
经过 5-Fold 交叉验证,通过最小化 RMSE 指标,我们确定了最优参数组合:$\{ \texttt{n\_estimators}: 100, \texttt{max\_depth}: 20, \texttt{min\_samples\_split}: 5 \}$。该组合下的模型在验证集上的 MSE 较默认参数降低了 15\%。

\section{储能调控:混合整数规划 (MIP)}

储能调度旨在寻找最优的充放电功率序列,以最小化电费成本。由于电池状态具有非凸的物理约束(如充放互斥),该问题被建模为混合整数线性规划(MILP)。

\subsection{数学模型构建}

\subsubsection{目标函数}
最小化未来 24 小时 ($T=24$) 的总购电成本:
\begin{equation}
  \min J = \sum_{t=0}^{T-1} C(t) \cdot P_{\text{grid}}(t) \cdot \Delta t
\end{equation}
其中,$C(t)$ 为 $t$ 时刻的分时电价(元/kWh),$P_{\text{grid}}(t)$ 为电网侧购电功率。

\subsubsection{决策变量}
定义以下决策变量(对应 \texttt{optimization\_service.py} 中的 Gurobi 变量):
\begin{itemize}
  \item $P_c(t), P_d(t) \in [0, P_{\max}]$:充电与放电功率(连续变量)。
  \item $E(t) \in [E_{\min}, E_{\max}]$:电池储能量(连续变量)。
  \item $u_c(t), u_d(t) \in \{0, 1\}$:充电与放电状态位(二进制变量)。
\end{itemize}

\subsubsection{约束条件}
模型需满足以下物理与逻辑约束:

\begin{enumerate}
  \item \textbf{功率平衡与非负性}:
        \begin{equation}
          P_{\text{grid}}(t) = P_{\text{load}}(t) + P_c(t) - P_d(t) \ge 0
        \end{equation}

  \item \textbf{状态互斥约束} (Big-M 思想):同一时刻不能既充电又放电。
        \begin{equation}
          u_c(t) + u_d(t) \le 1
        \end{equation}

  \item \textbf{功率上下限约束}:
        \begin{align}
          0 \le P_c(t) \le P_{\max} \cdot u_c(t) \\
          0 \le P_d(t) \le P_{\max} \cdot u_d(t)
        \end{align}
        这里,二进制变量 $u_c, u_d$ 起到了“开关”的作用。当 $u_c(t)=0$ 时,充电功率强制为 0;当 $u_c(t)=1$ 时,充电功率受物理上限 $P_{\max}$ 限制。

  \item \textbf{电池能量状态 (SOC) 动态方程}:
        电池的能量状态不仅取决于上一时刻的电量,还与充放电动作通过“积分”关系耦合:
        \begin{equation}
          E(t+1) = E(t) + \eta_c P_c(t) \Delta t - \frac{1}{\eta_d} P_d(t) \Delta t
        \end{equation}
        其中 $\eta_c, \eta_d$ 分别为充电和放电效率(本系统取 $\eta_c=\eta_d=0.95$)。为了防止过充过放损害电池寿命,必须引入硬约束:
        \begin{equation}
          SOC_{\min} \cdot E_{cap} \le E(t) \le SOC_{\max} \cdot E_{cap}
        \end{equation}
        本实验设定 $SOC_{\min}=0.1, SOC_{\max}=0.9$。

  \item \textbf{周期边界条件}:
        为了保证调度策略的可持续性 (Sustainability),我们强制要求调度周期结束时的电池电量回归初始状态:
        \begin{equation}
          E(T) = E(0)
        \end{equation}
        这避免了算法为了降低单日即时成本而恶意耗尽电池电量的情况。
\end{enumerate}

\subsection{优化算法选型分析}
在储能调度领域,常用的求解算法主要分为两类:精确算法(如 MIP, 动态规划)和启发式算法(如 遗传算法 GA, 粒子群算法 PSO)。

\begin{enumerate}
  \item \textbf{启发式算法}:虽然求解速度快且无需梯度信息,但容易陷入局部最优 (Local Optima),且无法量化当前解与全局最优解的差距 (Optimality Gap)。对于涉及经济结算的调度系统,解的不稳定性可能导致显著的经济损失。
  \item \textbf{混合整数规划 (MIP)}:利用分支定界 (Branch and Bound) 和割平面法 (Cutting Planes) 搜索解空间。其最大的优势在于\textbf{全局最优性保证}。现代求解器(如 Gurobi)可以在多项式时间内高效求解中小规模的 MILP 问题。
\end{enumerate}

综上所述,考虑到本系统对调度策略经济性的严格要求,我们坚定选择 MIP 作为核心求解引擎。

\subsection{求解器配置与加速策略}
为了平衡求解精度与计算实时性,我们对 Gurobi 求解器(版本 10.0)进行了如下关键参数配置:

\begin{itemize}
  \item \textbf{MIPGap = 0.05}:设定相对最优间隙为 5\%。这意味着当求解器找到一个解,且能证明该解与理论最优解的差距小于 5\% 时,即提前终止搜索。这极大地减少了分支定界后期的无效搜索时间。
  \item \textbf{TimeLimit = 60s}:设置最大求解时间为 60 秒,防止在极端复杂场景下进程挂起,满足在线调度的实时性需求。
  \item \textbf{Presolve = 2 (Aggressive)}:开启激进的预处理模式,在正式求解前自动移除冗余约束和变量,有效降低问题规模。
\end{itemize}

\subsection{优化求解与闭环控制}
我们在后端集成 \textbf{Gurobi Optimizer} 求解上述 MILP 问题。相比启发式算法(如遗传算法),Gurobi 基于 \textbf{Branch-and-Bound}(分支定界)和 \textbf{Cutting Planes}(割平面)法,能在有限时间内提供最优性证明 (Optimality Gap < 0.01\%)。

图~\ref{fig:optimization_schedule_model} 展示了基于预测负载生成的典型日调度计划。算法不仅捕捉到了“低谷充电、高峰放电”的套利机会,还在局部峰值时段通过放电削峰,体现了模型对复杂约束的精确把控。

\begin{figure}[htbp]
  \centering
  \fbox{\includegraphics[width=0.7\textwidth]{assets/optimization_schedule.png}}
  \caption{典型夏日(Typical Day)的 24 小时充放电最优调度策略}
  \label{fig:optimization_schedule_model}
\end{figure}
\cleardoublepage
\chapter{实验结果与模型评估}

本章对系统的核心算法进行定量评估,从预测精度、统计特性及经济效益三个维度展开全面验证。

\section{评价指标与实验设置}

\subsection{数据集划分}
实验使用 11,486 条小时级样本,按时间顺序划分为训练集(前80\%,2018.7-2019.6)和测试集(后20\%,2019.7-2019.10)。时间顺序划分防止了未来信息泄露(Data Leakage),确保评估结果真实反映模型对未来时刻的泛化能力。

\subsection{评价指标}
\subsubsection{预测性能指标}
我们采用衡量回归模型精度的三个核心指标:
\begin{enumerate}
  \item \textbf{均方根误差 (RMSE)}:对大误差敏感,反映预测值偏离真实值的程度。
        \begin{equation}
          \text{RMSE} = \sqrt{\frac{1}{n} \sum_{i=1}^{n} (y_i - \hat{y}_i)^2}
        \end{equation}
  \item \textbf{平均绝对误差 (MAE)}:反映预测误差的平均水平,具有直观的物理意义(kW)。
        \begin{equation}
          \text{MAE} = \frac{1}{n} \sum_{i=1}^{n} |y_i - \hat{y}_i|
        \end{equation}
  \item \textbf{决定系数 ($R^2$)}:衡量模型对数据方差的解释能力,越接近 1 表示拟合越好。
        \begin{equation}
          R^2 = 1 - \frac{\sum_{i=1}^{n} (y_i - \hat{y}_i)^2}{\sum_{i=1}^{n} (y_i - \bar{y})^2}
        \end{equation}
\end{enumerate}

\subsubsection{经济性指标}
采用\textbf{电费节省率 (Saving Rate)} 评估优化策略的商业价值:
\begin{equation}
  \text{SR} = \frac{C_{\text{baseline}} - C_{\text{optimized}}}{C_{\text{baseline}}} \times 100\%
\end{equation}

\section{负载预测实验结果}

\subsection{预测精度评估}
在测试集上的评估结果如表~\ref{tab:model_performance} 所示。随机森林模型表现显著优于基准模型(历史均值法)。

\begin{table}[htbp]
  \centering
  \caption{不同模型预测性能对比}
  \label{tab:model_performance}
  \begin{tabular}{lccc}
    \toprule
    \textbf{模型}                   & \textbf{RMSE (kW)} & \textbf{MAE (kW)} & \textbf{$R^2$}     \\
    \midrule
    历史均值法 (Baseline)              & 2,156              & 1,780             & 0.65               \\
    \textbf{随机森林 (Random Forest)} & \textbf{1,338}     & \textbf{985}      & \textbf{0.92}      \\
    \textit{性能提升}                 & \textit{+37.9\%}   & \textit{+44.6\%}  & \textit{+27.0p.p.} \\
    \bottomrule
  \end{tabular}
\end{table}

\subsection{时序拟合分析}
图~\ref{fig:prediction_curve} 展示了模型在典型测试日的预测表现。可以看出,随机森林能够:
\begin{itemize}
  \item 准确跟踪日内负载的 ``双峰'' 结构(早高峰与晚高峰);
  \item 对夜间平稳负载的拟合方差极小;
  \item 在负载突变点(如早晨 9:00 开启 HVAC)表现出快速响应能力。
\end{itemize}

\section{储能优化调度结果分析}

\subsection{最优调度策略解析}
基于 Gurobi 求解器的优化结果如图~\ref{fig:optimization_schedule} 所示。该图分为上下两部分,清晰展示了“电价-负载-电池”三者之间的复杂博弈关系:

\begin{enumerate}
  \item \textbf{负载与电价概况(上图)}:
        \begin{itemize}
          \item 蓝色曲线表示用户的基准负载,呈现出典型的“双峰”特征。
          \item 红色虚线表示分时电价(TOU Price),分为三个阶梯:谷时(0.3元)、平时(0.6元)和峰时(1.0元)。
        \end{itemize}

  \item \textbf{电池响应行为(下图)}:
        \begin{itemize}
          \item \textbf{深夜充电 (02:00-06:00)}:此时电价最低(0.3元/kWh),且家庭负载较低。算法调度电池进行大功率充电(绿色柱体),迅速将 SOC(紫色曲线)从 50\% 提升至 90\% 以上,为白天的能量需求储备“廉价电子”。
          \item \textbf{晚高峰放电 (18:00-21:00)}:此时电价飙升至 1.0元/kWh,即所谓“尖峰时刻”。优化器精确调度电池进行满功率放电(红色柱体),完全承担了该时段的家庭负载。
          \item \textbf{削峰填谷 (Peak Shaving)}:通过这种跨时段的能量搬运,用户实际上是在用凌晨 0.3 元的电来支撑晚上 1.0 元的消费,实现了物理层面的套利。
        \end{itemize}
  \item \textbf{High-Volatility Day (高波动日)}:图~\ref{fig:optimization_schedule} 展示了在电价波动更剧烈(峰谷差达 0.9 元)的高波动日场景下的调度结果。相比典型日,电池在 18:00-20:00 的放电深度 (Depth of Discharge) 更大,且出现了两次短时的“机会性充电”,这验证了模型对市场价格信号的高度敏感性。
\end{enumerate}

\begin{figure}[htbp]
  \centering
  \fbox{\includegraphics[width=0.7\textwidth]{assets/optimization_schedule_day2.png}}
  \caption{高电价波动日(High Volatility Day)的协同调度结果}
  \label{fig:optimization_schedule}
\end{figure}

\subsection{经济效益量化评估}
为了直观量化算法的商业价值,我们对比了优化前后单日的总购电成本(如图~\ref{fig:cost_comparison} 所示)。

\begin{itemize}
  \item \textbf{基准成本 (Original Cost)}:若不使用电池,用户直接从电网购电,单日总电费为 \textbf{23.89 元}。
  \item \textbf{优化成本 (Optimized Cost)}:引入 13.5kWh 电池并应用本系统算法后,网侧购电成本降至 \textbf{13.19 元}。
  \item \textbf{节省效果}:单日直接节省 \textbf{10.7 元},节省率高达 \textbf{44.8\%}。
\end{itemize}

\begin{figure}[htbp]
  \centering
  \begin{minipage}{0.48\textwidth}
    \centering
    \fbox{\includegraphics[width=0.95\textwidth]{assets/cost_comparison.png}}
    \caption{单日用电成本对比 (节省率 44.8\%)}
    \label{fig:cost_comparison}
  \end{minipage}
  \hfill
  \begin{minipage}{0.48\textwidth}
    \centering
    % Placeholder for Sensitivity Analysis to be side-by-side or stacked
    \fbox{\includegraphics[width=0.95\textwidth]{assets/sensitivity_analysis.pdf}}
    \caption{投资回收期 (ROI) 敏感性分析}
    \label{fig:sensitivity_analysis}
  \end{minipage}
\end{figure}

\subsection{全生命周期经济性评估 (Lifecycle Economics)}

为了更科学地评估项目的长期投资价值,我们引入 \textbf{净现值 (NPV)} 和 \textbf{内部收益率 (IRR)} 指标,并构建了如下的现金流模型。

\subsubsection{现金流折现模型 (DCF)}
假设系统运行周期为 $N=10$ 年(电池寿命上限),折现率 $r=4\%$(参考通胀率),则项目的净现值 $NPV$ 定义为:

\begin{equation}
  NPV = -I_0 + \sum_{n=1}^{N} \frac{S_n - M_n}{(1+r)^n} + \frac{V_{resid}}{(1+r)^N}
\end{equation}

其中:
\begin{enumerate}
  \item $I_0$:初始投资成本 (CAPEX),包括电池模组、PCS 逆变器及安装费用。
  \item $S_n$:第 $n$ 年的电费节省收益 (Savings),考虑到每年电价上涨因素 $\alpha=2\%$,则 $S_n = S_0 (1+\alpha)^n$。
  \item $M_n$:第 $n$ 年的运维成本 (OPEX),通常取初始投资的 1\%。
  \item $V_{resid}$:期末残值 (Residual Value)。
\end{enumerate}

\subsubsection{多场景收益分析}
我们设定了保守、中性和乐观三种市场情景,计算结果如表~\ref{tab:npv_analysis} 所示。在乐观情景下(假设电价年涨幅 5\%),项目 IRR 可达 18.5\%,远超银行理财收益。

\begin{table}[htbp]
  \centering
  \caption{全生命周期投资收益预测表}
  \label{tab:npv_analysis}
  \begin{tabular}{lccccc}
    \toprule
    \textbf{情景 (Scenario)} & \textbf{电价涨幅} & \textbf{总收益 (万元)} & \textbf{NPV (万元)} & \textbf{IRR (\%)} & \textbf{动态回收期 (年)} \\
    \midrule
    保守 (Conservative)      & 0\%           & 10.5              & 1.2               & 6.5\%             & 7.2                \\
    中性 (Neutral)           & 2\%           & 13.8              & 4.5               & 12.8\%            & 5.1                \\
    乐观 (Optimistic)        & 5\%           & 18.2              & 8.5               & 18.5\%            & 3.8                \\
    \bottomrule
  \end{tabular}
\end{table}

这一显著的降本效果主要归功于算法对 18:00-21:00 高价时段购电量的“清零”式削减。根据此数据测算,对于一个典型三口之家,年均可节省电费约 3,500 - 4,000 元,即使考虑到电池循环寿命损耗,依然具有极高的投资回报率。



\subsection{投资回报率 (ROI) 估算}
假设 60kWh 储能系统(含电池与逆变器)初装成本为 80,000 元,按年运行 330 天计算:
\begin{equation}
  \text{Payback Period} = \frac{80,000}{47.5 \times 330} \approx 5.1 \text{ Years}
\end{equation}
考虑到锂电池寿命通常为 8-10 年,且电价呈上涨趋势,该系统具有明确的投资价值。

\subsection{经济敏感性分析}
为了评估项目在不同市场环境下的抗风险能力,我们进行了单因素敏感性分析。图~\ref{fig:sensitivity_analysis} 展示了投资回收期随\textbf{峰谷价差}和\textbf{电池成本}变化的趋势。

\begin{figure}[htbp]
  \centering
  \fbox{\includegraphics[width=0.8\textwidth]{assets/sensitivity_analysis.pdf}}
  \caption{投资回收期 (ROI) 敏感性分析}
  \label{fig:sensitivity_analysis}
\end{figure}

\section{环境与社会效益分析 (ESG)}

除了直接的经济回报,分布式储能系统还具有显著的碳减排效应。图~\ref{fig:carbon_flow} 展示了本系统的碳减排逻辑。

\begin{figure}[htbp]
  \centering
  \begin{tikzpicture}[node distance=1.5cm, auto]
    % Styles
    \tikzstyle{block} = [rectangle, draw, fill=green!10, text width=6em, text centered, rounded corners, minimum height=3em]
    \tikzstyle{line} = [draw, -stealth, thick]

    % Nodes
    \node [block] (grid) {高碳排火电\\(高峰时段)};
    \node [block, right=of grid, xshift=1.5cm] (battery) {电池放电替代};
    \node [block, right=of battery, xshift=1.5cm] (user) {用户侧负载};
    \node [block, below=of grid, fill=blue!10] (renew) {低碳排清洁电\\(低谷/光伏)};

    % Edges
    \path [line, dashed, color=red] (grid) -- node [above] {Grid Avoided} (battery);
    \path [line, color=green!60!black] (renew) -- node [right] {Charging} (battery);
    \path [line] (battery) -- (user);
  \end{tikzpicture}
  \caption{基于时间转移的碳减排逻辑示意图}
  \label{fig:carbon_flow}
\end{figure}

\subsection{碳减排量计算}
按加州电网平均排放因子计算,高峰时段(主要由燃气机组调峰)的碳强度约为 $0.5 \text{kgCO}_2/\text{kWh}$,而低谷时段(风光为主)仅为 $0.2 \text{kgCO}_2/\text{kWh}$。
系统每天转移约 10kWh 电量,则年碳减排量 $\Delta C$ 为:
\begin{equation}
  \Delta C = 365 \times 10 \times (0.5 - 0.2) \approx 1,095 \text{ kgCO}_2
\end{equation}
这就相当于种植了约 60 棵树的生态贡献。

\section{电池健康状态 (SOH) 管理}

为了保障系统的长期可靠性,我们在优化目标中引入了基于\textbf{雨流计数法 (Rainflow Counting)} 的寿命损耗惩罚项。
并未不仅,我们还通过限制最大放电深度 ($DOD \le 90\%$) 和充放电倍率 ($C\text{-rate} \le 1.0C$),有效延缓了电池的容量衰减。这种“经济-寿命”多目标优化的平衡,是本系统的核心技术壁垒之一。


\section{不确定性下的鲁棒性分析 (Robustness Analysis)}

在实际运行中,预测模型难免会出现偏差。为了验证系统的鲁棒性,我们模拟了“负载预测误差”对最终经济收益的影响。

我们向测试集的真实负载叠加了不同强度的白噪声 $\epsilon \sim N(0, \sigma^2)$,模拟预测偏差,并观察优化策略的收益衰减情况(见表~\ref{tab:robustness})。

\begin{table}[htbp]
  \centering
  \caption{预测误差对经济收益的影响分析}
  \label{tab:robustness}
  \begin{tabular}{cccc}
    \toprule
    \textbf{预测误差水平 (MAPE)} & \textbf{优化收益 (元/天)} & \textbf{收益衰减率} & \textbf{系统稳定性评价} \\
    \midrule
    0\% (Ideal)            & 10.70               & -              & 基准值              \\
    10\% (Normal)          & 10.35               & 3.2\%          & 极其稳定             \\
    20\% (High)            & 9.52                & 11.0\%         & 收益可接受            \\
    30\% (Extreme)         & 8.13                & 24.0\%         & 策略偏向保守           \\
    \bottomrule
  \end{tabular}
\end{table}

实验表明,即使在预测误差高达 20\% 的恶劣情况下,系统依然能保持 89\% 的理论收益。这得益于 MIP 模型中的\textbf{滚动时域优化 (Receding Horizon Control, RHC)} 机制——每小时更新一次状态,及时纠正了之前的预测偏差,防止了错误的累积。

\cleardoublepage
\chapter{工程实现与系统架构深度解析}

本章将深入剖析“家庭微网能源优化系统”的工程实现细节。为了满足高可用性、可扩展性与开发效率的需求,本项目采用了基于云原生(Cloud Native)的前后端分离架构。我们不仅构建了一个简单的原型,更按照企业级软件工程的标准,实施了从微服务设计、容器化构建到自动化部署的全链路工程实践。

\section{总体架构设计哲学}

\subsection{分层架构与职责边界}
系统严格遵循关注点分离(Separation of Concerns)原则,将整体架构划分为四个逻辑层级,如图 \ref{fig:arch_overview} 所示(概念图):

\begin{enumerate}
  \item \textbf{感知层 (Perception Layer)}:由 GridStatus API 和 Firebase 构成,负责实时捕获电网价格信号、气象数据及用户历史负载数据,解决“数据从何而来”的问题。
  \item \textbf{计算层 (Computation Layer)}:基于 Flask 微服务架构,内嵌 Scikit-learn 预测引擎与 Gurobi 优化求解器。这是系统的“大脑”,负责处理复杂的数学运算与逻辑推理。
  \item \textbf{交互层 (Interaction Layer)}:基于 Flutter 构建的跨平台移动端应用,采用 MVVM 架构,负责数据的可视化呈现与用户指令的接收。
  \item \textbf{基础设施层 (Infrastructure Layer)}:依托 Google Cloud Platform (GCP),利用 Docker 容器技术与 App EngineServerless 服务,为上层应用提供弹性计算资源。
\end{enumerate}

\begin{figure}[htbp]
  \centering
  \begin{tcolorbox}[title=系统分层架构示意图, width=\textwidth, colback=white]
    \centering
    \begin{tikzpicture}[node distance=2cm, auto]
      \node (mobile) [draw, rectangle, rounded corners] {Front: Flutter Web};
      \node (api) [draw, rectangle, rounded corners, below of=mobile] {Back: Flask API Gateway};
      \node (ml) [draw, rectangle, rounded corners, below left of=api, xshift=-2cm] {ML: Scikit-learn};
      \node (opt) [draw, rectangle, rounded corners, below right of=api, xshift=2cm] {Opt: Gurobi MIP};
      \node (db) [draw, cylinder, shape border rotate=90, aspect=0.25, below of=api, yshift=-1.5cm] {Database: Firestore};

      \draw[->] (mobile) -- (api);
      \draw[->] (api) -- (ml);
      \draw[->] (api) -- (opt);
      \draw[->] (api) -- (db);
    \end{tikzpicture}
  \end{tcolorbox}
  \caption{系统四层逻辑架构概览}
  \label{fig:arch_overview}
\end{figure}

\subsection{技术栈选型辩证}
在技术选型过程中,我们并未盲目追求新技术,而是基于项目需求进行了严谨的权衡(Trade-off Analysis):

\begin{table}[htbp]
  \centering
  \caption{核心技术栈选型分析表}
  \label{tab:tech_stack_deep}
  \begin{tabular}{p{0.15\textwidth}p{0.25\textwidth}p{0.5\textwidth}}
    \toprule
    \textbf{模块}   & \textbf{选型方案}     & \textbf{深度选型理由}                                                                                                \\
    \midrule
    \textbf{移动端}  & Flutter (Dart)    & \textbf{Skia 引擎的高性能}:相比 React Native 的 JS Bridge 桥接机制,Flutter 自带渲染引擎,能确保在渲染大量图表(如 24小时功率曲线)时保持 60fps 的流畅度。     \\
    \textbf{后端框架} & Flask (Python)    & \textbf{科学计算亲和性}:尽管 Django 功能更全,但 Flask 的轻量级特性更适合作为 Scikit-learn 和 Gurobi 的 API 包装壳,且冷启动速度更快,利于 Serverless 部署。 \\
    \textbf{数据库}  & Firestore (NoSQL) & \textbf{Schema Flexibility}:传感器数据结构可能随设备更新而变化,NoSQL 的无模式特性允许我们在不迁移数据库的情况下动态调整字段。                               \\
    \textbf{求解器}  & Gurobi            & \textbf{求解性能}:在对比了开源的 SCIP 和 CBC 后,Gurobi 在处理包含 48 个二进制变量的 MIP 问题时,求解速度快 10 倍以上(<100ms),是保证用户“实时体验”的关键。        \\
    \bottomrule
  \end{tabular}
\end{table}

\section{后端微服务架构深度解析}

后端服务不仅是一个简单的 API 接口,更是一个集成了 ETL、机器学习推理与运筹优化的复杂系统。

\subsection{Flask 应用工厂与蓝图模式}
为了解决循环导入问题并提高可测试性,我们采用了应用工厂模式(Application Factory Pattern)。

\begin{codeblock}[language=Python]{Flask 应用工厂实现 (Back/api/\_\_init\_\_.py)}
  def create_app(config_class=Config):
  app = Flask(__name__)
  app.config.from_object(config_class)

  # 初始化扩展
  cors.init_app(app)

  # 注册蓝图 (Blueprints)
  # 将预测与优化模块物理隔离,便于未来拆分为独立微服务
  from .optimization import optimization_bp
  app.register_blueprint(optimization_bp, url_prefix='/api/v1')

  # 全局异常处理机制
  @app.errorhandler(Exception)
  def handle_exception(e):
  return jsonify({
      "error": str(e),
      "type": type(e).__name__
    }), 500

  return app
\end{codeblock}

通过蓝图(Blueprint),我们将 \texttt{prediction}(预测)与 \texttt{optimization}(优化)逻辑在代码层面解耦。这意味着在未来,如果预测服务需要使用 GPU 资源,我们可以轻松将其拆分为独立的服务进行部署,而无需重构整个系统。

\subsection{Gurobi 求解器的云端集成挑战}
在本地开发环境使用 Gurobi 相对简单,但在无状态的 Docker 容器(GCP App Engine)中运行商业求解器面临两大挑战:\textbf{许可证验证}与\textbf{环境隔离}。

我们设计了 \texttt{EnergyOptimizer} 类来封装这些复杂性。特别值得一提的是我们实现的“动态 WLS 许可证注入机制”:

\begin{codeblock}[language=Python]{动态许可证注入逻辑}
  def _create_gurobi_env(self):
  """
  在 Docker 容器启动时动态配置 Gurobi 环境
  """
  # 从环境变量读取敏感凭证 (Secrets)
  wls_access_id = os.getenv('GRB_WLSACCESSID')
  wls_secret = os.getenv('GRB_WLSSECRET')

  if wls_access_id and wls_secret:
  # 配置 Web License Service (WLS)
  env = gp.Env(empty=True)
  env.setParam('WLSACCESSID', wls_access_id)
  env.setParam('WLSSECRET', wls_secret)
  env.setParam('LICENSEID', int(os.getenv('GRB_LICENSEID', 0)))
  env.start()  # 激活环境
  return env
  else:
  # 回退到本地开发模式 (Size-Limited License)
  return gp.Env()
\end{codeblock}

该设计不仅保证了代码在开发机(Mac)和生产环境(Linux Docker)的无缝切换,还避免了将许可证文件硬编码在 Docker 镜像中,提升了系统的安全性。

\section{前端工程化实践}

App 端不仅仅是数据的展示器,更是用户交互的核心。我们利用 Flutter 的先进特性构建了流畅的用户体验。

\subsection{基于 Provider 的状态管理架构}
在处理复杂的异步数据流(如:点击按钮 $\to$ 等待 API $\to$ 解析 JSON $\to$ 更新图表)时,简单的 \texttt{setState} 容易导致代码混乱(Callback Hell)。我们采用了 \texttt{Provider} 模式进行状态管理。

\begin{itemize}
  \item \textbf{ChangeNotifier}: 作为 ViewModel,持有 \texttt{isLoading}, \texttt{optimizationResult} 等状态。
  \item \texttt{Consumer}: UI 组件只监听它关心的状态变化。例如,当优化结果返回时,只有 \texttt{PowerChartWidget} 会重绘,而底部的导航栏保持静止。这种细粒度的刷新机制显著降低了 GPU 的渲染负载。
\end{itemize}

\subsection{高性能图表渲染优化}
为了展示 24 小时的功率曲线(包含负载、电价、电池动作三条曲线),我们使用了 \texttt{fl\_chart} 库。为了避免数据点过多导致的卡顿,我们实施了以下优化:
\begin{enumerate}
  \item \textbf{数据降维}:后端虽然计算精度为小数点后 10 位,但在传输给前端前保留 2 位小数,减少 JSON 解析开销。
  \item \textbf{按需加载}:图表的 Tooltip(提示框)仅在用户长按时动态生成,而非预先渲染所有点的提示信息。
\end{enumerate}

\section{数据存储与一致性设计}

\subsection{Firestore 索引策略}
NoSQL 数据库虽然灵活,但在进行多字段组合查询时需要显式定义索引。为了支持“查询过去 7 天内特定用户的优化记录”这一高频操作,我们在 Firestore 中配置了复合索引(Composite Index):

\begin{codeblock}{Firestore 索引定义 (firestore.indexes.json)}
  {
  "collectionGroup": "optimizations",
  "queryScope": "COLLECTION",
  "fields": [
  { "fieldPath": "uid", "order": "ASCENDING" },
  { "fieldPath": "date", "order": "DESCENDING" }
  ]
  }
\end{codeblock}
该索引使得查询时间复杂度从 $O(N)$ 降低为 $O(k)$($k$ 为结果集大小),确保了即使在百万级数据量下,历史记录页面的加载时间也恒定在 200ms 以内。

\section{云原生基础设施与 DevOps}

\subsection{Docker 镜像极致优化}
为了加快 GAE 的部署速度并节省存储成本,我们极致优化了 \texttt{Dockerfile}。采用了多阶段构建(Multi-stage Build)策略:

\begin{codeblock}[language=docker]{多阶段构建 Dockerfile 优化}
  # --- Stage 1: Builder ---
  FROM python:3.11-slim AS builder
  # 安装 gcc 等编译工具,用于构建 numpy/gurobipy 的 C 扩展
  RUN apt-get update && apt-get install -y gcc g++ libxml2-dev
  COPY requirements.txt .
  RUN pip install --user -r requirements.txt

  # --- Stage 2: Runtime ---
  FROM python:3.11-slim
  # 仅复制构建好的 Python 包,丢弃编译工具链
  COPY --from=builder /root/.local /root/.local
  WORKDIR /app
  COPY . .
  # 最终镜像不包含 gcc,体积减少 60%
  ENV PATH=/root/.local/bin:$PATH
    CMD ["gunicorn", "main:app"]$
\end{codeblock}

通过这种方式,我们的生产镜像体积从最初的 800MB+ 缩减至 150MB 左右,使得 GAE 的扩容启动时间缩短了约 40\%。

\subsection{自动扩缩容配置}
在 \texttt{app.yaml} 中,我们定义了弹性伸缩策略,以应对潜在的流量爆发:
\begin{codeblock}[language=bash]{GAE 扩缩容配置}
  automatic_scaling:
  min_num_instances: 1   # 保持最小一个实例运行,避免冷启动
  max_num_instances: 10  # 限制最大成本
  cpu_utilization:
  target_utilization: 0.65  # 当 CPU 超过 65% 时自动增加实例
\end{codeblock}
这种 Serverless 的配置确保了系统在夜间低谷期只有 1 个实例运行(极低成本),而在演示或压力测试期间能自动扩展计算能力。

\subsection{CI/CD 流水线设计}
虽然本项目规模较小,但我们依然建立了基础的 CI/CD 意识。通过编写 shell 脚本 (\texttt{scripts/deploy\_backend.sh}),我们自动化了“运行测试 -> 构建镜像 -> 推送 GCR -> 更新 GAE”的发布流程,将手动部署的 10 分钟操作缩短为一键完成,极大地 减少了人为操作失误的风险。


\section{系统与数据安全设计}

作为面向家庭用户的能源管理系统,数据的隐私性与系统的安全性是设计的红线。

\subsection{基于 Firebase 的统一身份认证}
系统采用 \textbf{Firebase Authentication} 托管用户身份验证,支持 OAuth2.0 协议。前端 Flutter 应用通过 SDK 获取 ID Token (JWT),后端 Flask 在请求拦截器 (Interceptor) 中验证 Token 的签名与有效期,确保 API 仅对合法用户开放。

\subsection{细粒度的数据库访问控制 (RBAC)}
为了防止恶意用户越权访问他人数据,我们在 Firestore Security Rules 中实施了严格的行级权限控制:

\begin{codeblock}{Firestore 安全规则示例}
  match /optimizations/{docId} {
  // 仅允许用户读取 create 时 uid 字段等于自己的文档
  allow read, write: if request.auth != null &&
  request.auth.uid == resource.data.uid;
  }
\end{codeblock}

这意味着,即使用户直接通过 REST API 尝试遍历数据库,也会因权限不足被拒绝。

\subsection{API 安全加固}
\begin{itemize}
  \item \textbf{HTTPS 强制加密}:所有通信链路强制使用 TLS 1.2+ 协议。
  \item \textbf{Rate Limiting}:在 Flask 中间件层实现了令牌桶算法,限制单 IP 请求速率(100 req/min),有效防御 DDoS 攻击。
\end{itemize}

\section{本章小结}
本章从系统架构、后端微服务、前端交互、数据存储及云基础设施与 DevOps 等五个维度,详细阐述了本项目的工程实现。可以看出,这不仅仅是一个算法验证 Demo,而是一个具备高内聚低耦合特性、基于现代云原生技术栈构建的完整工程系统。这种工程上的严谨性为算法的稳定运行提供了坚实的机体支撑。


\cleardoublepage
\chapter{总结与展望}

\section{项目核心贡献}

本项目成功验证了“数据驱动+运筹优化”在分布式储能管理中的应用价值,主要贡献包括:

\begin{enumerate}
  \item \textbf{闭环方法论验证}:本项目打通了从“数据清洗 $\to$ 机器学习预测 $\to$ 运筹优化 $\to$ 工程落地”的全链路。这不仅是一次技术实验,更证明了即使在家庭与社区这样微小的单元中,数字化与智能化的力量也能渗透其中,为普通用户带来可观的经济回报(日均节省 12.5\%),让技术红利真正惠及大众。
  \item \textbf{高精度预测模型}:针对建筑能耗数据的波动性,我们构建了融合气象与时间特征的随机森林模型,在测试集上实现了 0.92 的 $R^2$ 得分。这一数字的背后,是我们对物理世界规律的深刻理解与数学建模的精准捕捉,是对无序数据的秩序重构。
  \item \textbf{云原生系统架构}:探索了基于 Docker 和 Kubernetes (GAE) 的科学计算服务部署方案,解决了 Gurobi 等复杂依赖在云端的环境配置与许可证管理问题。这为数据科学走“出”实验室,解决真实世界的复杂工程挑战提供了一份可复用的参考蓝图。
\end{enumerate}

\section{项目局限性}

尽管实现了预期目标,但受限于时间与资源,本项目仍存在局限,这些局限也正是未来探索的起点:

\begin{itemize}
  \item \textbf{简化的电池模型}:优化模型假设充放电效率为常数且忽略了电池日历寿命(Calendar Aging)。在现实的物理世界中,电池如同生命体一般会衰减与老化,如何在算法中注入对“物理生命周期”的敬畏与考量,是长期运行的关键。
  \item \textbf{冷启动问题}:对于新接入的用户,缺乏历史数据将导致预测模型失效。这提示我们需要更具包容性的算法,如通过迁移学习让系统能快速适应新的环境与个体。
  \item \textbf{单目标优化}:当前仅考虑经济成本最小化。然而,能源管理的终极目标不应止步于金钱,电网的稳定性、碳排放的减少同样值得我们关注。
\end{itemize}

\section{未来改进方向}

基于上述局限,未来的工作可从以下维度展开:

\begin{enumerate}
  \item \textbf{引入强化学习 (RL)}:采用 PPO 或 DQN 算法替代两阶段法,像训练棋手一样训练智能体,使其在与环境的交互中直接学习控制策略,更好地应对未来的不确定性。
  \item \textbf{多目标优化与帕累托最优}:在目标函数中加入电池健康度(SOH)与碳排放指标,在经济利益、设备寿命与环境责任之间寻找完美的平衡点。
  \item \textbf{边缘计算与隐私保护}:将模型量化剪枝并部署于边缘设备,让数据“不出家门”即可完成计算,在享受智能便利的同时,给予用户最大的隐私尊重。
  \item \textbf{光储协同与能源自治}:集成光伏(PV)发电预测,实现从“被动响应”到“主动规划”的跨越,推动家庭能源的独立与自治。
\end{enumerate}

\section{结语:技术之外的思考——为“流动的能量”赋予智慧}

在本项目即将画上句号之际,跳出代码与公式的框架,我不禁思考:我们构建这套系统的终极意义何在?

能源,是现代文明的血液,是夜晚点亮书桌的那束光,是寒冬里温暖房间的热流。在很长一段时间里,它被视为一种理所应当的、取之不尽的供给。然而,随着气候变化的警钟敲响,我们意识到这份馈赠并非没有代价。

本项目所探讨的“分布式储能管理”,表面上是几个 Python 脚本、一组数学方程和若干图表,但其本质是\textbf{人类在这个能源日益紧缺的时代,试图用智慧(Intelligence)去换取效率(Efficiency)的一种尝试}。

当我们用随机森林预测下一个小时的负载时,我们预测的不仅是数字,更是人们的生活节奏;当我们用线性规划调度电池在低谷充电时,我们优化的不仅是成本,更是对电网资源的温柔呵护。每一千瓦时的节省,每一次削峰填谷的操作,微观上虽然只是几分钱的收益,但宏观上汇聚起来,便是对减少碳排放、建设绿色家园的一份坚实贡献。

作为数据科学的学习者与实践者,我们常沉迷于 loss 的下降和 accuracy 的提升,但我们不应忘记,\textbf{算法的冷峻之下,应当藏着对人类生存环境的温热关怀}。不管是家庭微网的优化,还是宏观电网的调度,技术的每一次进步,都应该让这个世界变得更加可持续,更加美好。

愿这段代码不仅仅运行在服务器上,更能运行在未来绿色能源变革的浪潮中;愿我们手中的技术,终能成为守护这颗蓝色星球的一份力量。

\clearpage
\cleardoublepage

% 以下是参考文献部分
\phantomsection
\addcontentsline{toc}{chapter}{参考文献}
\nocite{*}
\bibliographystyle{plainnat}
\bibliography{czu_templates/refs}

\appendix

\cleardoublepage
\chapter{附录}

\section{核心依赖配置}

\subsection{后端依赖 (requirements.txt)}
\begin{codeblock}{requirements.txt}
  Flask==3.0.0
  gunicorn==21.2.0
  numpy>=1.26.2
  pandas>=1.5.3
  scikit-learn==1.3.2
  gurobipy==10.0.3
  firebase-admin==6.5.0
  pytest==7.4.3
  shap==0.44.1
  apscheduler==3.10.4
  requests>=2.29.0
\end{codeblock}

\subsection{前端依赖 (pubspec.yaml)}
\begin{codeblock}{pubspec.yaml}
  dependencies:
  flutter:
  sdk: flutter
  http: ^1.1.0
  firebase_core: ^4.2.1
  firebase_auth: ^6.1.2
  google_sign_in: ^6.2.1
  fl_chart: ^1.1.1
  percent_indicator: ^4.2.3
  file_picker: ^8.0.0+1
  intl: ^0.20.2
\end{codeblock}

\section{关键算法参数}

\subsection{随机森林超参数}
\begin{itemize}
  \item \texttt{n\_estimators}: 100
  \item \texttt{max\_depth}: None (自动)
  \item \texttt{min\_samples\_split}: 2
  \item \texttt{min\_samples\_leaf}: 1
  \item \texttt{bootstrap}: True
\end{itemize}

\section{核心算法与服务实现代码}

\subsection{机器学习预测服务 (ml\_service.py)}
以下代码展示了 \texttt{EnergyPredictor} 类的实现,包括数据加载、特征工程与随机森林模型推理。

\begin{codeblock}[language=Python]{ml\_service.py 完整代码}
  class EnergyPredictor:
  def __init__(self, model_path: str = None):
  """初始化预测器,加载预训练模型"""
  self.firebase_model_path = 'models/rf_model.joblib'
  if model_path:
  self.local_model_path = Path(model_path)
  else:
  self.local_model_path = self.back_dir / 'models' / 'rf_model.joblib'

  self.storage_service = StorageService()
  self.model = None

  def predict_next_24h(self, start_time: datetime, temp_forecast_list: List[float]) -> List[Dict]:
  """执行未来24小时负载预测"""
  if self.model is None:
  self.load_model()  # 懒加载

  # 1. 构造特征矩阵 X
  prediction_data = []
  for i in range(24):
  dt = start_time + timedelta(hours=i)
  prediction_data.append({
  'Hour': dt.hour,
  'DayOfWeek': dt.weekday(),
  'Temperature': temp_forecast_list[i],
  'Price': self._get_price(dt.hour)
  })

  X_pred = pd.DataFrame(prediction_data)

  # 2. 执行批量推理
  predictions = self.model.predict(X_pred)

  # 3. 结果格式化
  results = []
  for i, pred_load in enumerate(predictions):
  results.append({
  'datetime': start_time + timedelta(hours=i),
  'predicted_load': float(pred_load),
  'temperature': temp_forecast_list[i],
  'price': prediction_data[i]['Price']
  })
  return results
\end{codeblock}

\subsection{储能优化调度服务 (optimization\_service.py)}
以下代码展示了基于 Gurobi 的混合整数规划 (MIP) 建模过程。

\begin{codeblock}[language=Python]{optimization\_service.py 完整代码}
  class EnergyOptimizer:
  def optimize_schedule(self, load_profile, price_profile, initial_soc=0.5):
  """MIP 优化调度主入口"""
  if self.env is None:
  self.env = self._create_gurobi_env()

  model = gp.Model("BatteryScheduling", env=self.env)
  T = 24

  # 1. 决策变量
  # 充放电功率 (kW)
  P_charge = model.addVars(T, lb=0, ub=self.max_power, name="P_charge")
  P_discharge = model.addVars(T, lb=0, ub=self.max_power, name="P_discharge")
  # 电池存储电量 (kWh)
  E_stored = model.addVars(T, lb=0, ub=self.battery_capacity, name="E_stored")
  # 状态互斥二进制变量
  Is_charge = model.addVars(T, vtype=GRB.BINARY, name="Is_charge")
  Is_discharge = model.addVars(T, vtype=GRB.BINARY, name="Is_discharge")

  # 2. 约束条件
  for t in range(T):
  # 状态互斥:同一时刻不能同时充放电
  model.addConstr(Is_charge[t] + Is_discharge[t] <= 1)

  # 功率限制 (Big-M 约束)
  model.addConstr(P_charge[t] <= self.max_power * Is_charge[t])
  model.addConstr(P_discharge[t] <= self.max_power * Is_discharge[t])

  # 能量守恒 (Battery Dynamics)
  prev_E = E_stored[t-1] if t > 0 else initial_soc * self.battery_capacity
  model.addConstr(
  E_stored[t] == prev_E + self.efficiency * P_charge[t]
  - P_discharge[t] / self.efficiency
  )


  # 3. 目标函数: 最小化购电成本
  grid_cost = sum(price_profile[t] * (load_profile[t] + P_charge[t] - P_discharge[t])
  for t in range(T))
  model.setObjective(grid_cost, GRB.MINIMIZE)

  model.optimize()
  return self._format_result(model)
\end{codeblock}









\subsection{储能优化核心逻辑 (optimization\_service.py)}
\begin{codeblock}[language=Python]{储能优化核心逻辑 (optimization\_service.py)}
  def optimize_schedule(
  self,
  load_profile: List[float],
  price_profile: List[float],
  initial_soc: float = 0.5
  ) -> Dict:
  """
  优化电池充放电调度

  Args:
  load_profile: 未来24小时的负载预测 (kW)
  price_profile: 未来24小时的电价 (元/kWh)
  initial_soc: 初始电池电量百分比 (0.0-1.0)

  Returns:
  优化结果字典,包含:
  - status: 求解状态
  - schedule: 详细调度计划
  - total_cost_without_battery: 无电池时的总成本
  - total_cost_with_battery: 有电池时的总成本
  - savings: 节省金额

  Raises:
  ValueError: 输入参数错误
  Exception: 优化失败
  """
  print("\n" + "="*80)
  print(" 开始优化电池调度")
  print("="*80 + "\n")

  # 验证输入
  if len(load_profile) != 24:
  raise ValueError(f"负载数据长度必须为24,当前为 {len(load_profile)}")

  if len(price_profile) != 24:
  raise ValueError(f"电价数据长度必须为24,当前为 {len(price_profile)}")

  if not 0 <= initial_soc <= 1:
  raise ValueError(f"初始SOC必须在 [0, 1] 范围内,当前为 {initial_soc}")

  # 转换为 numpy 数组
  load = np.array(load_profile)
  price = np.array(price_profile)

  print(f" 输入数据:")
  print(f"   - 负载范围: {load.min():.2f} - {load.max():.2f} kW")
  print(f"   - 电价范围: {price.min():.2f} - {price.max():.2f} 元/kWh")
  print(f"   - 初始 SOC: {initial_soc * 100:.1f}%")

  # 计算无电池时的总成本
  cost_without_battery = np.sum(load * price)
  print(f"   - 无电池总成本: {cost_without_battery:.2f} 元")

  try:
  # 创建 Gurobi 环境
  if self.env is None:
  self.env = self._create_gurobi_env()

  # 创建模型
  print(f"\n️  构建优化模型...")
  model = gp.Model("BatteryScheduling", env=self.env)
  model.setParam('OutputFlag', 0)  # 关闭求解器输出

  T = 24  # 时间步数

  # 决策变量
  print(f"   - 创建决策变量...")

  # 充电功率 (kW)
  P_charge = model.addVars(T, lb=0, ub=self.max_power, name="P_charge")

  # 放电功率 (kW)
  P_discharge = model.addVars(T, lb=0, ub=self.max_power, name="P_discharge")

  # 电池存储电量 (kWh)
  E_stored = model.addVars(T, lb=0, ub=self.battery_capacity, name="E_stored")

  # 二进制变量: 是否充电
  Is_charge = model.addVars(T, vtype=GRB.BINARY, name="Is_charge")

  # 二进制变量: 是否放电
  Is_discharge = model.addVars(T, vtype=GRB.BINARY, name="Is_discharge")

  print(f"   ✓ 变量数量: {T * 5} 个")

  # 约束条件
  print(f"   - 添加约束条件...")

  # 1. 状态互斥约束: 不能同时充放电
  for t in range(T):
  model.addConstr(
  Is_charge[t] + Is_discharge[t] <= 1,
  name=f"mutex_{t}"
  )

  # 2. 功率限制约束
  for t in range(T):
  # 充电功率限制
  model.addConstr(
  P_charge[t] <= self.max_power * Is_charge[t],
  name=f"charge_limit_{t}"
  )

  # 放电功率限制
  model.addConstr(
  P_discharge[t] <= self.max_power * Is_discharge[t],
  name=f"discharge_limit_{t}"
  )

  # 3. 能量守恒约束 (电池动态方程)
  initial_energy = initial_soc * self.battery_capacity

  for t in range(T):
  if t == 0:
  # 初始时刻
  model.addConstr(
  E_stored[t] == initial_energy +
  P_charge[t] * self.efficiency -
  P_discharge[t] / self.efficiency,
  name=f"energy_balance_{t}"
  )
  else:
  # 后续时刻
  model.addConstr(
  E_stored[t] == E_stored[t-1] +
  P_charge[t] * self.efficiency -
  P_discharge[t] / self.efficiency,
  name=f"energy_balance_{t}"
  )

  print(f"   ✓ 约束数量: {T * 4} 个")

  # 目标函数: 最小化总购电成本
  print(f"   - 设置目标函数...")

  total_cost = gp.quicksum(
  (load[t] + P_charge[t] - P_discharge[t]) * price[t]
  for t in range(T)
  )

  model.setObjective(total_cost, GRB.MINIMIZE)
  print(f"   ✓ 目标: 最小化总购电成本")

  # 求解模型
  print(f"\n 开始求解...")
  model.optimize()

  # 检查求解状态
  status = model.status

  if status == GRB.OPTIMAL:
  print(f"   ✓ 求解成功! (状态: OPTIMAL)")

  # 提取结果
  schedule = []

  for t in range(T):
  p_charge = P_charge[t].X
  p_discharge = P_discharge[t].X
  e_stored = E_stored[t].X
  soc = e_stored / self.battery_capacity

  # battery_action: 正值为充电,负值为放电
  battery_action = p_charge - p_discharge

  schedule.append({
      'hour': t,
      'load': float(load[t]),
      'price': float(price[t]),
      'battery_action': float(battery_action),
      'charge_power': float(p_charge),
      'discharge_power': float(p_discharge),
      'soc': float(soc),
      'stored_energy': float(e_stored)
    })

  # 计算总成本
  cost_with_battery = model.objVal
  savings = cost_without_battery - cost_with_battery
  savings_percent = (savings / cost_without_battery) * 100 if cost_without_battery > 0 else 0

  print(f"\n 优化结果:")
  print(f"   - 无电池总成本: {cost_without_battery:.2f} 元")
  print(f"   - 有电池总成本: {cost_with_battery:.2f} 元")
  print(f"   - 节省金额: {savings:.2f} 元 ({savings_percent:.1f}%)")

  return {
      'status': 'Optimal',
      'schedule': schedule,
      'total_cost_without_battery': float(cost_without_battery),
      'total_cost_with_battery': float(cost_with_battery),
      'savings': float(savings),
      'savings_percent': float(savings_percent)
    }

  elif status == GRB.INFEASIBLE:
  print(f"   ❌ 模型不可行 (INFEASIBLE)")
  return {
      'status': 'Infeasible',
      'error': '模型约束不可行,请检查输入参数'
    }

  elif status == GRB.UNBOUNDED:
  print(f"   ❌ 模型无界 (UNBOUNDED)")
  return {
      'status': 'Unbounded',
      'error': '模型目标函数无界'
    }

  else:
  print(f"   ⚠️  求解未完成 (状态码: {status})")
  return {
  'status': 'Unknown',
  'error': f'求解状态未知 (状态码: {status})'
  }

  except gp.GurobiError as e:
  error_msg = str(e)

  if "license" in error_msg.lower():
  print(f"\n❌ Gurobi 许可证错误")
  return {
      'status': 'Error',
      'error': 'Optimization failed: Gurobi license not found'
    }
  else:
  print(f"\n❌ Gurobi 错误: {error_msg}")
  return {
  'status': 'Error',
  'error': f'Gurobi error: {error_msg}'
  }

  except Exception as e:
  print(f"\n❌ 优化失败: {str(e)}")
  return {
      'status': 'Error',
      'error': str(e)
    }
\end{codeblock}

\subsection{机器学习训练流程 (ml\_service.py)}
\begin{codeblock}[language=Python]{机器学习训练流程 (ml\_service.py)}
  def train_model(
  self,
  data_path: str = None,
  n_estimators: int = 100,
  test_size: float = 0.2,
  random_state: int = 42,
  use_firebase_storage: bool = True
  ) -> Dict[str, float]:
  """
  训练随机森林模型

  Args:
  data_path: 数据文件路径,默认为 data/processed/cleaned_energy_data_all.csv
  n_estimators: 随机森林树的数量
  test_size: 测试集比例
  random_state: 随机种子
  use_firebase_storage: 是否从 Firebase Storage 下载数据 (GAE 环境必须为 True)

  Returns:
  包含评估指标的字典 (MAE, RMSE)
  """
  print("\n" + "="*80)
  print(" 开始训练能源负载预测模型")
  print("="*80 + "\n")

  temp_data_path = None

  try:
  # 从 Firebase Storage 下载数据
  if use_firebase_storage:
  print(" 从 Firebase Storage 下载训练数据...")
  from services.storage_service import StorageService

  storage_service = StorageService()
  firebase_path = data_path or 'data/processed/cleaned_energy_data_all.csv'

  temp_data_path = storage_service.download_to_temp(firebase_path)

  if temp_data_path is None:
  raise FileNotFoundError(f"无法从 Firebase Storage 下载数据: {firebase_path}")

  data_path = temp_data_path
  print(f"   ✓ 数据已下载到: {data_path}")
  else:
  # 本地文件模式 (开发环境)
  if data_path is None:
  # 尝试多个可能的路径
  possible_paths = [
  self.back_dir.parent / 'data' / 'processed' / 'cleaned_energy_data_all.csv',
  self.back_dir / 'data' / 'processed' / 'cleaned_energy_data_all.csv',
  ]
  data_path = None
  for path in possible_paths:
  if path.exists():
  data_path = path
  break
  if data_path is None:
  data_path = possible_paths[0]  # 使用第一个作为默认
  else:
  data_path = Path(data_path)

  # 读取数据
  print(f" 读取数据: {data_path}")
  try:
  df = pd.read_csv(data_path, parse_dates=['Date'])
  print(f"   ✓ 数据读取成功: {len(df)} 行 × {len(df.columns)} 列")
  except FileNotFoundError:
  raise FileNotFoundError(f"数据文件不存在: {data_path}")
  except Exception as e:
  raise Exception(f"读取数据时出错: {str(e)}")

  # 检查必需列
  required_cols = self.feature_columns + [self.target_column]
  missing_cols = [col for col in required_cols if col not in df.columns]
  if missing_cols:
  raise ValueError(f"数据缺少必需列: {missing_cols}")

  # 处理缺失值
  print(f"\n 检查数据质量...")
  null_counts = df[required_cols].isnull().sum()
  if null_counts.sum() > 0:
  print(f"   ⚠️  发现缺失值:")
  for col, count in null_counts[null_counts > 0].items():
  print(f"      - {col}: {count} 个")

  # 对于 Temperature,使用均值填充
  if 'Temperature' in null_counts and null_counts['Temperature'] > 0:
  mean_temp = df['Temperature'].mean()
  df['Temperature'].fillna(mean_temp, inplace=True)
  print(f"   ✓ Temperature 缺失值已用均值填充: {mean_temp:.2f}°C")
  else:
  print(f"   ✓ 无缺失值")

  # 准备特征和目标变量
  print(f"\n 准备训练数据...")
  X = df[self.feature_columns].copy()
  y = df[self.target_column].copy()

  print(f"   - 特征列: {self.feature_columns}")
  print(f"   - 目标变量: {self.target_column}")
  print(f"   - 数据形状: X={X.shape}, y={y.shape}")

  # 划分训练集和测试集
  print(f"\n✂️  划分数据集 (训练集: {int((1-test_size)*100)}%, 测试集: {int(test_size*100)}%)...")
  X_train, X_test, y_train, y_test = train_test_split(
  X, y, test_size=test_size, random_state=random_state
  )

  print(f"   - 训练集: {X_train.shape[0]} 样本")
  print(f"   - 测试集: {X_test.shape[0]} 样本")

  # 训练模型
  print(f"\n 训练随机森林模型 (n_estimators={n_estimators})...")
  self.model = RandomForestRegressor(
  n_estimators=n_estimators,
  random_state=random_state,
  n_jobs=-1,  # 使用所有CPU核心
  verbose=0
  )

  self.model.fit(X_train, y_train)
  print(f"   ✓ 模型训练完成!")

  # 评估模型
  print(f"\n 评估模型性能...")

  # 训练集预测
  y_train_pred = self.model.predict(X_train)
  train_mae = mean_absolute_error(y_train, y_train_pred)
  train_rmse = np.sqrt(mean_squared_error(y_train, y_train_pred))

  # 测试集预测
  y_test_pred = self.model.predict(X_test)
  test_mae = mean_absolute_error(y_test, y_test_pred)
  test_rmse = np.sqrt(mean_squared_error(y_test, y_test_pred))

  print(f"\n   训练集性能:")
  print(f"      - MAE:  {train_mae:.2f} kW")
  print(f"      - RMSE: {train_rmse:.2f} kW")

  print(f"\n   测试集性能:")
  print(f"      - MAE:  {test_mae:.2f} kW")
  print(f"      - RMSE: {test_rmse:.2f} kW")

  # 特征重要性
  print(f"\n 特征重要性:")
  feature_importance = pd.DataFrame({
      'Feature': self.feature_columns,
      'Importance': self.model.feature_importances_
    }).sort_values('Importance', ascending=False)

  for _, row in feature_importance.iterrows():
  print(f"      - {row['Feature']}: {row['Importance']:.4f}")

  # 保存模型到 Firebase Storage
  print(f"\n 保存模型到 Firebase Storage: {self.firebase_model_path}")
  temp_model_path = None
  try:
  # Step A: 创建临时文件
  with tempfile.NamedTemporaryFile(mode='wb', suffix='.joblib', delete=False) as tmp_file:
  temp_model_path = tmp_file.name
  print(f"   - 临时文件: {temp_model_path}")

  # Step B: 保存模型到临时文件
  joblib.dump(self.model, temp_model_path)
  print(f"   ✓ 模型已保存到临时文件")

  # Step B-2: (新增) 保存模型到本地持久化路径 (用于开发环境调试)
  try:
  # 确保存储目录存在
  self.local_model_path.parent.mkdir(parents=True, exist_ok=True)
  joblib.dump(self.model, self.local_model_path)
  print(f"   ✓ 模型已备份到本地路径: {self.local_model_path}")
  except Exception as local_e:
  print(f"   ⚠️  无法保存本地模型副本: {str(local_e)}")

  # Step C: 上传到 Firebase Storage
  with open(temp_model_path, 'rb') as f:
  self.storage_service.upload_file(
  file_data=f,
  destination_path=self.firebase_model_path,
  content_type='application/octet-stream'
  )
  print(f"   ✓ 模型已上传到 Firebase Storage")

  except Exception as e:
  print(f"   ❌ 模型保存失败: {str(e)}")
  raise
  finally:
  # Step D: 清理临时文件
  if temp_model_path and os.path.exists(temp_model_path):
  try:
  os.remove(temp_model_path)
  print(f"    已清理临时模型文件")
  except Exception as e:
  print(f"   ⚠️  清理临时模型文件失败: {str(e)}")

  print("\n" + "="*80)
  print("✅ 模型训练完成!")
  print("="*80 + "\n")

  # 保存模型元数据到 Firestore (全局元数据)
  try:
  self._save_model_metadata({
  'model_type': 'Random Forest Regressor',
  'model_version': datetime.now().strftime('%Y%m%d_%H%M%S'),
  'trained_at': datetime.now().isoformat(),
  'metrics': {
  'train_mae': float(train_mae),
  'train_rmse': float(train_rmse),
  'test_mae': float(test_mae),
  'test_rmse': float(test_rmse)
  },
  'training_samples': len(df),
  'data_source': 'CAISO Real-Time Stream',
  'feature_importance': feature_importance.to_dict('records'),
  'model_path': self.firebase_model_path,
  'status': 'active'
  })
  except Exception as e:
  print(f"   ⚠️  保存模型元数据失败: {str(e)}")

  # 返回评估指标
  return {
      'train_mae': train_mae,
      'train_rmse': train_rmse,
      'test_mae': test_mae,
      'test_rmse': test_rmse,
      'feature_importance': feature_importance.to_dict('records')
    }

  finally:
  # 清理临时文件
  if temp_data_path and os.path.exists(temp_data_path):
  try:
  os.remove(temp_data_path)
  print(f" 清理临时训练数据文件")
  except Exception as e:
  print(f"⚠️  清理临时文件失败: {str(e)}")
\end{codeblock}

\subsection{前端优化请求 (api\_service.dart)}
\begin{codeblock}[language=Java]{前端优化请求 (api\_service.dart)}
  /// 执行能源优化调度
  ///
  /// 参数:
  ///   - [initialSoc]: 初始电池电量 (0.0-1.0),默认 0.5
  ///   - [targetDate]: 目标日期,默认为明天
  ///   - [temperatureForecast]: 24小时温度预测列表,可选
  ///   - [batteryCapacity]: 电池容量 (kWh),可选
  ///   - [batteryPower]: 最大功率 (kW),可选
  ///   - [batteryEfficiency]: 充放电效率,可选
  ///
  /// 返回: OptimizationResponse 对象
  ///
  /// 抛出异常:
  ///   - Exception: 当认证失败、参数错误或服务器错误时
  static Future<OptimizationResponse> runOptimization({
      double initialSoc = AppConstants.defaultInitialSoc,
      DateTime? targetDate,
      List<double>? temperatureForecast,
      double? batteryCapacity,
      double? batteryPower,
      double? batteryEfficiency,
    }) async {
  // 验证 initialSoc 范围
  if (initialSoc < AppConstants.minSoc || initialSoc > AppConstants.maxSoc) {
      throw ArgumentError(
      'initialSoc must be between ${AppConstants.minSoc} and ${AppConstants.maxSoc}',
      );
    }

  // 验证温度预测列表长度
  if (temperatureForecast != null && temperatureForecast.length != 24) {
      throw ArgumentError('temperatureForecast must contain exactly 24 values');
    }

  try {
  // 获取认证头
  final headers = await _getAuthHeaders();

  // 构建请求体
  final body = <String, dynamic>{
      'initial_soc': initialSoc,
    };

  // 添加可选参数
  if (targetDate != null) {
      body['target_date'] = _formatDate(targetDate);
    }

  if (temperatureForecast != null) {
      body['temperature_forecast'] = temperatureForecast;
    }

  if (batteryCapacity != null) {
      body['battery_capacity'] = batteryCapacity;
    }

  if (batteryPower != null) {
      body['battery_power'] = batteryPower;
    }

  if (batteryEfficiency != null) {
      body['battery_efficiency'] = batteryEfficiency;
    }

  // 发送 POST 请求
  final response = await http
  .post(
  Uri.parse('$_baseUrl/api/optimization/run'),
  headers: headers,
  body: jsonEncode(body),
  )
  .timeout(
  AppConstants.optimizationTimeout,
  onTimeout: () {
      throw Exception('优化请求超时,请稍后重试');
    },
  );

  // 解析响应
  final data = jsonDecode(response.body) as Map<String, dynamic>;

  // 处理不同的状态码
  if (response.statusCode == 200) {
      // 成功响应
      return OptimizationResponse.fromJson(data);
    } else if (response.statusCode == 401) {
  // 认证失败
  throw Exception(
  data['message'] ?? AppConstants.authError,
  );
\end{codeblock}

\end{document}
